%Observaci�n: Si bien la p�gina del prefacio dice que sea empty, deber�a comenzar alli la numeraci�n. Se sugiere numeraci�n romana. Recomenzar la numeraci�n en el primer cap�tulo de la tesis  con numeraci�n ar�biga.

\titlepage

\begin{center}
\ \\
\ \\
\vspace{-1cm}
 

\ \\

\vspace{0.5cm}
{\Large{\bf \sc Universidad Nacional del Comahue}}\\

\ \\
{\Large { \sc Facultad de Inform�tica}}\\

\vspace{-2.5cm}
\mbox{\hspace{-1cm}\includegraphics[width=2.5cm,height=2.5cm]{unc.png}\hspace{13cm} \includegraphics[width=2.5cm,height=2.5cm]{fai.png}}


\vspace{6cm}

{\Large {\bf\sc Tesis de Licenciado en Ciencias de la Computaci�n}}\\
\ \\
\ \\
{\LARGE {\bf Planificaci\'on Continua mediante PDDL}}\\
\vspace{3cm}


{\Large Germ�n Alejandro Braun}\\
\vspace{2cm}

{\Large Director: Mg. Gerardo Parra}\\
\ \\
%{\Large [Nombre del CoDirector]}\\

\vfill
{\Large {\sc Neuqu�n}\hspace{6cm}{\sc Argentina}}\\
\ \\

{\Large 2012}\\

\end{center}

\pagebreak

