
\chapter{Conclusiones} \label{pagcap7}

El trabajo principal de esta tesis estuvo motivado por la necesidad de dotar
al Framework de Planificaci\'on Continua de un m\'odulo traductor
para el lenguaje PDDL. Este objetivo principal ha ido evolucionando durante el transcurso de
este trabajo y, en consecuencia, nos ha 
permitido tambi\'en concluir una serie de resultados te\'oricos y
pr\'acticos complementarios a la implementaci\'on
propuesta.

La investigaci\'on realizada nos ha posibilitado
implementar un traductor para un subconjunto
de PDDL y, de esta manera, facultar a un agente para que pueda percibir y actuar mediante
especificaciones en este lenguaje. Con este aporte, dicho agente puede disponer de un conjunto de acciones con
el nivel de expresividad y abstracci\'on necesario para operar en ambientes m\'as
complejos. Bajo este novedoso enfoque, es esperable poder abordar
problemas de planificaci\'on sobre cualquier dominio real
que requiera la intervenci\'on de agentes inteligentes. 

La presente tesis comenz\'o, en el cap\'itulo \ref{pagcap2}, con un an\'alisis del lenguaje
PDDL y sus caracter\'isticas principales. En el cap\'itulo siguiente, se present\'o el Framework de
Planificaci\'on Continua desarrollado en \cite{gbraun:tesisMarioMoya}. Tambi\'en se
estudi\'o el algoritmo de planificaci\'on continua implementado en
dicho framework y se analiz\'o su lenguaje de representaci\'on. 
Luego, en el cap\'itulo \ref{pagcap4}, se es\-tu\-dia\-ron algunas implementaciones existentes
y se remarcaron las diferencias con respecto a nuestra propuesta. En el cap\'itulo \ref{pagcap5} se
defini\'o el marco te\'orico subyacente introduciendo los conceptos de
esquemas de compilaci\'on y compilabilidad. Adem\'as, se
definieron y ejemplificaron algunas va\-rian\-tes PDDL que se mapearon a
STRIPS mediante los esquemas de compilaci\'on correspondientes. Por \'ultimo, en el
cap\'itulo \ref{pagcap6}, se analiz\'o la soluci\'on propuesta
y se trabaj\'o sobre un ejemplo real a fin de ilustrar c\'omo escribir
PDDL en Ciao Prolog y c\'omo planificar con estas especificaciones.


\section{Resultados y Contribuciones}

A continuaci\'on, resumimos los principales resultados y
contribuciones de esta tesis.

\begin{itemize}

\item Se formaliz\'o y demostr\'o una nueva variante del lenguaje
STRIPS llamada STRIPS$_{\emph{u}}$. Esta variante surge de adicionar cuantificaci\'on universal 
a la definici\'on de precondiciones para las acciones del dominio de
planificaci\'on. 

\item Se enunciaron las dem\'as variantes de STRIPS: STRIPS$_{\emph{C}}$, STRIPS$_{D}$ y
STRIPS$_{L}$ y se definieron notaciones equivalentes en PDDL que se denotaron
PDDL$_{\emph{C}}$, PDDL$_{D}$ y PDDL$_{L}$, respectivamente. 
Estas variantes de PDDL conforman el lenguaje fuente del traductor implementado.

\item La abstracci\'on lograda, mediante las variantes de PDDL, posibilita
la especificaci\'on de problemas de planificaci\'on complejos.

\item El producto final consta de un m\'odulo traductor del lenguaje fuente
definido hacia un lenguaje destino. La implementaci\'on ofrece una
representaci\'on gen\'erica que permite adaptar este
m\'odulo a otros planificadores cuyo lenguaje de representaci\'on
tiene una estructura similar a STRIPS.

\item En particular, la integraci\'on del m\'odulo traductor al
Framework, presentado por Moya en \cite{gbraun:tesisMarioMoya}, 
permite redefinir el sistema de creencias del agente de
planificaci\'on continua. Esta nueva arquitectura posibilita la
definici\'on, en PDDL, de las percepciones y las acciones del agente.

\item El traductor es una expansi\'on sint\'actica para Ciao
Prolog. Esta novedosa extensi\'on genera una representaci\'on STRIPS,
con sintaxis Prolog, equivalente a una entrada
especificada en un subconjunto del lenguaje PDDL. Hasta el momento de presentaci\'on de
esta tesis, el sistema Ciao no ofrece ning\'un soporte para la
manipulaci\'on del lenguaje PDDL.

\end{itemize}

\section{Trabajo Futuro}

Esta primera etapa de la investigaci\'on abarca un subconjunto
acotado del lenguaje PDDL. Se espera poder ampliar el lenguaje fuente
de este traductor incluyendo otros requerimientos y definiendo los 
esquemas de compilaci\'on asociados. De la misma manera, se espera poder
realizar un an\'alisis m\'as exhaustivo de la complejidad de la
implementaci\'on de manera tal que las traducciones propuestas tengan
el menor impacto posible en la performance de los planificadores.

Uno de los aspectos pendientes es la implementaci\'on de una
interface para el Framework de Planificaci\'on
Continua. Se espera poder aplicarlo a dominios reales y, por lo 
tanto, es necesario realizar traducciones, en tiempo
real, de las percepciones del agente.

Finalmente, planteamos dos desaf\'ios. Por un lado, con el objetivo de 
mejorar el traductor, proponemos la aplicaci\'on de otros conceptos 
de Compiladores e Int\'erpretes, como la manipulaci\'on y
recuperaci\'on de errores \cite{gbraun:Aho:2007}. Por \'ultimo,
proponemos combinar el traductor con otros planificadores implementados
en Ciao Prolog y realizar comparaciones de performance entre ellos.


