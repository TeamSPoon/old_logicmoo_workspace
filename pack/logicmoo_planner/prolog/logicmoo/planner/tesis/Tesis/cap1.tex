
\chapter{Introducci\'on} \label{pagcap1}

Esta tesis aborda dos t\'opicos complementarios implicados en la
resoluci\'on de problemas utilizando agentes inteligentes: la
planificaci\'on continua y los lenguajes de representaci\'on de problemas.

Russell y Norving \cite{gbraun:Rus09} definen a la planificaci\'on como la tarea de
obtener una secuencia de acciones que pueden ser aplicadas a un
conjunto discreto de estados para lograr una meta. Puntualmente, la
planificaci\'on continua est\'a orientada a resolver problemas en
ambientes din\'amicos y, por lo tanto, se encarga de atacar problemas en
ambientes reales.

La complejidad inherente a este tipo de planificaci\'on y a los ambientes
reales, implica trabajar con otros lenguajes de
representaci\'on de problemas. Un lenguaje de representaci\'on debe 
permitir la definici\'on de estados, acciones y metas y, adem\'as, 
posibilitar que los algoritmos de planificaci\'on puedan obtener 
ventajas de la estructura l\'ogica de los problemas. Esta necesidad de un lenguaje con un poder expresivo mucho mayor que los existentes 
hasta el momento, impuls\'o  el desarrollo de un novedoso lenguaje est\'andar llamado PDDL (\emph{Planning Domain Definition Language}) \cite{mder:pddl}. 
PDDL es un lenguaje centrado en acciones y est\'a inspirado 
en la formulaci\'on de problemas de planificaci\'on en STRIPS. La
importancia de PDDL radica en el nivel de abs\-trac\-ci\'on y la
expresividad para modelar dominios y problemas m\'as
complejos. Adem\'as, trabajar con un lenguaje est\'andar permite
realizar comparaciones de performance entre diferentes algoritmos planificadores.

La principal motivaci\'on de esta investigaci\'on es dotar al Planificador Continuo,
presentado por Moya en \cite{gbraun:tesisMarioMoya}, de un m\'odulo
traductor del lenguaje PDDL con el objetivo de lograr que el sistema
de creencias de un agente soporte percepciones y acciones
especificadas en dicho len\-gua\-je. Esto nos posibilitar\'a combinar la
expresividad de PDDL con el Planificador Continuo y modelar dominios
de planificaci\'on con un nivel de abstracci\'on necesario en el trato
con ambientes reales y din\'amicos.
La presentaci\'on preliminar de esta investigaci\'on 
fue publicada en \cite{gbraun:wicc2011}.

Como uno de los resultados principales de esta tesis, se presenta un 
m\'odulo traductor para un subconjunto del lenguaje PDDL cuyo lenguaje 
destino es similar a STRIPS. La traducci\'on pro\-pues\-ta involucra un an\'alisis
exhaustivo de PDDL y la definici\'on de esquemas de compilaci\'on que
permitan obtener la representaci\'on STRIPS equivalente a las
especificaciones PDDL de entrada. Adem\'as, se abordan diferentes
variantes STRIPS, presentadas en la literatura de Inteligencia
Artificial (IA), que surjen
de la adici\'on de caracter\'isticas m\'as expresivas al STRIPS
est\'andar. En este contexto te\'orico y como un resultado
complementario de esta investigaci\'on, se formaliza 
una nueva variante que involucra el uso de cuantificaci\'on universal.

El traductor es implementado como una expansi\'on
sint\'actica de Ciao Prolog. Esta soluci\'on ofrece una notaci\'on
gen\'erica Prolog que permite adaptar el m\'odulo a diferentes
planificadores cuyo lenguaje de representaci\'on sea similar a
STRIPS. Para ilustrar los resultados obtenidos, se muestra el uso del 
traductor en el dominio del ``Mundo de Bloques'' y c\'omo un
planificador, cuyo lenguaje de representaci\'on es STRIPS, resuelve un problema especificado
inicialmente en PDDL. 


\section{Trabajos Relacionados}

Algunos resultados preliminares han formado parte de las siguientes
publicaciones:

\begin{itemize}
%\subsubsection{Publicaciones}
%\noindent
\item Braun, G., Moya, M. y Parra, G. 
Sistemas Multiagentes en Ambientes Din\'amicos: Planificaci\'on Continua mediante PDDL. 
En \emph{XIII Workshop de Investigaci\'on en Ciencias de la Computaci\'on}, Rosario, Santa F\'e. Universidad Nacional de
Rosario. 2011.

%\subsubsection{Trabajos Presentados}
%\noindent
\item Braun, G. y Parra, G. Multiagent Systems in Dynamic Environments: Continuous Planning using PDDL. 
Enviado a las \emph{41 Jornadas Argentinas de Inform\'atica}, La Plata, Buenos Aires. Universidad Nacional de La
Plata. 2012.

%Moya-Vaucheret 1
\item Moya, M. y Vaucheret, C. Agentes Deliberativos Basados en Planficaci\'on Continua. 
En \emph{X Workshop Agentes y Sistemas Inteligentes (WASI)}, S.S.
de Jujuy. Universidad Nacional de Jujuy. Facultad de Ingenier�a. 2009.

%Moya-Vaucheret 2
\item Moya, M. y Vaucheret, C. Planificador Continuo como Controlador de Agentes Robots. 
En \emph{X Workshop de Investigadores en Ciencias de la Computaci\'on}, 
General Pico, La Pampa, Argentina. Universidad Nacional de la
Pampa. 2008.

\end{itemize}


\section{Estructura de la Tesis}

En el cap\'itulo \ref{pagcap2} se presenta un an\'alisis del lenguaje PDDL,
c\'omo ha sido su evoluci\'on y cu\'ales son sus caracter\'isticas m\'as
relevantes. El cap\'itulo finaliza con una descripci\'on
de los requerimientos del lenguaje considerados para ser implementados
en el traductor propuesto.

En el cap\'itulo \ref{pagcap3} se introduce el Framework de
Planificaci\'on Continua, se analizan sus ca\-rac\-te\-r\'is\-ti\-cas
principales y su implementaci\'on. Adem\'as, se concluye con la
especificaci\'on de una nueva arquitectura para el framework.

Las implementaciones existentes, hasta el momento de presentaci\'on de
este trabajo, se de\-ta\-llan en el cap\'itulo \ref{pagcap4}. Para cada una de las tres
implementaciones abordadas, se consideran las caracter\'isticas PDDL que
soportan, el ambiente de programaci\'on en el que est\'an
desarrolladas, la especificaci\'on del lenguaje de salida y las
licencias correspondientes. En todos los casos, se remarcan las
diferencias con nuestra propuesta.

En el cap\'itulo \ref{pagcap5} se define el marco
te\'orico subyacente, se estudian variantes de los lenguajes STRIPS y
PDDL y se especifican esquemas de compilaci\'on que luego ser\'an 
implementados en el traductor.

Posteriormente, en el cap\'itulo \ref{pagcap6}, se detalla la
arquitectura propuesta para el traductor de PDDL, se analizan los
m\'odulos principales de la implementaci\'on y se concluye con un ejemplo de aplicaci\'on en
Ciao Prolog.

Por \'ultimo, en el cap\'itulo \ref{pagcap7}, se presentan las
conclusiones de esta tesis, sus resultados y contribuciones y
se proponen algunas ideas para trabajos futuros.






