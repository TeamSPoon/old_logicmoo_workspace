\ \\
\ \\
\label{pagresum}
\noindent{\LARGE \sc Resumen}\\
\ \\
\ \\

\ \\

\ \\
\ \\

La tem\'atica que se investiga en este trabajo es la planificaci\'on
continua mediante especificaciones en un lenguaje de
definici\'on de dominios de planificaci\'on denominado PDDL. El objetivo es implementar
un traductor de un subconjunto de este lenguaje a fin de que el
Framework de Planificaci\'on Continua, tambi\'en presentado aqu\'i,
pueda aprovechar las caracter\'isticas de PDDL. Este enfoque es
relevante ya que plantea la posibilidad de combinar
la expresividad de un lenguaje est\'andar, como PDDL, con un planificador continuo capaz de resolver 
problemas en ambientes reales. Es esperable alcanzar un mayor nivel 
de abstracci\'on para tratar dominios m\'as complejos y, adem\'as, realizar comparaciones emp\'iricas
de performance con otras soluciones para un mismo problema.

El lenguaje del Planificador Continuo, y uno de los lenguajes
fundacionales de representaci\'on de problemas de planificaci\'on, es STRIPS. Este formalismo
permite modelar acciones simples como listas de precondiciones y
efectos. Adem\'as, cada lista es una conjunci\'on de proposiciones. 
Por su parte, PDDL es un lenguaje sensiblemente m\'as
expresivo que STRIPS. Este lenguaje prove\'e caracter\'isticas
adicionales que definen varios niveles de
expresividad permitiendo enriquecer la definici\'on de dominios y
problemas de planificaci\'on. 

En base a lo expuesto, se plantea la necesidad de tratar con lenguajes m\'as
complejos con el objetivo de modelar acciones aplicables en ambientes
reales. No obstante, esto resulta en un mayor
costo computacional de los algoritmos al momento de resolver problemas
de planificaci\'on y, por lo tanto, es importante encontrar un balance aceptable entre expresividad y complejidad.

Este trabajo presenta un traductor, desarrollado en Ciao Prolog, de un
subconjunto PDDL que soporta algunas de sus caracter\'isticas. El
subconjunto conforma el lenguaje fuente del traductor y su salida
consiste en una representaci\'on similar a STRIPS que permite adaptar este traductor a
diferentes planificadores, entre ellos, al Planificador Continuo.

La investigaci\'on tambi\'en presenta otros dos resultados te\'oricos. 
La redefinici\'on de la arquitectura modular del Framework
de Planificaci\'on Continua para soportar PDDL y 
una nueva variante del lenguaje STRIPS para modelar acciones
con cuantificaci\'on universal en sus precondiciones. 
Sobre este \'ultimo resultado, tambi\'en introducimos un esquema de
compilaci\'on que permite traducir las acciones, definidas 
en esta variante, a STRIPS est\'andar. Adem\'as, esta traducci\'on conserva los resultados de 
planificaci\'on que se obtienen con ambas especificaciones. 
Este esquema, junto con otros esquemas de compilaci\'on enunciados, 
son usados para el desarrollo del traductor propuesto.

La experimentaci\'on sobre la implementaci\'on presentada se realiza sobre distintas
instancias del dominio del ``Mundo de Bloques'' (un problema
cl\'asico de la literatura de Inteligencia Artificial). Cada instancia de
este problema es modelada en PDDL empleando las distintas
ca\-rac\-te\-r\'is\-ti\-cas del subconjunto considerado.


\vfill
\pagebreak
