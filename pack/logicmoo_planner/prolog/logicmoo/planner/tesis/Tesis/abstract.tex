\ \\
\ \\
\label{pagsumm}
\noindent{\LARGE \sc Abstract}\\
\ \\
\ \\

\ \\

\ \\
\ \\

The researched topic in this work is related to the continuous
planning using specifications written in the Planning Domain Definition
Language named PDDL. 
The objetive is to implement a traslator of a PDDL language subset 
in order to allow that the Continuous Planning Framework (also, it is introduced
here) supports PDDL features. This approach is relevant because 
supposes combining the expressiveness of PDDL standard language with a
continuous planner that try to solve problems in real environments.  
We hope to reach a high-level abstraction in order to define more
complex domains and to compare the performance with other solutions for the
same problem.

The continuous planner language is STRIPS. This formalism allows
modeling simple actions using precondition and effect lists. Also,
each list is a conjunction of propositions. Particulary, PDDL is more 
expressive than STRIPS because it provides other features defining
expressiveness levels and allowing to enrich the definition of domains and
problems of planning.

Based on the above, it is necessary to try with more expressive languages in order to model
actions in real environments. Nevertheless, this implies to work with
more complex algorithms too. Then, a trade-off between
complexity and expressiveness is important.
 
This thesis presents a translator implemented in Ciao Prolog of a 
PDDL subset with a\-ddi\-tio\-nal features. This subset defines the
source language of translator and its output is a
STRIPS-like notation. This output allows adapting the translator to different planning 
algorithms as the Continuous Planner. 

The research also presents other two theorical results. First, the
architecture of Planning Continuous Framework is redefined. After, a
new STRIPS variant with universal preconditions is presented. In the
last result, actions defined in this variant could be translated to
STRIPS preserving results for both specifications. This result
together with other compilation squemes are used in the translator
implementation.

The experimentation on the implementation has been done on different 
instances of the ``Blocks World'' problem (a classic problem of
Artificial Intelligence bibliography). Each instance of this problem is modeled in PDDL by using
different features. 

\vfill
\pagebreak
