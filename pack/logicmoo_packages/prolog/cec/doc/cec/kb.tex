\documentstyle[11pt,prospectra,openbib]{article}
%\documentstyle[11pt,prospectra,openbib,adbgerman]{article}

\textwidth 17cm
\addtolength{\baselineskip}{1.0ex}
\addtolength{\parsep}{0.5ex}
\oddsidemargin -0.4cm
\evensidemargin -0.4cm
\topmargin 0pt
\headheight 0pt
\headsep 0pt
\footheight 0pt
\footskip 40pt
\textheight 660pt

%--------------------------------- HACKS ------------------------------------

\newcommand{\inMath}[1]{\ifmmode{#1}\else${#1}$\fi}
\newcommand{\bold}[1]{{\bf #1}}

% \joinrel and \relbar for 11pt, used for "linear" \longrightarrow
\def\joinrel{\mathrel{\mkern-3.3mu}}
\def\relbar{\mathrel{\raise0.25pt\hbox{$\smash-$}}}

% Macros for preventing pagebreaks:
% No pagebreak appears between \connect and \disconnect
% ! The macros must be used in pairs
\catcode`\@=11
\def\connect{\global\advance\@beginparpenalty 10000\relax}
\def\disconnect{\global\advance\@beginparpenalty -10000\relax}
\catcode`\@=12

%------------------------- MACROS FOR TEXT STYLE -------------------------

% Environment for completion examples
% The script of an completion examples can be put between \begin{screen} 
% and \end{screen}. Blanks, carriage return, subscript, superscript, math 
% shift and parameter character will be typeset as in verbatim. 
% A frame will be put around the script

\catcode`\@=11
\def\blank{\ }
\def\obeyblanks{\catcode`\ \active}
{\obeyblanks\global\let =\blank}

\def\screen{\def\@halignto{}\@screen}

{\catcode`\^^M=\active % these lines must end with %
  \gdef\tablines{\catcode`\^^M\active \let^^M\@linecr}%
  \global\let^^M\par} % this is in case ^^M appears in a \write

\def\@screen{\begin{center}\leavevmode \hbox \bgroup\def\arraystretch{0.6}\def\tabcolsep{1mm}\footnotesize $\@makeother\_\@makeother\^\@makeother\$\@makeother\#\obeyblanks\frenchspacing\let\@acol\@tabacol 
   \let\@classz\@tabclassz
   \let\@classiv\@tabclassiv \tt \@noligs
   \let\\\@linecr\tablines\@array[c]{@{}|p{13cm}|@{}}\hline\@gobblecr\vskip-3pt\@gobblecr}

\def\endscreen{\hline\crcr\egroup\egroup{\catcode`\$=3}$\egroup\normalsize\vspace*{2ex}\pagebreak[0]\end{center}}

\def\@linecr{\cr}
\catcode`\@=12

% User input 
% Usage    : \user{#1}
% Arguments: 	1. User input
\def\user#1{{\bf #1}}

% Description of an user option
% Usage    : \userOption{#1}
% Arguments: 	1. Option
\def\userOption#1{``{\bf #1}''}

% Parts of a specification
% works like the verb-macro
% Usage: \cec{#1}
\def\doCECspecials{\do\\\do\$\do\&%
\do\#\do\^\do\^^K\do\_\do\^^A\do\%\do\~}
\def\cec{\begingroup \catcode``=13  \noligs 
\tt \let\do\makeother \doCECspecials\frenchspacing\obeyspaces\cecTerm}
\def\cecTerm#1{#1\endgroup}

% Filename
% Usage: \file{#1}
\newcommand{\file}[1]{\kw{#1}}

% Suffix of a file
% Usage: \suffix{#1}
\newcommand{\suffix}[1]{\kw{#1}}

% Reference-arrow
\newcommand{\refArrow}[0]{\mbox{$\rightarrow$}}


%-------------------------------- GRAMMARS --------------------------------

\catcode`\@=11

% Environment for grammar descriptions
% The grammar must be described between \begin{syntax} and \end{syntax}

\def\syntax{\def\@halignto{}\@syntax}
\def\@syntax{\bigskip\leavevmode \hbox \bgroup $\let\@acol\@tabacol 
   \let\@classz\@tabclassz
   \let\@classiv\@tabclassiv \let\\\@tabularcr\@array[c]{lcl}}
\def\endsyntax{\crcr\egroup\egroup $\egroup\vspace*{2ex}\pagebreak[0]}

% Non-Terminals
% Usage    : \nt{#1}
% Arguments: 1. Non-terminal symbol
\def\nt{\hbox{$<$}\begingroup\rm\catcode`\_=13\def\_{\_}\nonTerminal}
\def\nonTerminal#1{#1\endgroup\hbox{$>$}}

% Keywords
% Usage    : \kw{#1}
% Arguments: 1. Terminal symbol

\def\makeother#1{\catcode`#112\relax}

\def\doKWspecials{\do\ \do\\\do\$\do\&%
\do\#\do\^\do\^^K\do\_\do\^^A\do\%\do\~}

\def\kw{\begingroup \catcode``=13  \noligs 
\tt \let\do\makeother \doKWspecials\keyword}
\def\keyword#1{#1\endgroup}

\begingroup
\catcode``=13
\gdef\noligs{\let`=\lquote}
\endgroup

\def\lquote{{\kern\z@}`}
\catcode`\@=12

% Definition of a non-terminal
\def\IS{& ::= &}

% Alternative
\def\OR{\\ & & \ts}

% End of the definition of a non-terminal
\def\END{\\}

% Additional symbols on the right hand side of a definition
\def\AND{\\ & &}

\def\GETON{\\ & &}

% Alternative
\newcommand{\ts}{\mbox{$\mid$ \hspace{0.15ex}}}


%------------------------------- COMMAND DESCRIPTION ---------------------
% Definition of a command
% Arguments: 	1. Identifier of the command
%		2. Arguments of the command
\newcommand{\com}[2]{\kw{#1(}#2\kw{\hspace{0.1em}\kw{)}}}

% Identifier of the command
\newcommand{\comName}[1]{\kw{#1}}

% Argument of a command
% Arguments:	1. Argument of the command
\newcommand{\comArg}[1]{{\it #1}}

% Trennung between arguments 
\newcommand{\ad}{\kw{,}\,}

% Reference to an command definition
\newcommand{\comRef}[1]{\kw{#1}}

% Environment for command descriptions
\def\command[#1]{\topsep 0pt\connect\noindent #1\list{}{\listparindent 1.5em
 \itemindent\listparindent
 \rightmargin\leftmargin \parsep 0pt plus 1pt}\item[]\disconnect\hskip -\parindent}
\def\endcommand{\endlist\medskip}


%--------------------- SUBTERMS AND SUBTERMREPLACEMENT ----------------------

% Subterm
% Arguments: 	1. Term
%	     	2. Occurrence
\newcommand{\subterm}[2]{\inMath{{#1}/{#2}}}

% Subtermreplacement
% Arguments: 	1. The term in which the replacement takes place
%		2. Occurrence of the subterm that is to replace
%		3. Term that will take the place of the old subterm
\newcommand{\replace}[3]{\inMath{{#1}[{#2} \leftarrow {#3}]}}


%------------------------------ SUBSTITUTION ------------------------------

% Variable-Replacement
% Arguments:	1. Variable
% 		2. Term
\newcommand{\varRep}[2]{\inMath{{#1}/{#2}}}

% Composition of variable-replacements
\newcommand{\repSep}{,}

% Transformation of a composition of variable-replacements into a
% substitution
\newcommand{\subst}[1]{\inMath{\{{#1}\}}}

% Set of all substitutions
\newcommand{\subs}[1]{\inMath{{\rm SUB}_{#1}}}

% Subsumption
\newcommand{\subsumeq}{\inMath{\leq}}

% Domain-Operator
\newcommand{\domain}[1]{\inMath{{\cal D}({#1})}}

% Introduced-Operator
\newcommand{\introd}[1]{\inMath{{\cal I}({#1})}}

% Representation of a substitution as equation-system
\newcommand{\substeqrep}[1]{\inMath{[{#1}]}}

% Composition of two substitutions
% Arguments:	1. Substitution that will be applied first
%		2. Substitution 
\newcommand{\composesubst}[2]{\inMath{{#2}\circ{#1}}}

% Application of a substitution
% Arguments:    1. Substitution
%		2. Argument of application
\newcommand{\applysubst}[2]{\inMath{{#2}{#1}}}


%------------------------- EQUATIONS AND RULES -------------------------

% Equation
% Arguments:	1. Left side of the equation
%		2. Right side of the equation
\newcommand{\eq}[2]{\inMath{{#1} = {#2}}}

% Conditional equation
% Arguments:	1. Condition
%		2. Conclusion
% Examples: \condEq{\eq{a}{b}\condAnd\eq{c}{d}}{\eq{e}{f}}
%	    \condEq{\cond{\eq{a_1}{b_1}}{\eq{a_n}{b_n}}}{\eq{a}{b}}
\newcommand{\condEq}[2]{\inMath{{#1} \Rightarrow {#2}}}

% Condition of a conditional equation
% Arguments:	1. First equation of the condition
%		2. Last  equation of the condition
\newcommand{\cond}[2]{\inMath{{#1} \wedge \ldots \wedge {#2}}}

% Rule
% Arguments:	1. Left  side of the rule
%		2. Right side of the rule
\newcommand{\rewRule}[2]{\inMath{{#1} \rightarrow {#2}}}

% Condtional rule
% Arguments:    1. Condition
%		2. Conclusion
% Examples: \condRule{\eq{a}{b}\condAnd\eq{c}{d}}{\rewRule{e}{f}}
%	    \condEq{\cond{\eq{a_1}{b_1}}{\eq{a_n}{b_n}}}{\rewRule{a}{b}}
\newcommand{\condRule}[2]{\inMath{{#1} \Rightarrow {#2}}}

\newcommand{\condAnd}{\wedge}

% Left side of a rule with an given index
\newcommand{\ruleli}[1]{\inMath{l_{#1}}}

% Right side of a rule with an given index
\newcommand{\ruleri}[1]{\inMath{r_{#1}}}

% Rule of an given index
\newcommand{\rulei}[1]{\inMath{\ruleli{#1} \rightarrow \ruleri{#1}}}


%-------------------------------- REWRITING --------------------------------

\newcommand{\newarrow}[0]{-\hspace{-1.7mm}-\hspace{-2mm}-\hspace{-3mm}\succ}
\newcommand{\inewarrow}[0]{\prec\hspace{-3.5mm}-\hspace{-2mm}-\hspace{-2mm}-}

% Rewriting-Relations
% On-step-rewriting with one parameter (e.g. the rewriting system)
\newcommand{\rewRel}[2]{\inMath{\mathrel{{#1}_{#2}}}}
\newcommand{\rew}[1]{\rewRel{\longrightarrow}{#1}}
\newcommand{\irew}[1]{\rewRel{\longleftarrow}{#1}}
\newcommand{\rewqr}[1]{\rewRel{\newarrow}{#1}}
\newcommand{\irewqr}[1]{\rewRel{\inewarrow}{#1}}

% Transitive closure of one-step-rewriting with one parameter
\newcommand{\rewRelT}[2]{\inMath{\mathrel{\stackrel{+}{#1}_{#2}}}}
\newcommand{\rewT}[1]{\rewRelT{\longrightarrow}{#1}}
\newcommand{\irewT}[1]{\rewRelT{\longleftarrow}{#1}}
\newcommand{\rewqrT}[1]{\rewRelT{\newarrow}{#1}}
\newcommand{\irewqrT}[1]{\rewRelT{\inewarrow}{#1}}

% Reflexive, transitive closure of one-step-rewriting with one parameter
\newcommand{\rewRelRT}[2]{\inMath{\mathrel{\stackrel{*}{#1}_{#2}}}}
\newcommand{\rewRT}[1]{\rewRelRT{\longrightarrow}{#1}}
\newcommand{\irewRT}[1]{\rewRelRT{\longleftarrow}{#1}}
\newcommand{\rewqrRT}[1]{\rewRelRT{\newarrow}{#1}}
\newcommand{\irewqrRT}[1]{\rewRelRT{\inewarrow}{#1}}

\newcommand{\rewqrnf}[1]{\multIndRel{\newarrow}{}{nf}{}{#1}}
\newcommand{\irewqrnf}[1]{\multIndRel{\inewarrow}{}{nf}{}{#1}}
\newcommand{\nfrelqr}[1]{\rewqrnf{#1}\irewqrnf{#1}}

% Symmetrical, reflexive, transitive closure of one-step-rewriting with
% one parameter
\newcommand{\rewSRT}[1]{\inMath{\mathrel{\stackrel{*}{\longleftrightarrow}_{#1}}}}

% Symmetrical closure of one-step-rewriting
\newcommand{\equ}[1]{\mbox{$\mathrel{\leftrightarrow_{#1}}$}}

% Reflexiv, symmetrical and transitive closure of one-step-rewriting
\newcommand{\equs}[1]{\mbox{$\mathrel{\stackrel{\ast}{\leftrightarrow}_{#1}}$}}

% Normalform
\newcommand{\nf}[1]{\inMath{#1\hspace*{-0.2ex}\downarrow}}
\newcommand{\nfr}[1]{\inMath{#1\hspace*{-0.2ex}\downarrow}}
\catcode`\@=11
% "\condsub#1#2" yields "", if #2 is empty, and "#1{#2}" otherwise
\def\condsub#1#2{\c@ndsubx \c@ndsuba#2\c@ndsubb\c@ndsubc\c@ndsuba\c@ndsubb
    #1{#2}\c@ndsubc\c@ndsubd}
\def\c@ndsubx #1\c@ndsuba\c@ndsubb#2\c@ndsubc#3\c@ndsubd{#2}%
\catcode`\@=12
\newcommand{\multIndRel}[5]{\mathrel{\vphantom{\mathop{#1}\limits
    \condsub^{#2}\condsub_{#4}}\smash{\mathop{#1}\limits
    \condsub^{#2}\condsub_{#4}}\condsub^{#3}\condsub_{#5}}}
\newcommand{\rewr}[4]{\multIndRel{\longrightarrow}{#1}{#2}{#3}{#4}}
\newcommand{\irewr}[4]{\multIndRel{\longleftarrow}{#1}{#2}{#3}{#4}}
\newcommand{\srewr}[4]{\multIndRel{\longleftrightarrow}{#1}{#2}{#3}{#4}}
\newcommand{\eqrel}[2]{\multIndRel{\longleftrightarrow}{#1}{}{}{#2}}
\newcommand{\eqv}[4]{\multIndRel{\equiv}{#1}{#2}{#3}{#4}}
\newcommand{\nfrel}[1]{{\big\downarrow_{#1}}}

%------------------------------- COMPLETION ------------------------------

% Environment for the description of a inference rule for completion
% Examples: \begin{CRule} ... \end{CRule}
%	    \begin{CRule}[(cp), Computation of an critical pair] ... \end{CRule}
\catcode`\@=11
\def\CRule{\par\@ifnextchar[{\@namedRule}{}}
\def\@namedRule[#1,#2]{\noindent \makebox[11em][l]{\it #1} {\it #2}}
\def\endCRule{}

% Deduction Rule with additional condition
\def\ifdeducRule#1#2if#3{\[ \frac{#1}{#2},\ \ {\rm if\ }#3 \]}

% Simple deduction rule
\def\deducRule#1#2{\[ \frac{#1}{#2} \]}
\catcode`\@=12

% Proof ordering
\newcommand{\PO}{\inMath{>_{\cal P}}}

% Termination ordering
\def\TO>{>}

% Tripel
\newcommand{\tripel}[3]{\inMath{(#1,\ #2,\ #3)}}

% 
\newcommand{\stateEl}[2]{\inMath{#1 : #2}}


%------------------------- MATHEMATICAL CONSTRUCTS -------------------------

% Nat
\newcommand{\Nat}{\mbox{$I\hspace{-0.3em}N$}}

% Int
\newcommand{\Int}{\mbox{$Z\hspace{-0.5em}Z$}}

% Reals
\newcommand{\Real}{\mbox{$I\hspace{-0.3em}R$}}

% ... without ...
\newcommand{\without}[2]{\inMath{{#1} \backslash {#2}}}

% Power set
\newcommand{\powerset}[1]{\inMath{{\cal P}({#1})}}

% ... and ... are disjoint
\newcommand{\disjunct}[2]{\inMath{{#1}\cap{#2}=\emptyset}}

% Restriction to ...
\newcommand{\restrict}[1]{\inMath{|_{#1}}}

% not member of ...
\newcommand{\nin}{\inMath{{\not \in}}}

% Congruence relation
% Arguments:	1. Relation
\newcommand{\congruent}[1]{\mbox{$\,\equiv_{#1}\,$}}


% Logical operators
% Evaluation environment: math mode
\newcommand{\logand}{\;\wedge\;}
\newcommand{\logor}{\;\vee\;}
\newcommand{\implies}{\;\Longrightarrow\;}

% Symbols for definitions
% Evaluation environment: math mode
\newcommand{\dfiff}{\;:\Longleftrightarrow\;}
\newcommand{\dfeq}{\;:=\;}

% Set
\newcommand{\set}[1]{\inMath{\{{#1}\}}}

% ZF-Set
% Arguments:	1. Set
%		2. Predicate
\newcommand{\predset}[2]{\inMath{\{{#1}\;|\;{#2}\}}}

% QED
\newcommand{\qed}{\mbox{\kern2em}\hfill\llap{$\Box$}}

\newenvironment{spec}{\vspace{1ex}\par\addtolength{\parindent}{1.0cm}\obeylines\obeyspaces\tt}{\vspace{1ex}}


\begin{document}

\title{
{\bf CEC}\\
A System to Support Modular Order-Sorted Specifications
with Conditional Equations\\
User Manual (Version 1.5)}
\author{
Hubert Bertling\\
Harald Ganzinger\\
Renate Sch\"afers}

\prospectrano{M.1.3--R--18.0}
\prospectradist{public}

\maketitle

\begin{abstract}
{\bf CEC} is a rewrite rule laboratory for order-sorted specifications
with conditional equations. The major module of CEC is a powerful
completion procedure for conditional equations. This manual describes its 
use assuming the user to be familiar with basic notions in
conditional term rewriting.
\end{abstract}

\setcounter{page}{1}
\pagenumbering{roman}
\tableofcontents

\newpage

\setcounter{page}{1}
\pagenumbering{arabic}


\section{Introduction}

This manual describes the use of CEC,
a system for conditional equational completion. We assume familiarity
with well known notions in the term rewriting area e.g.
{\it
signature, term over a signature,
(conditional) term rewrite rule, applicability of a rewrite rule, 
\ldots}, cf. e.g. \cite{HO80}, \cite{Kap84}. The concepts are discussed only as far as
necessary to allow a meaningful interaction with the system. Theoretical
foundations of the concepts implemented in CEC are not discussed. 
Hints to the literature will be given whenever such concepts are introduced.

CEC can be used under Quintus-Prolog2.x.
An installation guide for CEC is to be found in the appendix of this manual.


There are some CEC-specific notions used throughout this manual. The input to 
CEC is called a 
{\it specification}. Also any later state of the initial
specification is called a specification. To distinguish the specification
which will be completed when invoking the completion procedure from any other
specification CEC presently knows about, this specification is called the 
{\it current specification}.

CEC includes a help-function. Type 
``\user{??}\nt{space}\user{.}'' to obtain a 
list of all available CEC-commands and type
``\user{?}\nt{FunctionName}\user{.}''
or ``\user{?(}\nt{FunctionName}\user{).}'' to 
get a short description for
CEC-command with the name {\it FunctionName}.
   % --.4.89
\section{An Example Session}
\label{exampleSession}

To provide a first impression of  the capabilities of the CEC-System, we
describe the development of a quicksort algorithm on lists of
natural numbers.

\subsection{The Specification of Quicksort}

Specifications can be written and completed in a {\em modular}
way. CEC can combine completed specifications without 
repeating previous work. The hierarchy of modules for the
specification of quicksort on lists of natural numbers is given
in the following diagram:

\begin{picture}(450,100)
\put(50,0){\makebox(400,100){}}
\put(205,85){\makebox(0,0){{\tt qsortnat}}}
\put(200,80){\vector(-2,-1){45}}
\put(200,80){\vector(2,-1){45}}

\put(150,50){\makebox(0,0){{\tt nat}}}

\put(257,50){\makebox(0,0){{\tt qsort}}}
\put(255,45){\vector(-3,-2){30}}
\put(255,45){\vector(3,-2){30}}

\put(215,20){\makebox(0,0){{\tt totalOrder}}}

\put(300,20){\makebox(0,0){{\tt lists}}}
\end{picture}

\noindent
The module \cec{qsort} --- saved in a file named \cec{qsort.eqn} ---
describes the quicksort algorithm:
\begin{spec}
module qsort using lists + totalOrder.

op sort  : (list -> list).
op split : (elem * list -> pair).
cons (',') : (list * list -> pair).

sort([]) = [].
split(x, l) = (l1 , l2) => sort([x|l]) = append(sort(l1), [x|sort(l2)]).

split(x, []) = ([] , []).
(y =< x) = true and split(x, l) = (l1,l2) => split(x, [y|l]) = ([y|l1],l2).
(y =< x) = false and split(x, l) = (l1,l2) => split(x, [y|l]) = (l1,[y|l2]).
\end{spec}

Complex specifications can be constructed from modules by the 
elementary operations {\em combination}, {\em enrichment} and {\em renaming}
(\refArrow chapter \ref{OperationsOnSpecifications}).
The combine operator (\cec{+}) forms the union of two specifications and
the rename operator (\cec{<-}) allows to renaming sorts and operators.
Any specification in CEC is the enrichment of a possibly empty base
specification (``{\tt using} $<$base$>$'') by new vocabulary and axioms.
The module \cec{qsort} is based on \cec{lists} 
and \cec{totalOrder}. Here \cec{lists} is the imported module
\begin{spec}
module lists.

cons []   : list .
cons '.'  : (elem * list -> list).
op append : (list * list -> list).

append([], l) = l.
append([e | l1], l2) = [e | append(l1, l2)].
\end{spec}

We use the constructor \cec{[]} to denote the empty list and 
the constructor `\cec{.}' for adding an element to a list
(This is exactly the way how lists are constructed in Prolog and we can
use the usual Prolog notation for lists in which \cec{[e|l]} is
a synonym for \cec{(e '.' l)}). 
Additionally we have defined an 
operator \cec{append} for the concatenation of two lists and described 
its behaviour through two equations.

\cec{totalOrder} plays the r$\hat{o}$le of a formal parameter
of \cec{qsort}:
\begin{spec}
module totalOrder.

op (=<) : (elem * elem -> bool) .

(x =< x) = true .
(x =< y) = false => (y =< x) = true .
(x =< y) = true and (y =< z) = true => (x =< z) = true .
(x =< y) = true and (y =< x) = true => x = y .
\end{spec}

It describes a usable approximation of the usual first-order axioms
for total orders in a Horn clause setting.
An actual parameter candidate for \cec{totalOrder} is
the following specification of natural numbers:
\begin{spec}
module nat using totalOrder(elem <- nat).

cons 0 : nat.
cons(s, 100, fy) :  (nat -> nat).

(0 =< n ) = true.
(s n =< 0) = false .
(s n =< s m) = (n =< m).
\end{spec}

Computing with the above specification one has to write the numbers
$n$ as \cec{s}$^n$\cec{0} which is a tedious work. In CEC it is possible to provide different external
representations for the terms of a specification by specifying
the translations from the external to the internal
and from the internal to the external representation 
(\refArrow chapter \ref{ParseAndPretty}).

The instantiation of the formal parameter of \cec{qsort} with
nat yields \cec{qsortnat}:
\begin{spec}
module qsortnat using qsort(elem <- nat) + nat.
\end{spec}

The modules \cec{qsort} and \cec{nat} are combined after
renaming the sort \cec{elem} of \cec{qsort} into \cec{nat}.
The completion of \cec{qsortnat} will check the consistency of the axioms 
for \cec{=<} in the actual parameter \cec{nat} and in the formal
parameter \cec{totalOrder}. The latter proves the correctness
of the actual parameter in cases where the actual parameter is
constructor-complete. \cec{nat} is constructor-complete.

\subsection{Order Specifications}

In CEC the following termination orderings are available:
\begin{itemize}
\item
Two  {\em precedence orderings} \kw{kns} and \kw{neqkns}, according to 
Kapur et. al \cite{KNS85} (\refArrow chapter \ref{kns}).
%They are based on the recursive comparison of paths occurring
%in the terms to be compared. This order is induced by a partial
%order on operators called the {\it precedence}. The precedence ordering
%\kw{neqkns} forbids that two different operators have the same precedence.
%All constructors are given a precedence less than any nonconstructor operator.
\item
The method of {\it polynomial interpretations}
where \[t_1 > t_2 : \iff I(t_1) > I(t_2). \]
if $I(t)$ is the polynomial or tuple of polynomials associated with
it.
The concrete version of this technique as it is used in CEC is due to 
\cite{CL86} (\refArrow chapter \ref{poly}).
\kw{poly}\nt{N} stands for polynomial interpretations with 
tuplelength $N$.
\end{itemize}
The precedence declarations or polynomial interpretations
for the operators of a specification are given interactively
during the completion process or in a {\em order specification}
associated with the specification
(\refArrow chapter \ref{OrderSpecification}).
%The order specification determines the termination ordering to be used for 
%\nt{specificationName}, the order names for its direct imports
%and gives precedence declarations for the operators of the
%specification (if the termination ordering is \kw{kns} or \kw{neqkns}) or
%polynomial interpretations (if the termination ordering is \kw{poly}\nt{N}).

For example, the order specification for the module \cec{lists}
may have the following form:

\begin{spec}
order poly3 for lists.

setInterpretation(['[]'        : [2, 2, 2],
                   '.'(x,y)    : [3 * x + y + 1, x + y, x + y],
                   append(x,y) : [x + y, x + y, 2 * x + y]]).
\end{spec}

In this case, polynomial interpretations with tuple length 3
have been chosen. If this information is stored in the file \cec{lists.pol.ord},
\cec{pol} is called the {\em order name} of this order specification.

\subsection{Reading and Displaying}

We want to demonstrate how CEC handles this specification. 
First, we read in the specification of \cec{lists} using
the \comRef{in}-command (\refArrow chapter \ref{InCommand})
with two arguments, the name of the module and the order name
(\refArrow chapter \ref{orderBase}):

\begin{screen}
| ?- in(lists,pol).
[collecting garbage...]
[evaluating base of lists.eqn with lists.pol.ord...]
 [thawing standard.poly3.q2.0 into user...]
 [storing to standard.poly3]
[reading body of lists.eqn and lists.pol.ord...]
[analyzing axioms...]
[collecting garbage...]

Time used: 3.634 sec.
\end{screen}

\noindent
We can use the \comRef{sig}-command to display the signature:

\begin{screen}
| ?- sig.

Signature :

cons true       : bool.
cons false      : bool.
cons []	: list.
cons .  : (elem * list -> list).
op append       : (list * list -> list).
\end{screen}

The boolean constants \cec{true} and \cec{false} are imported
from the \cec{standard}-module, which is automatically imported into 
every specification. The fact that \cec{true} and \cec{false}
are constructors implies that any equation, explicitly given in
a specification or derived through completion, equating these
constants is forbidden.
As expected, we also see our two constructors \cec{[]} and
`\cec{.}' and the operator \cec{append}.

The \comRef{show}-command can be used to display the 
set of equations and rules (\refArrow chapter \ref{Displaying}):

\begin{screen}
| ?- show.

Current equations

  1    append([],l) = l
  2    append([e|l1],l2) = [e|append(l1,l2)]

Current rules

Current nonoperational equations

All axioms reduced.
All superpositions computed.
The set of equations is not empty.
\end{screen}

\noindent
The information about the polynomial interpretations can be made
visible using the \comRef{interpretation}-command. If \cec{kns}
or \cec{neqkns} is used, this can be achieved by the
\comRef{operators}-command (\refArrow chapter \ref{InspecTermination}).

Every consulted specification is saved in a special {\em specification
variable}. The user can restore an older state of his system
using the \comRef{load}-command (\refArrow chapter \ref{LoadCommand})
or can update the content of the variable using the \comRef{store}-command
(\refArrow chapter \ref{StoreCommand}). 
Whenever a specification is to be imported, the CEC system uses the content
of the corresponding specification variable, if this specification variable 
exists. So it increases one's speed, if the completion process of a hierarchical
specification follows the tree structure from its leaves to its root, 
storing every participating specification after successful completion.
We assume that attention is paid to this advice for our example.
Information about the current specification variables
can be retrieved using the \comRef{specifications}-command.

It is also possible to save a system externally using the 
\comRef{freeze}-command (\refArrow chapter \ref{FreezeCommand}).
To restore a externally saved system use the \comRef{thaw}-command
(\refArrow chapter \ref{ThawCommand}).

\subsection{Completion}
We now apply the completion procedure (\refArrow chapter
\ref{Completion}) to our example.

\subsubsection{Completing {\tt lists}}

The completion process --- started with the \comRef{c}-command ---
is able to orient the two equations of \cec{lists}
without requesting any additional information:

\begin{screen}
| ?- c.

new rule   1    append([],l) = l .

new rule   2    append([e|l1],l2) = [e|append(l1,l2)] .

[4 superpositions yet to be considered.]

0 superpositions have been computed.
Time used: 0.883011 sec.
\end{screen}

\noindent
The system is complete, and has the following rules, displayed 
by the \comRef{show}-command:

\begin{screen}
| ?- show.

Current equations

Current rules

  1    append([],l) = l
  2    append([e|l1],l2) = [e|append(l1,l2)]

Current nonoperational equations

All axioms reduced.
All superpositions computed.
No more equations, the system is complete.
\end{screen}


\subsubsection{Completing {\tt totalOrder}}

An important feature is that CEC does not require to transform all
equations into rewrite rules. This can be demonstrated during the
completion of \cec{totalOrder}.

After orienting the first equation into a rewrite rule, 
the system discovers that it is unable to orient the second 
equation:

\begin{screen}
Checking reductivity constraints for rule
        x=<y = false => y=<x = true:
The current ordering fails to prove
[y=<x]  >  [x=<y,false].
At this point you may take any of the the following actions:
a. for assume to be proved
c. for checking quasi-reductivity of the equation
p. for postpone
n. for considering the equation as nonoperational
   Please answer with a. or c. or p. or n. (Type A. to abort) > \user{n.}
\end{screen}

\noindent
We want to consider this equation as nonoperational
(\refArrow chapter \ref{Completion}), so we answer with \user{n.}.

Nonoperational equations become useless for the equational theory and, hence,
in fact nonoperational, if they are superposed on at least one of their
conditions by all rewrite rules. This yields new conditional equations. 
What concerns the equational theory, the new equations together have 
the same power as the original equation. 
However, they may have better operational properties than the
equation they have been generated from.
Equations that cannot be oriented into a reductive rule must
either be eliminated eventually or considered nonoperational.

Checking the convergence of conditional equations requires to compare different
applications of equations and rewrite rules. The comparison of equation application is performed
by comparing the literals of the equation instance. The {\em status} of an equation 
determines the order in which the literals of the equation should be
inspected. The status $ms$ means that the literals are compared as a multiset.
Instead of $ms$ the user can choose an arbitrary sequence of the literals
by entering a permutation of [0 .. n] where n is the number of conditions
(\refArrow \cite{Gan88a}). 

We want to use $ms$ here:

\begin{screen}
In which order should the literals of the equation be
inspected when comparing proofs that use this equation?
Please enter ms (for multiset ordering) or a permutation of [0 .. 1]
(0 stands for the consequent, i>0 for the ith condition). > \user{ms.}
\end{screen}

\noindent
The other two nonreductive equations are handled in the same way and we get the 
following complete system:

\begin{screen}
| ?- \user{show.}

Current equations

Current rules

  1    x=<x = true

Current nonoperational equations

  1    x=<y = false => y=<x = true
  2    x=<y = true and y=<x = true => x = y
  3    x=<y = true and y=<z = true => x=<z = true

All axioms reduced.
All superpositions computed.
No more equations, the system is complete.
\end{screen}

In this case, superposing the only rule \cec{x =< x = true} on the first
condition of the nonoperational equations does not generate any nontrivial
(nonconvergent) consequences.

\subsubsection{Completing {\tt qsort}}

The next step is the completion of the \cec{qsort} specification.
The completion procedure is able to orient the first and third 
equation of our specification, but fails to orient the second one.

\begin{screen}
| ?- c.

new rule   9    sort([]) = [] .

new rule  10    split(x,[]) = [],[] .
The equation
        split(x,l) = l1,l2 => sort([x|l]) = append(sort(l1),[x|sort(l2)])
is not reductive.
\end{screen}

\noindent
This is due to the presence of the {\em extra variables} \cec{l1} and \cec{l2}
in the condition of the equation.

Equations with extra variables in the condition or in the right side of
the consequence are usually not admitted as rewrite rules. 
Fortunately some of these equations belong to the class of what we call
{\em quasi-reductive rules} (\refArrow chapter \ref{Completion}). 
CEC is able to prove the remaining equations 2, 4
and 5 of the module \cec{qsort} to be quasi-reductive.
Quasi-reductive rules are a generalization
of reductive conditional rewrite rules and the associated rewrite
process is similarly efficient.
It specifies e.g. for the second equation the replacement
of \cec{sort([x|l])} by the term \cec{append(sort(l1), [x|sort(l2)])} 
if the normalform of \cec{split(x, l)} matches \cec{(l1,l2)}.
After this match the extra variables \cec{l1} and \cec{l2} are instantiated to
terms which include only variables of the left hand side.
In quasi-reductive rules, conditions are oriented, too. To solve an instance
of a condition equation means to rewrite its left side into an instance of the
right side.

\begin{screen}
Do you want a check for quasi-reductivity?
c. for check
n. for considering the equation as nonoperational
p. for postpone
   Please answer with c. or n. or p. (Type A. to abort) >\user{c.}
\end{screen}

\noindent
To orient an equation into a quasi-reductive rule, we must first indicate the 
desired orientation of the equations in the condition and the conclusion. 
In this example, the equation in the condition
and the conclusion should be oriented from left to right (literal annotation
``\user{l}'').

\begin{screen}
Enter annotations of literals in
	split(x,l) = l1,l2 => sort([x|l]) = append(sort(l1),[x|sort(l2)])
	: \user{[l,l].}
\end{screen}

\noindent
Now CEC attemps to prove the quasi-reductivity of the equation according to the
definition in chapter \ref{Completion}. This involves giving  polynomial
interpretations to some auxiliary operators.
Here we must give an appropriate interpretation for \cec{$h5_0}:

\begin{screen}
Checking reductivity constraint:
Consider the terms
        $h5_0((l1,l2),x)
and
        append(sort(l1),[x|sort(l2)])

There is no interpretation of operator '$h5_0' with arity 2
The default interpretation is :
   [ 2 * x * y ,
     2 * x * y ,
     2 * x * y ]
Do you want to change it ? (if so type 'y') \user{y}
Do you want to change it component for component ? (if so type 'y') \user{n}
Type in the new interpretation tuple
[ $h5_0 ] (x,y) = \user{[2*x+3*y+1,2*x+2*y,2*x+2*y].}
Resulting interpretation for Operator '$h5_0' with arity 2 :
   [ 2 * x + 3 * y + 1 ,
     2 * x + 2 * y ,
     2 * x + 2 * y ]
Do you accept it ? (if not, type 'n') \user{y}
\end{screen}

\noindent
Now the proof of the quasi-reductivity of the equation is completed:

\begin{screen}
new rule  11    split(x,l) = l1,l2 => sort([x|l]) = 
                                      append(sort(l1),[x|sort(l2)]) .
\end{screen}

\noindent
In the same way the equations

\cec{y=<x = true and split(x,l) = (l1,l2) => split(x,[y|l]) = ([y|l1],l2)}

\noindent
and 

\cec{y=<x = false and split(x,l) = (l1,l2) => split(x,[y|l]) = (l1,[y|l2])}

\noindent
can be oriented into quasi-reductive rules.

For the quicksort specification three nontrivial superposition instances are
computed. For any nontrivial equation with at least one condition that is
generated during completion, CEC will ask the user what to do with it. In the
example we decide to declare these consequences as ``nonoperational''.

\begin{screen}
instance   6    l1,l2 = l4,l3 and split(x1,l5) = l1,l2 => 
                                     append(sort(l1),[x1|sort(l2)]) = 
                                     append(sort(l4),[x1|sort(l3)])
of        6    split(x,l) = l1,l2 => sort([x|l]) = 
                                     append(sort(l1),[x|sort(l2)])
by superposing
          6    split(x,l) = l1,l2 => sort([x|l]) = 
                                     append(sort(l1),[x|sort(l2)]) 
on the left side.

Consider the equation
        l1,l2 = l4,l3 and split(x1,l5) = l1,l2 => 
                                     append(sort(l1),[x1|sort(l2)]) = 
                                     append(sort(l4),[x1|sort(l3)]).
The following actions may be taken:
o. for attempting to orient into a (quasi-)reductive rule
p. for postpone
n. for considering equation as nonoperational
   Please answer with o. or p. or n. (Type A. to abort) >\user{n.}
\end{screen}

As mentioned before it is sufficient to superpose all rewrite rules
on just one condition of a nonoperational conditional equation. So
CEC asks the user on which condition superposition should be
applied. 
We will choose the first equation of the condition, since we know it
will generate no nontrivial superpositions:

\begin{screen}
Which of the condition equations in
        l1,l2 = l4,l3 and split(x1,l5) = l1,l2 
should be selected for superposition?
Please enter index from 1 to 2. > \user{1.}

The equation l1,l2 = l4,l3 and split(x1,l5) = l1,l2 => 
                                          append(sort(l1),[x1|sort(l2)]) = 
                                          append(sort(l4),[x1|sort(l3)]) 
will be considered as nonoperational.
\end{screen}

\subsubsection{Completing {\tt nat}}

The completion of the \cec{nat} specification is straightforward using
the following order specification:

\begin{spec}
order poly1 for nat. 

setInterpretation([0 : 2,
                   s(x) : 8 * x]).
\end{spec}

\subsubsection{Combining {\tt qsort} and {\tt nat}}

We now want to combine the \cec{nat} specification with the 
\cec{qsort} specification.

\begin{screen}
| ?- store.
yes
| ?- load(qsort,poly3).
\end{screen}

\noindent
The \comRef{store}-command saves the completed {\tt nat} specification
into its {\em specification variable}.
The \comRef{load}-command loads the completed {\tt qsort} specification
from its specification variable.

\begin{screen}
| ?- renameSpec(elem <- nat).
yes
| ?- sig.

Signature :

cons [] : list.
cons .  : (nat * list -> list).
op append       : (list * list -> list).
cons true       : bool.
cons false      : bool.
op =<   : (nat * nat -> bool).
op sort	: (list -> list).
op split        : (nat * list -> pair).
cons ,    : (list * list -> pair).
yes
| ?- store(qsortnatSpec).
yes
\end{screen}

\noindent
Now we combine the two specifications and make the result 
to the current specification:

\begin{screen}
| ?- combineSpecs(qsortnatSpec,'nat.poly1_qsort',user).
yes
\end{screen}

\noindent
Because \cec{totalOrder} is a formal parameter for \cec{qsort},
completion now checks the consistency of the axioms for
\cec{=<} in the actual parameter \cec{nat} and in the formal
parameter \cec{totalOrder}.
The completion process will not do unnecessary work again. 
For the above example the completion process
will only compute overlaps between axioms
of the renamed module \cec{qsort} and axioms of the module \cec{nat}
but it will not recompute overlaps between axioms of one of these
modules.

If two constructor terms are shown to be equal then the specification
is inconsistent. In our case the system is completed without any user 
interaction, no inconsistency showing up.

\subsection{Computing in Completed Specifications}

Computation in specifications is realized by term reduction with the
rules of the completed specification. The result of such computations are unique
normal forms (\refArrow chapter \ref{NormCommand}):

\begin{screen}
| ?- norm(sort([5,3,6,1])).
The normalform of sort([5,3,6,1]) is [1,3,5,6] .
\end{screen}

\noindent
If confluence can be achieved and if all rules are reductive equational theorems 
become decidable, e.g. it is decidable if two terms are equivalent with respect to the
equations in the specification (\refArrow chapter \ref{ProveCommand}):

\begin{screen}
| ?- prove(sort([5,3]) = [3,5]).
Normal forms are: [3,5] and [3,5]
yes
\end{screen}

\noindent 
Conditional narrowing (\refArrow chapter \ref{NarrowCommand}) can then be used to solve 
equations.

\begin{screen}
| ?- solve(sort([1,x]) = [x,1],U).

Time used: 7.43298 sec.

U = {x-nat/0} ;

Time used: 7.91595 sec.

U = {x-nat/1} ;

no
\end{screen}

\noindent
Here we proved that the only substitutions for \cec{x} such that
\cec{sort([1,x])} is equal to \cec{[x,1]} are \cec{0} and \cec{1}.

\subsection{Saving the CEC System}

The whole state of the CEC system can be saved using the
\comRef{saveCEC}-command. 

\begin{screen}
| ?- saveCEC('qsortCEC').
[ Prolog state saved into /home/helga/cec/cec/qsortCEC ]
\end{screen}

Using \cec{qsortCEC} instead of CEC offers the possibility to have
the complete hierarchy of our \cec{qsort} example present, without
wasting time to reconsult the necessary frozen states of all the
specification used for this example.

                                                         
\section{Input Format of Specifications}

Specifications consist of a {\em specification name}, an optional 
{\em base declaration} and an {\em enrichment}. Additionally an
{\em order and action specification} can be supplied to support the completion
process. Of course it is possible to have more than one order specification
associated with a specification.

\begin{syntax}
\nt{specification} \IS  \kw{module} \nt{specificationName}
                   \AND [ \nt{base} ] \nt{full stop} 
                   \AND \nt{enrichment} 
\end{syntax}

\noindent
The name declaration has the format:

\begin{syntax}
\nt{specificationName}  \IS \nt{Prolog atom} \END
\end{syntax}

\noindent
The specification name is used to denote the specification.
If the specification is stored in a file,
the file should be named \nt{specificationName}\suffix{.eqn}.

The base declaration allows to import other specifications as
modules. The imported modules
again form a module called {\em base module}.  An enrichment defines (parts
of) a {\em signature}, {\em axioms} and {\em pragmas}. The base module
together with the enrichment forms the current specification.  
% Pragmas may be added in order to speed up the completion process, e.g. the 
% choice of a termination ordering and declarations for that ordering. This 
% is to avoid unnecessary computations of suggestions and unwanted user interaction.
Pragmas may redefine ``parsing'' and ``pretty-printing'' of specifications.
Any element of a specification must be terminated by a full stop 
which is a ``.''
followed by a white space.  This syntactic requirement follows from the
conventions of the underlying Prolog-System. {\em Comments} also
follow Prolog syntax, e.g.\ anything in a line after a \kw{%} or
all text between \kw{/*} and \kw{*/} is a comment.
For those who are unfamiliar with Prolog the syntax of a Prolog atom is provided:

\begin{syntax}
\nt{Prolog atom} 	 \IS \nt{textual atom} \ts \nt{symbol atom} \END
\nt{textual atom}        \IS \nt{lower case letter} \{ \nt{letter} \ts \kw{_} \ts \nt{digit} \} 
                         \OR \kw{'} \{\nt{letter} \ts \nt{digit} \ts \nt{symbol} \ts \nt{white space} \ts \kw{''} \} \kw{'} 
			 \OR \_ \END
%                         \GETON \hspace*{0.2ex} \ts \kw{'} \{\nt{letter} \ts \nt{digit} \ts \nt{symbol} \} \kw{'}\\
\nt{symbol atom}         \IS \nt{symbol} \{ \nt{symbol} \} \ts \kw{[}\,\kw{]} \END
\nt{full stop} 	         \IS \kw{.} \nt{white space} \END
\nt{white space} 	 \IS \kw{\t} \ts \kw{\n} \ts \nt{blank} \END
\\
\nt{Prolog number} 	 \IS \nt{integer} \ts \nt{floating-point number} \END
\end{syntax}

\noindent
(The syntax is given in EBNF with \{\} for repetitions and [ ] for options.
The terminal symbols are in bold face font style.)

We distinguish between {\em many-sorted} and {\em order-sorted} specifications:
A specification is called an {\em order-sorted specification}, if the
base module or the enrichment includes a {\em subsort declaration}. 
Otherwise it is called {\em many-sorted}.

\subsection{Base}
\label{base}

The base has the form

\begin{syntax}
\nt{base} \IS \kw{using} \nt{moduleExpression}
\end{syntax}

The \nt{moduleExpression} describes which modules should be imported and
how certain sorts and operators should be renamed to fit in the 
importing specification.

\begin{syntax}
\nt{moduleExpression} \IS \nt{specificationName} 
 			\OR \nt{moduleExpression} \kw{+} \nt{moduleExpression}
			\OR \nt{specificationName} \kw{(} \nt{AssociationList} \kw{)} \END
\\
\nt{AssociationList} \IS \nt{Association} \{ \kw{,} \nt{Association} \} \END
\nt{Association}         \IS \nt{sortAssociation}
			 \OR \nt{operatorAssociation} \END
\\
\nt{sortAssociation}      \IS \nt{sortName} \kw{<-} \nt{sortName} \END
\nt{sortName} 		   \IS \nt{Prolog atom} \END
\\
\nt{operatorAssociation}   \IS \nt{simpleOpName} \kw{<-} \nt{simpleOpName} \END
\nt{simpleOpName}          \IS \nt{Prolog atom} \ts \nt{Prolog number} \END
\end{syntax}

The semantics of the `\kw{+}'-operation  is union of specifications
(signature, axioms, termination ordering).
The union fails, if the termination orderings cannot be combined
with the heuristics in CEC
(e.g.\ different interpretations are given for the same operator).

In the third case the given specification morphism is applied to a 
module expression.
For the operator renaming injectivity is required. Noninjective
operator renamings must be simulated using auxiliary operators
and equations.

All  specifications which are refered to by their \nt{specificationName}  
in the base of a 
specification are called {\em direct imports} of this specification.


If no base is specified, the module `\file{standard}' is
taken from its specification variable, if present. Otherwise it is taken from
the current directory (thaw file `\file{standard.q2.0}').
If no module standard is found in the current directory,
the module `\file{standard}' from the CEC-distribution is used.
The latter simply declares the sort \cec{bool} and the two constructor
constants \cec{true} and \cec{false} of sort \cec{bool}.

The evaluation of the base depends on the presence and the contents of
an {\em order specification} (cf. chapter \ref{OrderSpecification}).

%The base of a specification is evaluated whenever this specification
%is consulted using the \comRef{in}-command. If the \comRef{in}-command also specifies
%an \nt{orderName} of an orderSpecification, this order specification is
%also consulted. The order specification determines which files must
%be consulted to find the specifications of the direct antecedents.
%If some <orderName> beside `noorder' is associated with the 
%<specificationName> of a direct import, the system tries to thaw 
%a frozen state of a specification (\refArrow \comRef{freeze}) from a file name
%   <specificationName>.<orderName>.q2.0    (Quintus 2.x versions)
%in the current directory (\refArrow \comRef{cd}, \comRef{pwd}).
%If no such files are present the system looks for a file
%	<specificationName>.eqn
%with the text of the specification module according to the syntax of
%<specification> to be read in and for a file
%	<specificationName>.<orderName>.ord
%which contains pragmas concerning the termination ordering.
%If the the <orderName> associated with <specificationName> is 
%'noorder' or there is no <orderName> associated with <specificationName> 
%the system looks only for a file 
%	<specificationName>.eqn
%in the current directory (\refArrow \comRef{cd}, \comRef{pwd}).

\subsection{Enrichment}
\label{enrichment}

An enrichment consists of {\em subsort declarations},
{\em operator declarations}, {\em variable declarations}, 
{\em axioms} and {\em clauses for parsing and pretty printing}:

\begin{syntax}
\nt{enrichment} \IS \{ \nt{subsortDeclaration} \}
 \AND \{\nt{operatorDeclaration} \} 
 \AND [ \nt{variableDeclarations} ] 
 \AND \{ \nt{axiom} \} 
% \AND \{ \nt{pragma} \} 
 \AND [ \nt{clauses for parse and pretty} ]
\end{syntax}

The semantics of a \nt{specification} is the specification
which is obtained by enriching the specified base module
by the constituents of the enrichment. There is no implicit
assumption about consistency or sufficient-completeness of this
enrichment.

\subsubsection{Subsort Declarations}
\label{subsortdecl}

In order-sorted specifications subsort declarations are given 
with regard to the following syntax: 

\begin{syntax}
\nt{subsortDeclaration} \IS \nt{sortName} \kw{<} \nt{sortName} \nt{full stop} \END
\nt{sortName}           \IS \nt{Prolog atom}
\end{syntax}

\subsubsection{Operator Declarations}
\label{operators}

Operator declarations define the {\em signature} of a specification. 
An operator is characterized by its name, which is a Prolog atom, and
its arity.
In addition it is 
possible to declare the {\em constructor} property of an operator. The system
makes use of this information when doing {\em termination proofs} and
{\em consistency checks}. It is not allowed to orient an equation into a rule
such that the top of the left side is a constructor.
Operator declarations may be given according to the following syntax: 

\begin{syntax}
\nt{operatorDeclaration} \IS \nt{opKind} \nt{opName} \kw{:} \nt{arity} \nt{full stop}\END
\nt{opKind} \IS \kw{op} \ts \kw{cons} \END
\nt{opName} \IS \kw{(} \nt{Prolog atom}\kw{,} \nt{priority}\kw{,} \nt{fix} \kw{)}
 \OR \nt{Prolog atom}
 \OR \nt{Prolog number}
\end{syntax}

\noindent
{\em Note that for the first alternative of \nt{opName} there must not be
 any \nt{white space}
between a preceding \nt{opKind} and the open parenthesis ``\kw{(}''. }

\begin{syntax}
\nt{priority} \IS \nt{{\em number from 1 to 1200}} \END
\nt{fix} \IS \kw{fx} \ts \kw{fy} \ts \kw{xfx} \ts \kw{xfy} 
              \ts \kw{yfx} \ts \kw{xf} \ts \kw{yf} \END
\nt{arity} \IS \nt{resultSort} 
 	  \OR \kw{(} \nt{ argumentSorts} \kw{->} \nt{resultSort} \kw{)} \END
\nt{argumentSorts} \IS \nt{sortName} \{ \kw{*} \nt{sortName} \} \END
\nt{resultSort}    \IS \nt{sortName}\END
\nt{sortName}      \IS \nt{Prolog atom} \END
\end{syntax}

As CEC is embedded in a Prolog environment it inherits all facilities of term
notation offered there, e.g.\ the definition of {\em infix, prefix} 
or {\em  postfix} operators. 

Each infix, prefix or postfix operator has a {\em precedence}, which is a
number from 1 to 1200. The precedence is used to disambiguate expressions in
which the structure of the term is not made explicit through the use
of parentheses.  The general rule is that the operator with the {\em highest}
precedence is the principal functor. 

%Thus if \cec{+} has a higher precedence than ``\cec{/}'', then \bigskip
%
%\cec{a+b/c   a+(b/c)} \bigskip
%
%\noindent
%are equivalent and denote the term \cec{+(a,/(b,c))}.
%Note that the infix form of the term \cec{/(+(a,b),c)}
%must be written with explicit brackets\bigskip
%
%\cec{(a+b)/c}  .\bigskip
%
%
If there are two operators in the expression having the same highest 
precedence,
the ambiguity must be resolved from the {\em fixes} of the operators. 
The possible fixes for an infix operator are \bigskip

\cec{xfx   xfy   yfx}  . \bigskip

Operators of fix ``\cec{xfx}'' are not associative: it is required 
that both arguments of the operator must be subexpressions of 
{\em lower} precedence than the operator itself, i.e. their principal 
functors must be of lower precedence, unless the subexpression is
written in parentheses (which gives it zero precedence).
Operators of fix ``\cec{xfy}'' are right-associative:
only the first subexpression must be of lower precedence; the second 
subexpression can be of the {\em same} precedence as the main
operator; and vice versa for an operator of fix ``\cec{yfx}''.

%For example, if the operators ``\cec{+}'' and ``\cec{-}'' both have fix 
%``\cec{yfx}'' and are
%of the same precedence, then the expression\bigskip
%
%\cec{a-b+c}\bigskip
%
%\noindent
%is valid, and means\bigskip
%
%\cec{(a-b)+c} \hspace{2em}     i.e.  \cec{+(-(a,b),c)}.\bigskip
%
%\noindent
%Note that the expression would be invalid if the 
%operators have fix ``\cec{xfx}'', and would mean\bigskip
%
%\cec{a - (b+c)} \hspace{2em}    i.e. \cec{-(a , +(b , c))}\bigskip
%
%\noindent
%if both operators have fix ``\cec{xfy}''.
%
\noindent
The possible fixes for a prefix operator are\bigskip

\cec{fx     fy} \bigskip

\noindent
and for a postfix operator they are\bigskip

\cec{xf     yf} .\bigskip

The meaning of the fixes should be clear by analogy with those for infix operators.
%As an example, if ``\cec{not}'' were declared as a prefix operator of fix 
%``\cec{fy}'', then \bigskip
%
%\cec{not not P} \bigskip
%
%\noindent
%would be a permissible way to write \cec{not(not(P))}. 
%If the fix were ``\cec{fx}'',
%the preceding expression would not be legal, although\bigskip
%
%\cec{not P}\bigskip
%
%\noindent
%would still be a permissible form for \cec{not(P)}.

It is possible to have more than one operator of the same name, as long as they are
of different kinds, i.e. infix, prefix or postfix. An operator of any kind may be
redefined by a new declaration of the same kind, except of some operators
listed in appendix \ref{PredefinedOperators}.

\noindent Examples of operator declarations are: \bigskip

\cec{cons 0 : nat.}\bigskip

\cec{cons(suc,100,fy) : (nat -> nat).}\hfill \bigskip

\cec{op + : (nat * nat -> nat).}\bigskip 

\cec{op =< : (nat * nat -> bool).}\bigskip

\noindent 
Note that in the last declaration the infix notation
\cec{op(=<,700,xfx)} will be inherited from Prolog. 

Overloading of operators is allowed. They will however be renamed internally.

\noindent
The following syntax restrictions serve to remove potential ambiguities associated with
prefix operators.
\begin{itemize}
\item
In a term written in standard syntax, the principal functor and its following 
``\kw{(}'' must {\em not} be separated by any blankspace. Thus
\begin{quotation}
\cec{point (X, Y, Z)} \medskip
\end{quotation}
is invalid syntax.
\item
If the argument of a prefix operator starts with a ``\cec{(}'', this ``\cec{(}'' 
must be separated from the operator by at least one space or other
non-printable character. Thus 
\begin{quotation}
\cec{:-(p ; q), r.}\medskip
\end{quotation}
(where ``\cec{:-}'' is the prefix operator) is invalid syntax, and must be written as
\begin{quotation}
\cec{:- (p ; q), r.}\medskip
\end{quotation}
\item
If a prefix operator is written without an argument, as an ordinary atom, the atom is
treated as an expression of the same precedence as the prefix operator, and must
therefore be written in parentheses where necessary. 
Thus the parentheses are necessary for \cec{-} in 
\begin{quotation}
\cec{op (-) : (nat * nat -> nat).}
\end{quotation}
\end{itemize}
For further details you may have to
look at the Quintus-Prolog user manual. 

\subsubsection{Variable Declarations}
\label{vardecl}

Each Prolog atom in the specification which is not declared as an 
operator will be taken as a variable. To declare the sort of
a variable you only have to add the sortname to the variablename:

\begin{syntax}
\nt{variable} \IS \nt{Prolog atom} [ \kw{:} \nt{sortName} ]
\end{syntax}

In many-sorted specifications without overloading it is not required to
attach a sort to a variable explicitly: A simple type inference mechanism is
used to find the correct sort for the variable. In many-sorted specifications
with overloading this type inference need not produce the desired typing. 
Then it is helpful to define the sorts of variables explicitly.
In order-sorted specifications it is usually not possible to deduce
the sort of a variable. In this case all variables
must be explicitly typed. This can be done by adding
the sort name as described above or by declaring the variable in the
variable declaration part: 

\begin{syntax}
\nt{variableDeclarations} \IS \kw{var} 
 \nt{variableDecl} \{ \kw{,} \nt{variableDecl} \} \nt{full stop} \END
\nt{variableDecl} \IS \nt{Prolog atom} \kw{:} \nt{sortName}
\end{syntax}

The scope of a variable declaration is not bound to the module it is
contained in, e.g.\ the variable declarations in a base module are
inherited by the current specification, if not redeclared.

\subsubsection{Axioms}
\label{axioms}

CEC includes a completion procedure for conditional (and unconditional)
equations. Equations are of the form:

\begin{syntax}
\nt{axiom} \IS \nt{unconditionalEquation} \nt{full stop}
           \OR  \nt{conditionalEquation} \nt{full stop} \END
\nt{unconditionalEquation} \IS \nt{equation} \END
\nt{conditionalEquation}   \IS \nt{condition} \nt{equation} \END
\nt{condition} \IS \nt{equation} \{ \kw{and} \nt{equation} \} \kw{=>}\END
\\
\nt{equation} \IS \nt{signatureTerm} \kw{=} \nt{signatureTerm} \END
\\
\nt{signatureTerm} \IS $<$well typed term with variables
		   \GETON \hspace{1ex} over the user-specified signature$>$
\end{syntax}

\noindent Thus \bigskip

\cec{isChar(c) = true => isLine([c | l]) = isLine(l).} \bigskip

\noindent and\bigskip

\cec{isSpace(c) = true => isLine([c | l]) = isLine(l).}\bigskip

\noindent
are valid conditional equations in a many-sorted specification as well as\bigskip

\cec{isChar(c) = true and isLine(l) = true => isLine([c | l]) = true.}\bigskip

\noindent
where \cec{c} and \cec{l} are variables.\\
In an order-sorted specification the variables must be declared 
in the variable declaration part:\bigskip

\cec{var c : char, l : string.} \bigskip

\noindent
or else the sort must be attached  to each occurrence of a variable:\bigskip

\cec{isChar(c:char) = true => isLine([c:char | l:string]) = isLine(l:string).} \bigskip

\noindent
Unconditional equations are treated as conditional equations with an empty condition.


\subsubsection{Clauses for Parse and Pretty}
\label{ParseAndPretty}

\noindent
The definition of a predicate {\em parse} allows the user to define his
own term notation. If parse fails, standard Prolog term notation is assumed.\bigskip

\begin{command}[\com{parse}{\comArg{Term}\ad\comArg{ParsedTerm}}]
should map the \comArg{Term} to its standard representation \comArg{ParsedTerm} as 
assumed by the system. This representation is the usual Prolog
representation formed from the operator symbols in the signature. The auxiliary
operator '@' is used to denote variables in the term (The term must not
contain any Prolog variables!).\bigskip


Example:\smallskip
\begin{tabbing}
12345\=12345\=eee\kill
\> Specification of integer with operators \cec{s} (successor) and 
\cec{p} (predecessor):\\
\\
\> \> \cec{parse(0,0).}\\
\> \> \cec{parse(I,s(T)) :- integer(I), I > 0, !, IM1 is I-1, parse(IM1,T).}\\
\> \> \cec{parse(I,p(T)) :- integer(I), I < 0, !, I1 is I+1, parse(I1,T).}\\
\\
\> \cec{parse(5, ParsedTerm)} yields \cec{ParsedTerm = s(s(s(s(s(0)))))}.
\end{tabbing}
\end{command}

\begin{command}[\com{pretty}{\comArg{Term}}]
Upon pretty printing of terms the system tries to call 
\com{pretty}{\comArg{Term}} (for each subterm). If this succeeds, 
the system assumes that the (sub-)term has been
printed. Otherwise, \com{print}{\comArg{Term}} is called.\bigskip

Example:\smallskip
\begin{tabbing}
12345\=12345\=eeee\kill
\> Specification of integer with operators \cec{s} (successor) and 
\cec{p} (predecessor):\\
\\
\> \> \cec{pretty(s(X)) :- predicate(fromUnary(s(X),N)), write(N).}\\
\> \> \cec{pretty(p(X)) :- predicate(fromUnary(p(X),N)), write(N).} \\
\\
\> \cec{pretty(s(s(s(s(s(0))))))} prints out {\tt 5}.
\end{tabbing}
\end{command}

\noindent
One may define arbitrary auxiliary predicates (i.e \comRef{fromUnary})
for the definition of \comRef{parse} or \comRef{pretty}. 
To prevent name clashes with internal CEC predicates
you have to enclose their definitions and calls in $predicate(\ldots)$. \bigskip

Example:\smallskip
\begin{tabbing}
12345\=12345\=eee\kill
\> Specification of integer with operators \cec{s} (successor) and 
\cec{p} (predecessor):\\
\\
\> \> \cec{predicate(fromUnary(X,_)) :- var(X), !, fail.}\\
\> \> \cec{predicate(fromUnary(0,0)).}\\
\> \> \cec{predicate(fromUnary(s(X),N1)) :- predicate(fromUnary(X,N)), N1 is N+1.}\\
\> \> \cec{predicate(fromUnary(p(X),N1)) :- predicate(fromUnary(X,N)), N1 is N-1.}
\end{tabbing}

If the current specification is a combination of several specifications, the clauses for \comRef{parse},
\comRef{pretty}, and \comRef{predicate} will be put together in
{\em arbitrary order}. Note that upon renaming of specifications parse and
pretty might no longer work for the renamed signature.

\subsection{Specification of Termination Orderings}
\label{OrderSpecification}

\noindent
While running the completion process you are interactively asked for additional
information, mainly
concerning the construction of the termination ordering for the given
specification. The system supports this process by making suggestions for precedence
extensions (path orderings). In many cases you will know a 
priori many of these properties. 
Then you can supply a corresponding {\em order specification} for your
initial specification to avoid unnecessary computations of suggestions 
and to reduce the number of CEC-questions.\bigskip

\begin{syntax}
\nt{orderSpecification} \IS \kw{order} \nt{orderType} \kw{for} \nt{specificationName} 
			 \AND [ \nt{orderBase} ] 
			 \AND [ \nt{orderPragmas} ] \END
\\
\nt{orderType} \IS  \kw{kns}
		\OR \kw{neqkns} 
		\OR \kw{poly}\nt{natural number}
		\OR \kw{manual}
\end{syntax}

\noindent
The order specification determines the termination ordering to be used for 
\nt{specificationName}, the order names for its direct imports (\refArrow
\ref{orderBase}) and gives precedence declarations for the operators of the
specification (if the termination ordering is \kw{kns} or \kw{neqkns}) or
polynomial interpretations (if the termination ordering is \kw{poly}\nt{N})
(\refArrow \ref{OrderPragmas}).

Available types of orderings are
\begin{itemize}
\item \kw{kns} \\
This ordering is a recursive path ordering based on the recursive comparison of paths
according to the definitions in \cite{KNS85}.
\item \kw{neqkns} \\
This ordering is a restriction of the kns ordering:
Different operators cannot have the same precedence.
\item \kw{poly}\nt{N}\\
Here, terms are given polynomial interpretations over +, * where
coefficients are natural numbers.
The order of terms is reduced to comparing polynomials in the order on natural
numbers (\cite{CL86}).
\item \kw{manual}\\
This is not a termination ordering because the user is asked to decide the
orientation of the rules manually. Therefore the termination and confluence of the
final rewrite system cannot be guaranteed
\footnote[1]{In CEC it is even more dangerous to switch to manual mode, as
the termination order is lifted internally to an ordering of proofs in
equational logic. The latter is used for elimination of conditional 
equations. This effect is completely outside of the user's control.}.
\end{itemize}

\noindent You should write order specifications which belong to the
specification \nt{specificationName} in a file named 

\hbox to \hsize{\hfill
\nt{specificationName}\kw{.}\nt{orderName}\suffix{.ord}, \hfill} 
\noindent where \nt{orderName} can be any Prolog atom.
\nt{orderName} is then the {\em current order name} of the specification
\nt{specificationName}. If no order specification is supplied, the current
order name is the actual termination ordering, which is \kw{neqkns} or
\kw{poly}\nt{N} depending on the presence of AC-operators. 
The termination ordering can be changed using the \comRef{order}-command.  Of
course it is also possible to have more than one order specification for a
single specification, simply distinguished by different order names.



\subsubsection{Order Base}
\label{orderBase}

\noindent
An order specification is an enrichment of an {\em order base definition}:

\begin{syntax}
\nt{orderBase} \IS \kw{using} \kw{(} \nt{orderExpression} \kw{)} \nt{full stop} \END
\\
\nt{orderExpression} \IS  \nt{orderName} \kw{for} \nt{specificationName} 
		      \OR  \nt{orderExpression} \kw{+} \nt{orderExpression} \END
\\
\nt{orderName} \IS \kw{noorder}
		\OR \nt{Prolog atom except \kw{noorder}}
\end{syntax}

For an explanation of the interpretation of the order base refer to the description of the
\comRef{in}-command (\refArrow chapter \ref{InCommand}).
\subsubsection{Order Pragmas}
\label{OrderPragmas}

\begin{command}[\com{constructor}{\comArg{OpName}}]
declares operator \comArg{OpName} to be a constructor operator. 
For order-sorted specifications also 
the many-sorted disambiguated operators are allowed for \comArg{OpName}.
\end{command}

\noindent
{\bf CEC Pragmas concering the termination orderings KNS and NEQKNS:}\bigskip

\begin{command}[\com{equal}{\kw{[}\kw{[}\comArg{a}\kw{,}\comArg{b}\kw{,}\comArg{c}\kw{,}\ldots\kw{]}, 
\kw{[}\comArg{g}\kw{,}\comArg{h}\kw{,}\comArg{i}\kw{,} \ldots \kw{],} \ldots \kw{]}}\kw{.}]
declares operators to have equivalent precedences (only allowed for \kw{kns}).\\
Meaning: \comArg{a} = \comArg{b} = \comArg{c} = \ldots and 
\comArg{g} = \comArg{h} = \comArg{i} = \ldots
\end{command}

\begin{command}[\com{greater}{\kw{[}\kw{[}\comArg{a}\kw{,}\comArg{b}\kw{,}\comArg{c}\kw{,} \ldots \kw{],}
\kw{[}\comArg{g}\kw{,}\comArg{h}\kw{,}\comArg{i}\kw{,} \ldots \kw{],} \ldots \kw{]}}\kw{.}]
adds ordered pairs of operators to the precedence.\\
Meaning: \comArg{a} $>$ \comArg{b} $>$ \comArg{c} $>$ \ldots 
and \comArg{g} $>$ \comArg{h} $>$ \comArg{i} $>$ \ldots
\end{command}

\begin{command}[\com{status}{\kw{[}\comArg{Operator} : \comArg{Status}\kw{,} \ldots \kw{]}}\kw{.}]
declares that \comArg{Operator} has status \comArg{Status} provided 
\kw{kns} or \kw{neqkns} is the chosen termination ordering.\\
\comArg{Status} can be
\kw{lr} for {\em left-to-right}, \kw{rl} for {\em right-to-left} or
\kw{ms} for {\em multiset}.
\end{command}

\noindent
{\bf CEC Pragmas concerning the termination ordering POLY$<$N$>$:}\bigskip

\begin{command}[\com{setInterpretation}{\kw{[}\comArg{Operator(Arguments)} : \comArg{Interpretation}\kw{,} \ldots\kw{]}}\kw{.}]
Provided that \kw{poly}\nt{N} is the current termination ordering a new interpretation
\comArg{Interpretation} for an operator \comArg{Operator} is added to the current state. The interpretation
\comArg{Interpretation} must be a polynomial over the variables in 
\comArg{Arguments} (if N = 1) or a list with 
 N such polynomials. Commutative and associative-commutative operators
require interpretations of a particular kind.
An example: \bigskip

\cec{setInterpretation(['=='(x, y) : x + y]).}\bigskip

\noindent
For more detailed explanations concerning termination proofs based on polynomial 
interpretations see below. Initially operators have no interpretations.
\end{command}

These CEC-pragmas should not be used interactively while CEC is running. Use the
corresponding CEC-commands  \comRef{constructor}, \comRef{equal}, \comRef{greater}, \comRef{status}
or \comRef{setInterpretation} without parameters which 
request their input interactively.\bigskip


\subsection{Syntax and Type Errors}

When reading in a specification two kinds of errors may occur:
\begin{itemize}
\item
The Prolog parser detects a syntax error at the level of Prolog terms. 
\item
The CEC-parser finds a syntax error.
\item
The type checker detects a type inconsistency or is unable to infer a type for some
term because of missing operator definitions.
\end{itemize}

\noindent
In this case the specifications is rejected and CEC sets back to its
state before reading.

   % --.4.89

\section{Input and Output of Specifications}
\label{InputOutput}

CEC has a variety of different I/O-commands. There are two commands for reading in
original specifications from a file (\comRef{in} and \comRef{enrich})
two commands for saving and restoring partially completed specifications 
to/from files (\comRef{freeze} and \comRef{thaw})
two commands for saving and restoring specifications to/from 
specification variables (\comRef{store} and \comRef{load}) and
two commands for storing and loading {\em log-files}
(\comRef{storeLog} and \comRef{loadLog}).
Also, the commands for saving and 
restoring the whole CEC state, \comRef{saveCEC} and \comRef{restoreCEC}, can be used. But
notice that saving CEC states (prolog states) usually generates very
large files. 
%Conventions concerning the allowed file and variable names are 
%those of the underlying prolog system. 
Reading specifications from standard input
(keyboard, terminal) is possible by using the predefined file name \cec{user}.


\subsection{Reading Specifications from Files}
\label{InCommand}

\begin{command}[\com{in}{\comArg{ModuleName}\ad\comArg{OrderName}}]
reads in a specification from the file \comArg{ModuleName}\suffix{.eqn}
and the associated order specification from the file
\comArg{ModuleName}\suffix{.}\comArg{OrderName}\suffix{.ord}.
As log-files are ``enriched'' order specifications, any log-file can
be used as an order file.
The  order  base of the order specification determines  the order names for the direct
imports of a specification. The system will first look for the
specification variable 

\hbox to \hsize{\hfill
\nt{specificationName}\kw{.}\nt{orderName}\hfill.}
If such a variable exists its content will be used. If such a variable does not
exist and some \nt{orderName} $\neq$ \kw{noorder} is associated with  
\nt{specificationName} the system will look for the file

\hbox to \hsize{\hfill
\nt{specificationName}\kw{.}\nt{orderName}\suffix{.q2.0}\hfill}
\noindent in the current directory containing a frozen state of the
referenced specification.
\noindent
If this file does'nt exist the system looks for a file

\hbox to \hsize{\hfill
\nt{specificationName}\suffix{.eqn}\hfill}
\noindent
which contains the specification according to the syntax of 
\nt{specification} and for a file 

\hbox to \hsize{\hfill
\nt{specificationName}\kw{.}\nt{orderName}\suffix{.ord}\hfill}
\noindent
which contains the order specification for the specification.
If the \nt{orderName} associated with \nt{specificationName} is 
\kw{noorder} the system looks for the specification in file

\hbox to \hsize{\hfill
\nt{specificationName}\suffix{.eqn}\hfill}
\noindent
in the current directory.

If not stated otherwise these files are assumed to be in the current 
directory. Before the specification is read in, CEC will be re-initialized, 
e.g. the current specification will be deleted. Specifications saved in 
variables will not be affected. \comArg{ModuleName} and \comArg{OrderName}
must be Prolog atoms. \comArg{OrderName} becomes the
current order name for the specification.\\
\com{in}{\comArg{ModuleName}\ad\cec{noorder}}
has the effect that no order specification is consulted.
The termination ordering for the new
specification will be initialized to a default value 
(\kw{neqkns} or \kw{poly1}, depending on the presence of AC-operators).
Using \comRef{in} only with the parameter \comArg{ModuleName} yields the same
effect. \comArg{ModuleName} = \kw{user} expects input from terminal.
(For Quintus-Prolog2.x under EMACS: {\it in} without
parameter reads from \cec{Scratch.pl}).
\end{command}

If you want to access files from other directories you will have to specify these
directories relative to the current directory at the time of invoking Prolog, e.g.
\cec{in('examples/math/int')}. A more convenient way will be to specify the
necessary prefix to all files once and for all by\bigskip

\begin{command}[\com{cd}{\comArg{Path}}]
Changes, as the cd command in UNIX, the directory for all following
file-related CEC-commands.
The path is given in form of a Prolog atom, hence don't forget the
quotes, if the path contains `\kw{/}', `\kw{.}', or `\kw{..}' or other special
characters.

\comRef{cd} is predeclared as prefix operator. So after execution 
of \bigskip

\cec{cd 'examples/math'.} \bigskip

\noindent
the command \cec{in(int)} will read in \cec{examples/math/int.eqn}.

\comRef{cd} without an argument resets the current directory to the one in
which the CEC-system was initially invoked.
\end{command}

\begin{command}[\comName{pwd}]
prints out the current path for I/O-related commands.
\end{command}

\begin{command}[\com{enrich}{\comArg{ModuleName}\ad\comArg{OrderName}}]
reads in additional parts of a specification from the files 
\comArg{ModuleName}\suffix{.eqn} and \comArg{ModuleName}\suffix{.}\comArg{OrderName}\suffix{.eqn}
after saving the current state for later checks for consistency of the enrichment. 
These additional parts must form an enrichment (cf. chapter~\ref{enrichment}).
\comArg{ModuleName} and \comArg{OrderName} can be arbitrary Prolog atoms.
The \comArg{OrderName} or even both arguments 
can be omitted, with similar effects as for \comRef{in}.
%Specify \comArg{ModuleName} = \kw{user} if input from terminal is wanted.
\end{command}

\subsection{Freezing and Thawing Partially Completed Specifications to/from Files}
\label{FreezeCommand}
\label{ThawCommand}

For saving and restoring states of specifications to/from files there exist the 
commands \comRef{freeze} and \comRef{thaw}.\bigskip

\begin{command}[\com{freeze}{\comArg{ModuleName}\ad\comArg{OrderName}}]
writes the state of the current specification to the file
\comArg{ModuleName}.\comArg{OrderName}\suffix{.q2.0}.
If freeze is called without \comArg{OrderName} the state is written
to \comArg{ModuleName}\suffix{.q2.0}.
%the current order name is taken for \comArg{OrderName}. 
If freeze is called without any argument at all the module name of the current 
specification is used for \comArg{ModuleName} and the current order name
is used for \comArg{OrderName}.
In this case, also a log-file is produced.
The specification may later be reused by thawing it from this file, cf. the 
\comRef{thaw}-command.
The state of CEC remains unchanged by this operation.
\end{command}

\begin{command}[\com{thaw}{\comArg{ModuleName}\ad\comArg{OrderName}\ad\comArg{SpecificationVariable}}]
This command is the inverse operation of \comRef{freeze} and restores the specification
previously frozen in \comArg{ModuleName}\suffix{.}\comArg{OrderName}\suffix{.q2.0}
to the specification variable \comArg{SpecificationVariable}. The current specification and other variables 
will not be affected by this operation.
If \comRef{thaw} is called without \comArg{SpecificationVariable} 
the current specification is overwritten by the thawed specification. 
Specifications stored in variables will not be affected by this operation.
If \comRef{thaw} is used only with argument \comArg{ModuleName} the frozen 
specification will be taken from the file \comArg{ModuleName}\suffix{.q2.0}.
\end{command}

\subsection{Storing and Loading of Log-files}
\label{StorelogCommand}
\label{LoadlogCommand}

Log-files are used to save the information given from the user during the last
completion process and to save the definition of the current termination
ordering. The log-file can be used to replay a completion fully (or partially)
automatically on ``closely related'' specifications. \bigskip

\begin{command}[\com{storeLog}{\comArg{ModuleName}\ad\comArg{OrderName}}]
creates the log-file.
The name of the log-file is 
\comArg{ModuleName}\suffix{.}\comArg{OrderName}\suffix{.@.ord}.
It has the format of an order specification file which can be used
with the \comRef{in}-command or the \comRef{loadLog}-command. 
If \comRef{storeLog} is used only with argument \comArg{ModuleName}
the log-file is named \comArg{ModuleName}\suffix{.@.ord},
if \comRef{storeLog} is used without any argument, the file is
created as \nt{moduleName}\suffix{.}\nt{orderName}\suffix{.@.ord},
with names as they are currently associated with the specification.
\end{command}

\begin{command}[\com{loadLog}{\comArg{ModuleName}\ad\comArg{OrderName}}]
reads in the file \comArg{ModuleName}\suffix{.}\comArg{OrderName}\suffix{.@.ord}. 
%This file contains all the answer given during
%the completion process and the final termination ordering saved using
%the \comRef{saveLog}-command. 
If the completion process is started again, % all
questions whose answers are already contained in 
\comArg{ModuleName}\suffix{.}\comArg{OrderName}\suffix{.@.ord} 
will be suppressed. 
If \comRef{loadLog} is used without the argument \comArg{OrderName} the 
information will be taken from the file \comArg{ModuleName}\suffix{.@.ord}.
If \comRef{loadLog} is used without any argument
the name of the current specification together with the current order name
will be used.
%The current order name is the
%order name of the order specification for the current specification
%or the current termination ordering if no order specification was used.
\end{command}

\subsection{Assigning and Retrieving Specifications to/from Specification Variables }
\label{StoreCommand}
\label{LoadCommand}

Some operations on specifications like \comRef{combine}, cf. the
\comRef{combineSpecs}-command, need a way to reference different specifications. In CEC
this is done by storing specifications to named variables and referencing them
afterwards by these variable names. Variable names are arbitrary Prolog atoms.\bigskip

\begin{command}[\com{store}{\comArg{ModuleName}\ad\comArg{OrderName}}]
saves the current specification in a specification variable named
\comArg{ModuleName}\kw{.}\comArg{OrderName}. If \comRef{store} is used only with
argument \comArg{ModuleName} the specification is saved in a variable with this name,
if \comRef{store} is used without any argument, the name is created as 
\nt{moduleName}\kw{.}\nt{orderName}, with names as they are currently associated 
with the specification. For later restoring 
use the command \comArg{load}. The system remains unchanged except for this variable 
containing afterwards the current specification.
\end{command}

\begin{command}[\com{load}{\comArg{ModuleName}\ad\comArg{OrderName}}]
loads the system which is currently the value of the variable
\comArg{ModuleName}\kw{.}\comArg{OrderName}, cf. the \comRef{store}-command. 
If \comRef{load} is used only with argument \comArg{ModuleName}, this actual parameter 
completely specifies the name of the variable.
Specification variables remain unchanged.
\comArg{StateName} = \cec{'$initial'} re-initializes the system.
\end{command}

   % 2.5.89
\section{Displaying and Changing the State of a Specification}

\subsection{Displaying the State of a Specification}
\label{Displaying}

\begin{command}[\comName{moduleName}]
displays the name of the current specification.
\end{command}

\begin{command}[\comName{orderName}]
displays the name of the order specification associated with
the current specification.
\end{command}

The state of a specification consists of the current signature, the set of equations
and rules and of the current definitions for the termination ordering. There are
several commands to print out this information.\bigskip

\begin{command}[\comName{sig}]
prints out the signature of the current specification:
\end{command}

If you are writing an order-sorted specifications there exist
additional, automatically generated operators which result from the 
translation into a many-sorted specification:
For every subsort relation s $<$ s' there will be an injective operator
with domain s und codomain s', called \verb|$inj-|s'\verb|-|s. If the domain can be
determined from the context, \verb|$inj-|s'\verb|-|s is abbreviated by s'.
These additional operators are also shown by the \comRef{sig}-command.\bigskip

\begin{command}[\comName{show}]
shows the sets of equations, rules and nonoperational equations of the 
current specification in order-sorted form.
(Usually there exists more than one many-sorted representation of
an order-sorted axiom.)

Rules \condRule{C}{\rewRule{l}{r}} which are marked by an asterix 
\kw{*} have an associated auxiliary rule of form
\condRule{C}{\rewRule{l+X}{r+X}}
where $+$ is the AC-operator on top of $l$ and $X$ is a new
variable of appropriate sort.
Auxiliary rules are automatically generated when needed during
completion modulo AC.
\end{command}

\begin{command}[\comName{showms}]
shows the sets of equations, rules and nonoperational equations of the 
current specification in many-sorted form.
\end{command}

\subsection{Inspecting the Termination Ordering}
\label{InspecTermination}

\begin{command}[\comName{order}]
indicates the current termination ordering and asks the user whether he wants to 
change it.
\end{command}

\begin{command}[\comName{operators}]
displays all precedences and stati in \kw{kns} or \kw{neqkns} or
all polynomial interpretations in \kw{poly}\nt{N} respectively.
\end{command}

\begin{command}[\comName{interpretation}]
displays all operator interpretations, provided \kw{poly}\nt{N} is the chosen termination
ordering and asks the user if he wants to change any. {\em If so all rules will be turned 
back into equations.}
\end{command}


\subsection{Checking Preregularity and Regularity}

To ensure that the set of order-sorted equations of the specification and
the set of rewrite rules produced by CEC describe the same equational
theory, the signature must be {\em preregular} (\refArrow \cite{SNGM87}).
If in addition the signature is {\em regular} the term algebra is an
initial algebra in the semantics of \cite{GM87} (and \cite{SNGM87}).

\noindent
There are commands which check preregularity and regularity:\medskip

\begin{command}[\comName{preregular}]
succeeds if the current (order-sorted) signature is preregular,
and fails otherwise.\\
The preregularity condition is the regularity of 
\cite{SNGM87}:
A signature is preregular, iff for every function symbol
$f$ and every string $w$ of sorts the set
$$   \{ t \mid \mbox{\ there is a declaration\ } f : w' \rightarrow t 
\mbox{\ such that\ } w \leq w' \} 
$$
is either empty or has a minimal element.
\end{command}

\begin{command}[\comName{regular}]
succeeds if the current (order-sorted) signature is regular,
and fails otherwise.
The regularity condition is the one of \cite{GM87}:
A signature is regular, iff for every function symbol
$f$ and every string $w$ of sorts the set
$$
   \{ (w',t) \mid \mbox{\ there is a declaration\ } f : w' \rightarrow t 
      \mbox{\ such that\ } w \leq w' \}
$$
is either empty or has a minimal element.
\end{command}

\subsection{The UNDO Mechanism}

\begin{command}[\comName{undo}]
can be entered at the system's top level to reset the system to the state
before the last command that has caused a state change, if there was
any. \comRef{undo} can be used repeatedly to undo several steps of state changes.
It also undoes \comRef{undoUndo}-calls. At the moment, there is no way to backtrack
from single decisions that have been taken while running the completion
process.
\end{command}

\begin{command}[\comName{undoUndo}]
allows to undo the last \comRef{undo}-command at the system's top level. Repeated use of
this command undoes sequences of \comRef{undo}-commands. Chains of undo-calls begin at 
the last user interaction different from an \comRef{undo} or \comRef{undoUndo}.
\end{command}


  % 16.3.89
\section{Operations on Specifications}
\label{OperationsOnSpecifications}

The specification building operators \kw{+} and \kw{rename} in 
module expressions can also be invoked interactively.
%As CEC is planned to be an experimental system for specification (program) development
%it offers several operations for combining specifications. The operations include the
%renaming of operator and sort names as well as the union of two specifications.

With these primitive operations more complex ones, such as passing actual parameters to
specification modules can be derived, cf. chapter \ref{exampleSession}.
%The parameter passing is
%demonstrated by the example of sorted lists over an arbitrary ordered set, that is
%replaced by the set of natural numbers \\
%(\refArrow file \file{cec/KBMANUAL/combining\_example} in this distribution).

\subsection{Renaming of Operator and Sort Names}

\begin{command}[\com{renameSpec}{\kw{[}\comArg{OldSort1} \kw{<-} \comArg{NewSort1}, \ldots , 
\comArg{OldSortN} \kw{<-} \comArg{NewSortN},\\ \hspace*{6.2em}
\comArg{OldOperator1} \kw{<-} \comArg{NewOperator1}, \ldots , 
\comArg{OldOperatorM} \kw{<-} \comArg{NewOperatorM}\/\kw{]}}]
renames the current specification according to the given lists of sort associations
and operator associations. Only injective renamings of operators are allowed.
Sorts may be renamed arbitrarily. Sorts and operators which are not mentioned remain 
unchanged.
\end{command}

%The system tries to carry over the  previous termination proof to the renamed
%specification. This is not always possible, as the collapsing of previously distinct
%operator symbols might not be compatible with precedences or abstract interpretations.
%In such a case, all rules will be turned back into equations.


\subsection{Combining of Specifications}

The combine operator forms the union of two specifications if they can be combined, 
that is if their termination orderings can proved to be compatible. If so, the two orderings
are combined. \cec{kns} and \cec{poly}\nt{N} are assumed to be incompatible.\bigskip

\begin{command}[\com{combineSpecs}{\comArg{StateName1}\ad\comArg{StateName2}\ad\comArg{CombinedSpec}}]
The specifications  stored (\refArrow \comRef{store}) 
in specification variables \comArg{StateName1} and \comArg{StateName2}
will  be combined, if possible, by forming
the union of the signature, axioms and pragmas.
If \comArg{CombinedSpec} $\neq$ \cec{user}, the result will be stored in
\comArg{CombinedSpec}, and the current specification will not be affected.
Otherwise, the combined specification becomes the new current
specification.
\comRef{combineSpecs} requires the compatibility of the termination
orderings of the involved specifications:
\begin{itemize}
\item
Termination orderings of the same type may be combinable, also \kw{kns} together with
\kw{neqkns}, yields \kw{kns} for the combined specification.
In the case of \kw{kns} and \kw{neqkns} 
the combination fails if the operator precedences are contradictory,
e.g.\ $f > g$ in the first specification but $g \geq f$ in the second 
specifcation. Also, different stati for operators are not compatible.
\item
\kw{poly}\nt{N} and \kw{poly}\nt{M} can be combined 
into \kw{poly}\nt{N} if $N \geq M$. In addition, operators 
which occur in both operand specifications must have
the same interpretation polynomials in the first M components.
\end{itemize}
\end{command}
   % 2.5.89
\section{Completion of Specifications}

\subsection{The Objectives of Completion}

CEC is designed to support the methodology of software specification 
using modular order-sorted specifications with conditional equations.
Being considerably more concise and abstract than many-sorted specifications,
order-sorted specifications even more call for a system that provides
proof techniques needed for
\begin{itemize}
\item checking the consistency of a specification,
\item proving the correctness of actual specification parameters to
parametric modules, and
\item checking, or achieving by transformation, confluence and termination
as a prerequisite
for correct operational execution of specifications by (conditional)
term rewriting.
\end{itemize}
The completion procedure in CEC can be seen as both a compiler and a refutation
proof procedure.
For the latter, CEC distinguishes between two kinds of operators.
Operators can be declared
as {\em constructors} or as regular operators.
It is assumed that any two constructor terms
are different in a consistent equational theory.
If the completion procedure in CEC infers an equation between two
different constructor terms it will stop and report the inconsistency.
One application of this is checking the consistency of parameter passing.
Actual and formal parameter specifications are combined, possibly after
some renaming of sorts or operators, and then completed.
If the actual parameter is {\em constructor-complete} and the completion
process 
discovers no inconsistency then the actual parameter is correct, i.e.
the initial algebra of the actual parameter specification is a model of the
formal one.
This is exactly the {\em ``proof-by-consistency''} method of inductive theorem
proving, and in \cite{HH80} it is shown how to incorporate it into
a completion procedure for the particular case of constructor-complete
theories. In CEC this method is extended to the conditional case.


Unfailing conditional completion
transforms a finite initial set of conditional 
equations $E$ into a possibly infinite
set $E'$ of final conditional equations such that
\footnotetext[1]{
\rew{E} denotes conditional rewriting using the equations in $E$ from left to right,
by \rewRT{E} and \rewSRT{E} we denote the reflexive-transitive and
the reflexiv-transitive-symmetric closure of \rew{E} respectively.}
$$\eqrel{*}{E} \ =\  \rewRT{R(E')} \irewRT{R(E')} \ =:\ \nfrel{R(E')},\footnotemark[1]$$
where $R(E') = \{ \sigma(e)| e \in E' \cup E'^{-1},
\sigma(e) {\rm\ reductive} \}$\footnote[2]{
$E'^{-1} = \{ \condEq{C}{\eq{t}{s}} \ |\  \condEq{C}{\eq{s}{t}} \in E'\}$.}.
This means that, for any two terms
$t_1$ and $t_2$, $t_1$
and $t_2$ are equal in the equational theory, iff 
$t_1$ and $t_2$ have the same normalform with
respect to $R(E')$. $R(E')$ is the set of all reductive
instances of the
equations in $E'$. With the definition of reductivity
as given below, unfailing completion of unconditional equations
is just a particular case of the conditional one.
Rewriting in $R(E')$ means that a redex is always replaced
by a smaller one. In the case of a conditional equation
it also means that the condition instance must be smaller
than the redex, too.

The notion of a reductive conditional equation has originated from
\cite{Kap84},\cite{Kap85} and \cite{JW86}. 
A {\em reduction ordering} is a partial ordering on terms that is
stable under substitution, monotonic and well-founded.
Given a reduction ordering $>$
on the term algebra, an equation
$$ \condEq{\cond{\eq{u_1}{v_1}\condAnd\eq{u_2}{v_2}}
                {\eq{u_n}{v_n}}}
          {\eq{s}{t}} $$
is called {\em reductive}, if $s > t$, $s > u_i$,  and $s > v_i$,
i.e. if the term on the right side and each term that occurs in the condition
are smaller than the left side of the equation. 
In the confluent case, verifying an instance $\eq{u}{v}$ of a condition
equation then means to check whether or not
both terms $u$ and $v$ have the same
normal form. As these terms are smaller than the redex, the applicability
of a reductive equation is always decidable.
 
The main problem is that $R(E')$ is infinite even for finite $E'$.
A particularly fortunate case is when all equations in the final
$E'$ are reductive. As reduction orderings are stable under substitutions,
$\eq{\rew{R(E')}}{\rew{E'}}$, in this case.
Unfortunately in almost all cases $E'$ will not be uniformly reductive.
This is in particular so when the initial $E$ already contains
nontrivial nonreductive equations.
Any equation with extra variables is nonreductive.
Even if the initial equations are reductive, nonreductive ones
can be generated from these during completion.

\subsection{CEC Completion}
\label{Completion}

The completion procedure in CEC is based on three
observations:
\begin{enumerate}
\item
Reductivity can be weakened to what we call {\em quasi-reductivity}.
\item
Depending on the completion strategy, $E'$ can be redundant.
It can contain {\em nonoperational equations}.
$e \in E'$ is called nonoperational, if $\nfrel{R(E')} = \nfrel{R(E'-\{e\})}$,
i.e. if $e$ need not be used for computing normal forms.
\item
The completion strategy can be tailored so
that {\em arbitrarily selected equations} with at least one condition become
nonoperational in the final system.
\end{enumerate}
Hence, whenever the reductivity of a conditional equation is
violated because of the ``size'' of a particular condition, two cases may occur.
If the equation is quasi-reductive, then it can be handled
almost as if it was reductive. Otherwise, the equation
can be classified as ``should become nonoperational''. Then, completion
computes superpositions on its condition such that
in the final system the equation is nonoperational indeed. 

Let's now explain the notion of quasi-reductivity more precisely.
Let
$$ \condEq{\cond{\eq{u_1}{v_1}\condAnd\eq{u_2}{v_2}}
                {\eq{u_n}{v_n}}}
          {\eq{s}{t}} $$
be a conditional equation, $n\ge 1$. It is called {\em quasi-reductive}
if
there exists a sequence $h_i(\xi)$ of context terms,
such that $s>h_1(u_1)$, $h_i(v_i)\ge h_{i+1}(u_{i+1})$,
$1\le i < n$, and $h_n(v_n)\ge t$.

Quasi-reductivity is a proper generalization of reductivity.
If the equation
$$\condEq{\cond{\eq{u_1}{u_{n+1}}}
               {\eq{u_n}{u_{2n}}}}
         {\eq{s}{t}}$$ 
is reductive, then the equation
$$
\condEq{\cond{\eq{u_1}{x_1}\condAnd\eq{u_{n+1}}{x_1}}
             {\eq{u_n}{x_n}\condAnd\eq{u_{2n}}{x_n}}}
       {\eq{s}{t}},
$$
is quasi-reductive,
if the $x_i$ are new, pairwise distinct variables.
Hence, where appropriate we assume that reductive equations are
identified with quasi-reductive ones in the way just explained.

Quasi-reductive conditional rewriting $\rewqr{E}$
with quasi-reductive equations $E$
is defined as classical reductive rewriting with
conditional equations.
The only difference is that goal solving for the conditions is restricted to
{\em oriented goal solving}.
Hereby a substitution $\sigma$ is a solution to a
condition equation $\eq{u}{v}$,
if $u\sigma\rewqrnf{E}v\sigma$, i.e. the right side
$v$ of the condition matches
the normal form of the $\sigma$-instance $u\sigma$ of the left side.
The reductivity requirements for quasi-reductive equations
now imply that $\rewqr{E}\subset\rew{E}$, i.e. any step
of quasi-reductive rewriting is at the same time a step of reductive
rewriting. Moreover, for a quasi-reductive equation and
any matching substitution $\sigma$ for the left side $s$
there is at most one extension of $\sigma$ to a directed solution of
the condition. In addition, $\sigma$ can be constructed deterministically if
$\rewqr{E}$ is confluent. Altogether we see that $\rewqr{E}$
is as efficient as rewriting with reductive equations.

To sum up, completion in CEC produces a possibly infinite $E'$ from an initially given $E$
such that
$$ E' = R_\infty + N_\infty, $$
where
$R_\infty = R_\infty^{red} + R_\infty^{qred}$ is a set of {\em reductive} equations and
{\em quasi-reductive} equations 
and $N_\infty$ is a set of {\em nonoperational} equations,
such that $\rew{R(E')}$ is confluent,
$$\rewSRT{E} = \nfrel{R(E')}$$
{\em and}
$$\rew{R(E')} \subset
\rewqrRT{R_\infty} \irewqrRT{R_\infty}
= \rewSRT{E'}.$$
Hence $\rewqr{R_\infty}$ is confluent too,
efficiently computable and
produces the same normal forms as $\rew{R(E')}$.
%Note that since
%$\rew{E'} \subset \rew{R(E')}$ but $\rew{E'} \neq \rew{R(E')}$
%in general, to achieve
%$$\rew{R(E')} \subset \rewqrRT{R_\infty} \irewqrRT{R_\infty}$$
%completion must perform sufficiently many paramodulation superpositions
%on the right sides of the conditions of the equations in $R_\infty^{qred}$.

We will formalize the CEC-completion procedure within the framework of
an inference system. Since we distinguish between equations, rewrite rules and
nonoperational equations, the objects of this inference system are tupels 
$\tripel{E}{R}{N}$, where $E$ is a set of conditional equations,
$R$ is a set of (quasi-) reductive conditional rewrite rules and $N$ is a
set of nonoperational equations.

The CEC\--com\-pletion procedure computes (possibly infinite) se\-quen\-ces 
$$\tripel{E_0}{R_0}{N_0},\ \tripel{E_1}{R_1}{N_1}, \ldots$$ 
of derivations
using the in\-fer\-ence rules presented below. The limit of a derivation is the
tuple $\tripel{E_\infty}{R_\infty}{N_\infty}$. 
In the following, $P_\tripel{E}{R}{N}\ :\ \eq{s}{t}$ means that $P$ is a proof of
\eq{s}{t} consisting of applications of equations and rules in \tripel{E}{R}{N}. 

The basic idea behind the completion is that in any new state
$\tripel{E_i}{R_i}{N_i}$ proofs become simpler w.r.t. 
a well-founded {\em proof ordering} \PO .
The simplest proofs of equations are rewrite proofs, cf. \cite{Gan87b}.
The goal
 is to construct for each proof 
$P_\tripel{E_0}{R_0}{N_0}\ :\ \eq{s}{t}$ a rewrite proof 
$Q_\tripel{E_i}{R_i}{N_i}\ :\ \eq{s}{t}$ at some step $i$,
yielding 
$$\rewSRT{E} = \nfrel{R_\infty}$$
in the limit.
With this idea in mind we say
that a conditional equation \condEq{C}{\eq{s}{t}} is {\em trivial} if for all
proofs $P : \eq{m}{n}$, which apply \condEq{C}{\eq{s}{t}} there exists a proof
$Q : \eq{m}{n}$ such that $P \PO Q$.

\noindent
If for some $k$ 
\begin{itemize}
\item all critical pairs between rules in $R_k$ and 
\item all superpositions on some condition of any  equation in $N_k$
by the rules in $R_k$ 
\end{itemize}
have been computed and 
\begin{itemize}
\item all equations in $E_k$ and 
\item all unconditional equations in $N_k$ (except for the AC-axioms)
\end{itemize}
have been shown to be trivial then $\tripel{E_k}{R_k}{N_k}$ is complete.
% for each proof \stateEl{P_\tripel{E_0}{R_0}{N _0}}{\eq{s}{t}} there 
% exists a rewrite-proof \stateEl{Q_\tripel{E_k}{R_k}{N_k}}{\eq{s}{t}}. 
Therefore
the completion procedure either tries to show that an equation
in $E$ is trivial or it
transforms the equation into a rewrite rule
or into a nonoperational equation. 

Unfortunately the property of a conditional equations being trivial is not decidable.
CEC applies several advanced techniques for detecting the convergence of
conditional equations cf. \cite{Gan88a}. The CEC-technique of using nonoperational
equations for elimination will discussed together with
the inference rule {\em Deleting a trivial equation} below. \bigskip

\begin{CRule}[(cp), Adding a contextual critical pair]
\deducRule{ E , R \cup \{ \condRule{C}{\rewRule{s}{t}},  
                       \condRule{D}{\rewRule{l}{r}} \} , N}{
 E \cup \set{
\condEq{\applysubst{\sigma}{C} \condAnd \applysubst{\sigma}{D}}{
\eq{\applysubst{\sigma}{\replace{s}{o}{r}}}{\applysubst{\sigma}{t}}}}, 
R \cup \set{ \condRule{C}{\rewRule{s}{t}}, \condRule{D}{\rewRule{l}{r}}} , N} 
\end{CRule}

\noindent
where $o$ is a nonvariable occurrence in $s$ so that \subterm{s}{o} and $l$ 
can be unified with the mgu $\sigma$.\bigskip

\begin{CRule}[(superpose (condition)), Adding a condition superposition instance]
\deducRule{ E , R \cup \set{ \condRule{D}{\rewRule{l}{r}}} , 
          N  \cup \set{ \condEq{C \condAnd \eq{u}{v}}{\eq{s}{t}}} }
{E \cup \set{ 
\condEq{\applysubst{\sigma}{D} \condAnd
\applysubst{\sigma}{C} \condAnd \applysubst{\sigma}{\replace{(\eq{u}{v})}{o}{r}}}{
\eq{\applysubst{\sigma}{s}}{\applysubst{\sigma}{t}}}} , 
 R \cup \set{ \condRule{D}{\rewRule{l}{r}}} , 
 N \cup \set{ \condEq{C \condAnd \eq{u}{v}}{\eq{s}{t}}} }
\end{CRule}

where $o$ is a nonvariable occurrence in \eq{u}{v} such 
that \subterm{(\eq{u}{v})}{o} and $l$ can be unified
with the mgu $\sigma$. The rewrite rule \condRule{D}{\rewRule{l}{r}} is 
used for superposition on the condition equation \eq{u}{v} 
in the nonoperational equation \condEq{C \condAnd \eq{u}{v}}{\eq{s}{t}}. \bigskip

\begin{CRule}[(superpose (reflexivity)), Adding a condition superposition instance]
\deducRule{ E , R , N \cup \set{\condEq{C \condAnd \eq{u}{v}}{\eq{s}{t}}} }
        { E \cup \set{\condEq{\applysubst{\sigma}{C}}{
\eq{\applysubst{\sigma}{s}}{\applysubst{\sigma}{t}}}} , R ,
          N \cup \set{\condEq{C \condAnd \eq{u}{v}}{\eq{s}{t}}} }
\end{CRule}

\noindent
where $u$ and $v$ can be unified with a mgu $\sigma$.\bigskip

\begin{CRule}[(superpose (conclusion)), Adding a superposition instance]
\deducRule{ E , R \cup \set{ \condRule{D}{\rewRule{l}{r}}} , 
          N  \cup \set{ \eq{s}{t}}}
{E \cup \set{ 
\condEq{\applysubst{\sigma}{D}}{
\eq{\applysubst{\sigma}{\replace{s}{o}{r}}}{\applysubst{\sigma}{t}}}} , 
 R \cup \set{ \condRule{D}{\rewRule{l}{r}}} , 
 N \cup \set{\eq{s}{t}}} 
where $o$ is a nonvariable occurrence in $s$ so that $\subterm{s}{o}$
and $l$ can be unified with the mgu $\sigma$.
\end{CRule}\bigskip


\begin{CRule}[(orderEq), Orienting an equation into a reductive rule]
\deducRule{ E \cup 
 \set{\condEq{\eq{u_1}{v_1} \condAnd \eq{u_2}{v_2} \condAnd \ldots
 \condAnd \eq{u_n}{v_n}}{\eq{s}{t}}},
          R , N }{ E , R \cup 
 \set{\condRule{\eq{u_1}{v_1} \condAnd \eq{u_2}{v_2} \condAnd \ldots
 \condAnd \eq{u_n}{v_n}}{\rewRule{s}{t}}} , N }

if $\set{s} \TO> \set{u_1, v_1, u_2, v_2, \ldots, u_n , v_n, t} , n \geq 0 $.

\deducRule{ E \cup \set{\condEq{\eq{u_1}{v_1} \condAnd \eq{u_2}{v_2}
   \condAnd \ldots \condAnd \eq{u_n}{v_n}}{\eq{s}{t}}},
  R, N }{ E , R \cup \set{
\condRule{\eq{u_1}{v_1} \condAnd \eq{u_2}{v_2} \condAnd \ldots
\condAnd \eq{u_n}{v_n}}{\rewRule{t}{s}}} , N }

if $\set{t} \TO> \set{u_1, v_1, u_2, v_2, \ldots, u_n, v_n, s}, n \geq 0$ .
\end{CRule} \bigskip

\begin{CRule}[(orderEq), Orienting an equation into a quasi-reductive rule]
\deducRule{ E \cup 
 \set{\condEq{\eq{u_1}{v_1} \condAnd \eq{u_2}{v_2} \condAnd \ldots
 \condAnd \eq{u_n}{v_n}}{\eq{s}{t}}},
          R , N }{ E , R \cup 
 \set{\condRule{\eq{u_1}{v_1} \condAnd \eq{u_2}{v_2} \condAnd \ldots
 \condAnd \eq{u_n}{v_n}}{\rewRule{s}{t}}} , N }

if $\exists h_1(x),\ldots,h_n(x) \in T_\Sigma(X), x \in X, 
 s \TO> h_1(u_1), h_i(v_i) \geq h_{i+1}(u_{i+1}), 1 \leq i < n,
 h_n(v_n) \geq t$.

\deducRule{ E \cup \set{\condEq{\eq{u_1}{v_1} \condAnd \eq{u_2}{v_2}
   \condAnd \ldots \condAnd \eq{u_n}{v_n}}{\eq{s}{t}}},
  R, N }{ E , R \cup \set{
\condRule{\eq{u_1}{v_1} \condAnd \eq{u_2}{v_2} \condAnd \ldots
\condAnd \eq{u_n}{v_n}}{\rewRule{t}{s}}} , N }

if $\exists h_1(x),\ldots,h_n(x) \in T_\Sigma(X), x \in X,  
 t \TO> h_1(u_1), h_i(v_i) \geq h_{i+1}(u_{i+1}), 1 \leq i < n,
 h_n(v_n) \geq s$.
\end{CRule}
\vspace{1.5ex}

\noindent
The {\em reductivity} condition is essential to avoid an infinite recursive evaluation of the
condition when checking the applicability of a conditional rewrite rule.\bigskip
%As shown in \cite{Gan87b} a much more powerful simplification of
%equations and rules is possible in the presence of orientable condition equations. 
%Therefore the system tries to orient condition 
%equations. However, it is not required that each condition is orientable.\bigskip

\begin{CRule}[(nopEq), Declaring an equation as nonoperational]
\deducRule{ E \cup \set{\condEq{C}{\eq{s}{t}}}, R, N }
{ E, R, N \cup \set{\condEq{C}{\eq{s}{t}}} } 
\end{CRule}

\noindent
For any nonreductive initial equation the system asks if it should be
considered as nonoperational. In addition for every conditional equation generated during
completion by \comRef{cp} or \comRef{superpose} the user is asked to decide as to 
wether \comRef{orderEq} or \comRef{nopEq} should be applied. It may be the case 
that a generated conditional equation can be oriented into a reductive rewrite rule, 
nevertheless is operationally useless, and may in fact cause nontermination 
of the completion procedure.
Since it is sufficient to superpose all rewrite rules on just one condition of any
nonoperational conditional equation, CEC asks the user on which condition superposition should be 
applied. At this point it is often sensible to select a condition which is a 
maximal one w.r.t.\ the given reduction ordering. An exception might be the case
where such conditions contain AC-operators since superposition is then very 
costly. 
For unconditional equations only nonreductive equations should be
declared as nonoperational. In that case all rewrite rules must be superposed
on both sides of the nonoperational unconditional equation. With this feature
CEC provides a kind of (semi-)unfailing completion\footnote[1]{If such an equation cannot be eliminated eventually,
completion fails as CEC does not support rewriting with 
instances of unorientable equations}. \bigskip

\begin{CRule}[(redEq), Deleting a trivial condition of a conditional equation]
\deducRule{ E \cup \set{\condEq{C \condAnd \eq{u}{u}}{\eq{s}{t}}}, R, N}
{ E \cup \set{\condEq{C}{\eq{s}{t}}}, R, N }
\end{CRule}

\begin{CRule}[(redNopEq), Deleting a trivial condition of a nonoperational conditional equation]
\deducRule{ E, R, N \cup \set{\condEq{C \condAnd \eq{u}{u}}{\eq{s}{t}}} }
{ E, R, N \cup \set{\condEq{C}{\eq{s}{t}}} } 
\end{CRule}

\noindent Orientable condition equations in $C$ may 
be used as additional rewrite rules for simplification. Technically, simplification
with oriented condition equations requires the skolemization (replacement of
variables by new constants) of these 
equations. 
$\tilde{s}, \tilde{u}$ denote the skolemized versions of $ s, u $ and
$\tilde{C} $ the skolemized and oriented subset of $C$.\bigskip

\begin{CRule}[(redEq), Simplifying a condition of a conditional equation]
\ifdeducRule{ E \cup \set{\condEq{C \condAnd \eq{u}{v}}{\eq{s}{t}}}, R, N }
{ E \cup \set{\condEq{C \condAnd \eq{w}{v}}{\eq{s}{t}}}, R, N }if{
\tilde{u} \rewqr{R \cup \tilde{C}} \tilde{w}}
\end{CRule}

\noindent
A condition equation may be simplified under the assumption that the remaining condition 
equations hold true. Again, orientable condition equations may be used as additional
rewrite rules.\bigskip

\begin{CRule}[(redNopEq), Simplifying a condition of a nonoperational equation]
\ifdeducRule{ E, R, N \cup \set{\condEq{C \condAnd \eq{u}{v}}{\eq{s}{t}}} }
{ E \cup \set{\condEq{C \condAnd \eq{w}{v}}{\eq{s}{t}}}, R, N }if{
\tilde{u} \rewqr{R \cup \tilde{C} } \tilde{w}}
\end{CRule}

\noindent
Simplified nonoperational equations are turned back into conventional equations. \bigskip

\begin{CRule}[(redRule), Simplifying the condition of a conditional rewrite rule]
\ifdeducRule{ E, R \cup \set{\condRule{C \condAnd \eq{u}{v}}{\rewRule{s}{t}}}, N  }
{ E, R \cup \set{\condRule{C \condAnd \eq{w}{v}}{\rewRule{s}{t}}}, N }if{
\tilde{u} \rewqr{R \cup \tilde{C}} \tilde{w}}
\end{CRule}

\begin{CRule}[(redEq), Simplifying the conclusion of a conditional equation]
\ifdeducRule{ E \cup \set{\condEq{C}{\eq{s}{t}}}, R, N }
{ E \cup \set{\condEq{C}{\eq{u}{t}}}, R, N }if{\tilde{s} \rewqr{R \cup \tilde{C} } \tilde{u}}
\end{CRule}

\begin{CRule}[(redNOpEq), Simplifying the conclusion of a nonoperational conditional equation]
\ifdeducRule{ E, R, N \cup \set{\condEq{C}{\eq{s}{t}}} }
{ E, R, N \cup \set{\condEq{C}{\eq{u}{t}}} }if{
\tilde{s} \rewqr{R \cup \tilde{C}} \tilde{u}}
\end{CRule}


\begin{CRule}[(redRule), Simplifying the right hand side of a rule]
\ifdeducRule{ E, R \cup \set{\condRule{C}{\rewRule{s}{t}}}, N }
{ E, R \cup \set{\condRule{C}{\rewRule{s}{u}}}, N }if{
\tilde{t} \rewqr{R \cup \tilde{C} } \tilde{u}}
\end{CRule}

\begin{CRule}[(redRule), Simplifying the left hand side of a rule]
\ifdeducRule{ E, R \cup \set{\condRule{C}{\rewRule{s}{t}}} , N }
{ E \cup \set{\condEq{C}{\eq{u}{t}}}, R, N }if{
\tilde{s} \rewqr{R \cup \tilde{C} } \tilde{u}}
\end{CRule}

\noindent
Rewrite rules with simplified left hand side must be turned back into equations.
This rule has a further restriction for its application which we do not want to
mention here.\bigskip

\begin{CRule}[(redEq), Deleting a trivial equation]
\ifdeducRule{ E \cup \set{\condEq{C}{\eq{s}{t}}}, R, N }
{ E, R, N }if{\condEq{C}{\eq{s}{t}} \mbox{{\rm \  is\ trivial}}}
\end{CRule}

\noindent
Convergent equations may be eliminated. Particularly simple cases of
($redEq$) are $s \equiv t$ or $ s=t \in C $. Apart from usual
simplification and subsumption methods, CEC also applies nonoperational
equations from $N$ for constructing simpler proofs for equations from $E$.
The principal schema can be sketched as follows. 
The equation $\condEq{C}{\eq{s}{t}} \in E$ is trivial if the following
cases apply:
\begin{enumerate}
\item
$s \equiv t$ or $\eq{s}{t} \in C$.
\item
\condEq{C}{\eq{s}{t}} is subsumed by another equation from $E$ or $N$
\item
\condEq{C}{\eq{s}{t}} can be simplified to a trivial equation.
\item
$C$ contains an equation \eq{u}{v} that is not satisfiable for consistent
specifications. 
An equation \eq{u}{v} is unsatisfiable if $u$ and $v$ are not identical and
both consist of constructor operators only\footnote[1]{Note that CEC does 
not allow rules with constructors at the top of the left sides. Any such
situation leeds to the failure of the completion}.
\item
if \condEq{C \condAnd (\eq{\applysubst{\sigma}{m}}{\applysubst{\sigma}{n}})}{\eq{s}{t}}
is trivial, where $\condEq{D}{\eq{m}{n}} \in N$, such that $C$ 
implies \applysubst{\sigma}{D} 
and additional constraints
concerning the \PO - relation on the involved proofs are fulfilled
cf. \cite{Gan87b}.
We say that \condEq{C}{\eq{s}{t}} is {\it forward chained} using the nonoperational 
equation \condEq{D}{\eq{m}{n}}. CEC reports every forward chaining of equations and
the result of the corresponding comparisons on proofs.
\item
Similar to 5.\ with backward chaining (resolution) instead of forward chains.
\end{enumerate}

\subsection{Incremental Termination Orderings}
\label{TerminationOrderings}

As described above the definition of a appropriate {\em reduction ordering} is 
fundamental for the completion procedure.
An important improvement of completion procedures today over the original one
by Knuth and Bendix 
\cite{KB69} is the use of {\it incremental termination orderings}. 
In the original procedure the reduction ordering had to be given
{\it a priori}. Incremental orderings are orderings where the partial order on terms
is induced from another partial order on operators or on interpretations of operators
occurring in terms. If two incomparable terms $t_1$ and $t_2$ shall be ordered,
one looks for a consistent extension of the current ordering so that
$t_1  > t_2 $ or $t_2  > t_1 $. Since there are
usually several different possible extensions of the underlying ordering which all
induce the desired extension of the term ordering, implementations of these orderings
ask the user to select one.

Together with the possibility to write an order specification, the user is now
able to supply some information about the termination ordering right at the
beginning, to extend this information interactively during the completion process
and to save the final termination ordering in an order specification again.
So he avoids to answer the same questions concerning the termination ordering 
when he tries to complete the specification again.

The default termination ordering used by CEC is \kw{neqkns}, except if the specification
contains associative and commutative operators. In this case \kw{poly1} is the
default ordering since termination proofs with \kw{kns} or \kw{neqkns} 
are invalid in the presence of AC-operators. 

The user can change the termination ordering using the 
\comRef{oder}-command (\refArrow\ chapter \ref{InspecTermination}).

%\begin{command}[\comName{order}]
%indicates the current termination ordering and asks the user whether he wants to 
%change it.
%When invoked it displays the following: 
%
%\begin{screen}
%The current termination ordering is "neqkns".
%The following alternative orderings may be selected:
%
%recursive path ordering without equality (after Kapur, Narendran,
%Sivakumar)
%
%recursive path ordering (after Kapur, Narendran, Sivakumar)
%
%polynomial abstraction with tuplelength N
%
%manual ordering
%
%"file." for reading from a file
%
%"no." for no change
%   Please answer with neqkns. or kns. or poly<N>. or manual. or
%file. or no. (Type A. to abort) >
%\end{screen}
%
%If a new ordering is selected and if the previous termination
%ordering is incompatible with the new ordering, all rules are turned back into
%equations and the completion must be repeated from the beginning.
%\end{command}

\subsubsection{KNS and NEQKNS: Recursive Path Orderings}
\label{kns}

\kw{kns} and \kw{neqkns} are {\it precedence orderings}. That means the partial order on terms is
induced by a partial order on operators called the {\it precedence}. 
The precedence ordering \kw{neqkns} forbids that two different operators have the same
precedence. Initially the
precedence is empty. Order Specifications may include precedence declarations, cf. 
chapter~\ref{OrderPragmas}. 
All constructors are given a precedence less than any nonconstructor operator.
The precedence ordering is extended during 
the completion process or using the \comRef{equal} and \comRef{greater}
commands. The current state of the precedence definitions can be
displayed using the \comRef{operators} command.

\subsubsection{POLY$<$N$>$: Polynomial Orderings.}
\label{poly}

The second group of termination orderings available in CEC is \kw{poly}\nt{N}. Here the idea is to
give {\it polynomial interpretations} to terms in a way that
\[t_1 > t_2 : \iff I(t_1) > I(t_2). \]
An m-tuple of integer polynomials $F_i(x_1, \ldots, x_n)$ is associated with
each $n$-ary operator $f$. The choice of coefficents must ensure {\it monotonicity},
e.g.
\[
I(t_1 ) > I(t_2 )\ \  {\rm implies}\ \  I(f(\ldots, t_1, \ldots)) >
I(f(\ldots, t_2, \ldots)) \]
and that terms are mapped into nonnegative integers only; this is the case if all
coefficients are positive.

The concrete version of this technique as it is implemented CEC is due to \cite{CL86}. Again
we only need to order the equations to infer the termination property for all
reductions on terms. 
Changing to a smaller tuple length {\rm N} can
make old termination proofs invalid. In that case CEC turns back all rules
into equations. The default termination ordering in the presence of AC-operators
is \kw{poly1}.

The difficulty of these termination orderings from a users point of view is how to
guess the appropriate polynomials such that all equations will be ordered in the
desired way. It will take some time to become familiar with this technique.\bigskip

Order Specifications may include polynomial interpretations for operators, cf. 
chapter~\ref{OrderPragmas}. Interpretations are added during the 
completion process or using the \comRef{setInterpretation} command.
The current polynomial interpretations can be displayed using the
\comRef{operators} command or the \comRef{interpretation}-command
(\refArrow\ chapter \ref{InspecTermination}).

There are two operators which allow to compute the interpretation of a term
and to compare two interpretations:\bigskip

\begin{command}[\com{polGreater}{\comArg{Interpretation1}\ad\comArg{Interpretation2}}]
Only useful with ordering \kw{poly}\nt{N}.
If it succeeds, \comArg{Interpretation1} $>$ \comArg{Interpretation2} holds true
(if $>$ is the ordering on tuples of polynomials).
Interpretations (i.e. tupels of polynomials) of terms can be generated
via \comRef{polynomial}.
\end{command}

\begin{command}[\com{polynomial}{\comArg{Term}\ad\comArg{Interpretation}}]
yields the polynomial interpretation of \comArg{Term}. It
fails, if the ordering is not \kw{poly}\nt{N}. 
If there are operators in \comArg{Term},
for which no polynomial interpretation is known, the user is asked for
such an interpretation (and the given interpretation is stored).
\end{command}

\subsection{Specifications with Associative and Commutative Operators}

As a commutativity axiom immediately destroys the termination property of a term
rewriting system, this property cannot be expressed by a term rewrite rule. A well
known solution to this problem is unification and rewriting modulo associativity and
commutativity, cf. 
\cite{Sti81} and \cite{Fag83}. CEC must know the AC-operators of the
signature. CEC extracts this information from the given set of axioms and treats
these equations differently, i.e. they will not be oriented into rewrite rules.
The AC-equations will be added to the set of nonoperational equations, but in contrast
to other nonoperational equations they are not superposed with rewrite rules to compute 
coherence pairs. In CEC the extended rule technique is implemented instead,
cf.\ \cite{PS81} and \cite{JK86b}.

Note that in the presence of AC-operators termination proofs using path orderings
like \kw{kns} are invalid. But termination proofs with polynomial orderings may
be possible. Therefore CEC sets the default termination ordering
to \kw{poly1} if there are AC-operators in a specification. 


\subsection{Running the Completion Procedure}

\begin{command}[\comName{c}]
calls the Knuth-Bendix completion procedure. This executes a fixed strategy of
applications of the ``completion inference'' predicates \comRef{orderEq}, 
\comRef{cp}, \comRef{superpose}, \comRef{redRule}, \comRef{redEq} and
\comRef{redNopEq}.
\end{command}

\begin{command}[\comName{cResume}]
restarts the completion procedure after the completion process was
aborted by answering ``\cec{A.}'' to some query of the system.
\end{command}

\noindent
A manual guidance of the process is possible by explicitly calling
completion inference rules. \bigskip

\begin{command}[\com{orderEq}{\comArg{EquationIndex}}]
orients equation with index \comArg{EquationIndex}. The predicate fails if equation 
\comArg{EquationIndex} does
not exist or if the equation cannot be oriented or turned into a nonoperational
equation or if the equation is eliminated during reduction.
\end{command}

\begin{command}[\com{nopEq}{\comArg{EquationIndex}}]
declares equation with index \comArg{EquationIndex} as nonoperational. 
The predicate fails if equation \comArg{EquationIndex} does not exist 
or if the equation is trivial.
\end{command}

\begin{command}[\com{cp}{\comArg{RuleIndex1}\ad\comArg{RuleIndex2}}]
computes all critical pairs of rule \comArg{RuleIndex1} on rule 
\comArg{RuleIndex2}. 
The predicate fails, if no nontrivial critical pair can be found.
\end{command}

\begin{command}[\com{superpose}{\comArg{RuleIndex}\ad\comArg{NopEqIndex}\ad\comArg{Literal}\ad\comArg{LiteralSide}}]
superposes the left-side of the rule \comArg{RuleIndex} on the 
\comArg{LiteralSide}-side of the
literal \comArg{Literal} of the nonoperational equation with index 
\comArg{NopEqIndex}.
It fails if no nontrivial superpositions can be found, if
any of the two axioms can be reduced, or if
superpositions of the specified type need not be computed to achieve 
fairness.\\
To denote the \comArg{LiteralSide} \kw{left} and \kw{right} are used. 
\comArg{NopEqIndex} must be the index of a nonoperational equation.
Considering superposition with
\condEq{L_1\condAnd\ldots\condAnd L_n}{L}, 
we use \kw{conclusion} to denote $L$, and \com{condition}{\comArg{$i$}}
to denote $L_i$ in \comArg{Literal}.
\end{command}

\begin{command}[\com{superpose}{\kw{reflexivity}\ad\comArg{NopEqIndex}\ad
\comArg{Literal}\ad\_}]
superposes \rewRule{x = x}{true} on the literal \comArg{Literal} of the 
nonoperational equation with index \comArg{NopEqIndex}.
It fails if no nontrivial superpositions can be found, if
any of the two axioms can be reduced, or if
superpositions of the specified type need not be computed to achieve 
fairness.
\end{command}

\begin{command}[\com{redRule}{\comArg{RuleIndex}}]
reduces rule with index \comArg{RuleIndex}. The predicate fails if rule 
\comArg{RuleIndex} does not
exist or if the rule cannot be reduced or if the rule is eliminated during 
reduction.
\end{command}

\begin{command}[\com{redEq}{\comArg{EquationIndex}}]
reduces equation with index \comArg{EquationIndex}. 
The predicate fails if equation \comArg{EquationIndex} does
not exist or if the equation cannot be reduced or if the equation is eliminated
during reduction.
\end{command}
 
\begin{command}[\com{redNopEq}{\comArg{NopEqIndex}}]
reduces nonoperational equation with index \comArg{NopEqIndex}. 
The predicate fails if equation
\comArg{NopEqIndex} does not exist or if the equation cannot be reduced or if the 
equation is eliminated during reduction.
\end{command}

\noindent
The \comRef{repeat} predicate can be used to execute a predicate repeatedly 
until no more instances of it can be applied.\bigskip

\begin{command}[\com{repeat}{\comArg{Predicate}}]
causes repeated backtracking of \comArg{Predicate} until \comArg{Predicate} fails.
\end{command}

\noindent
An arbitrary interleaving of manual and automatic completion is supported. Also
completion --- manual or automatic --- can always safely be restarted after any
abortion caused by answering ``\cec{A.}'' to some query of the system.





  % 16.3.89
\section{Computing with Completed Specifications}

One may regard the completed specification as a first step towards an operational
implementation. 
Terms can be efficiently normalized, identities efficiently proved or disproved.
Moreover the solutions of equations can be found through narrowing.

\subsection{Normalization of Terms}
\label{NormCommand}

Computation in specifications is realized by term reduction with the
rules of the completed specification. The result of such computations are unique
normal forms. \bigskip

\begin{command}[\com{norm}{\comArg{Expression}}]
normalizes the input expression \comArg{Expression}. Three kinds of normalization are
provided:
\begin{itemize}
\item
if \comArg{Expression} is a signature term, the current rules are used to compute the
normalform of \comArg{Expression}.
\item
if \comArg{Expression} has the form \condRule{\comArg{conjunction}}{\comArg{term}} 
where \comArg{conjunction} is
a conjunction of equations, CEC will try to orient these equations into
rules before normalizing \comArg{Term} with the current rules and these oriented
condition equations.
\item
if \comArg{Expression} is a {\em let-expression}, the definitions (which
are equations between constructor terms and let-expressions) 
are evaluated first,
binding variables to the evaluated terms, before normalizing the body
of the let-expression using the current set of rules. This feature
should be used in the order-sorted case to avoid typing problems.
\end{itemize}
Let-expressions are given according to the following syntax:

\begin{syntax}
\nt{let-expression} \IS \kw{let} \nt{definitions} \kw{in} \nt{let-expression}\END
\\
\nt{definitions}    \IS \nt{definition} \{ \kw{and} \nt{definition} \} \END
\nt{definition}     \IS \nt{pattern} \kw{=} \nt{let-expression}
		    \OR \nt{pattern} \kw{=} \nt{signatureTerm} \END
\\
\nt{pattern}        \IS $<$ \rm well typed term with variables build up from
                    \GETON \hspace{2ex} constructors of the user-specified signature$>$
\end{syntax}

\noindent 
It is possible to trace the application of all rules or some selected rules:\medskip

\kw{trace:=on}

\noindent enables the trace mechanism.\medskip

\kw{rulesToTrace:=[}\comArg{RuleIndex1}\ad\ldots\ad\comArg{RuleIndexN}\kw{]}

\noindent restricts the trace mechanism to applications of the
rules \comArg{RuleIndex1},\ldots,\comArg{RuleIndexN}.
\end{command}

\begin{command}[\com{applyRule}{\comArg{Term}\ad\comArg{RuleIndex}\ad\comArg{ReducedTerm}}]
attempts to apply the rule with the given index once to the given term.
Different redexes are tried upon backtracking. If successful, the reduced
term is computed.
\end{command}

%\noindent
%Given the following confluent and Noetherian specification: \bigskip
%
%\begin{screen}
%| ?- {\bf show.}
%
%Current equations
%
%
%
%Current rules:
%
%  1  append([],l) -> l
%  2  append([e|l1], l2) -> [e|append(l1,l2)]
%  3  rev([]) -> []
%  4  rev([e|l]) -> append(rev(l),[e])
%
%Current nonoperational equations
%
%
%All axioms reduced.
%All critical pairs computed.
%All superpositions on nonoperational equations computed.
%No more equations, the system is complete.
%yes
%| ?-
%\end{screen}
%
%\noindent
%Then we can compute the reverse of a given list by computing it's normal form.
%
%\begin{screen}
%| ?- {\bf norm(rev([a, b, c, d]), NormalForm).}
%
%NormalForm = [ d, c, b, a]
%
%| ?-
%\end{screen}
%
%\noindent
%With applyRule we can apply one specific rule to reduce a term:
%
%\begin{screen}
%| ?- {\bf applyRule(rev([a, b, c, d]), 4, ReducedForm).}
%
%ReducedForm = append(rev([b, c, d]), [a])
%
%| ?- {\bf applyRule(rev([a, b, c, d]), 3, ReducedForm).}
%
%no
%
%| ?-
%\end{screen}

\noindent
To improve the computation of the normal form of a term, one can {\it compile}
the current set of rewrite rules into compiled Prolog and
use this compiled set of rewrite rules to normalize given terms.\bigskip

\begin{command}[\comName{compile}]
compiles the current set of rewrite rules into \comRef{compiled} Prolog.
The compiled rules are used when calling \comRef{eval}. 
Later changes to the set of rewrite rules
have no effect on \comRef{eval} unless a new call to \comRef{compile} is
performed. The predicate \comRef{norm} always uses the current set of
rewrite rules.
\end{command}

\begin{command}[\com{eval}{\comArg{Expression}}]
computes the normalform of a expression using the most recently compiled set of rewrite
rules. \comRef{eval} fails, if \comRef{compile} has not been called yet, cf. \comRef{compile}.
The trace mechanism applies also to \comRef{eval}.
\end{command}

\subsection{Proving Theorems in the Equational Theory}
\label{ProveCommand}

If general confluence can be achieved equational theorems become
decidable, e.g. it is decidable if two terms are equivalent with respect to the
equations in the specification: simply reduce the two terms to their unique normal
form and check if they are identical. This can be done using the
\comRef{prove}-command:\bigskip

\begin{command}[\com{prove}{\comArg{ConditionalEquation}}]
proves or disproves the conditional equation \comArg{ConditionalEquation} by 
rewriting the conclusion to normalforms, using the equations in the 
condition as additional rewrite rules. The method is incomplete for
nonempty conditions and/or noncanonical systems.
\end{command}

%In the following example a theorem of the equational theory is proved.
%
%\begin{screen}
%| ?- {\bf prove((append(rev([x, y]), [c, d]) = [y, x, c, d])).}
%
%Normal forms are : [y,x,c,d] and [y,x,c,d]
%yes
%| ?-
%\end{screen}


\subsection{Solving Equations in the Equational Theory}
\label{NarrowCommand}

Narrowing can be used to solve equations in an equational theory if a
canonical set of rewrite rules equivalent to the set of equations  exists.\bigskip

\begin{command}[\com{solve}{\comArg{Goal}\ad\comArg{Solution}}]
tries to solve \comArg{Goal} and to return an answer-substitution 
if successful.
The set of all answer-substitutions can be obtained by backtracking
(enter '\kw{;}'\nt{return} at the user level). Enter \nt{return} if
no more solutions are wanted.

\noindent
Goals are given using the following syntax:

\begin{syntax}
\nt{Goal} \IS \nt{condition} \nt{equation} 
          \OR \nt{equation} \{ \kw{and} \nt{equation} \} \END
\\
\nt{condition} \IS \nt{equation} \{ \kw{and} \nt{equation} \} \kw{=>} \END
\nt{equation}  \IS \nt{signatureTerm} \kw{=} \nt{signatureTerm} \END
\end{syntax}
\end{command}
   % 2.5.89


\section{{\tt ANSWER\_\-ELLIPSIS\_\-OFF}}
\label{Section:ANSWER--ELLIPSIS--OFF}

{\em [Switch off answer ellipsis (default).]}

The converse of {\tt ANSWER\_ELLIPSIS\_ON}. Relevant to bidirectional translation applications.




See Section~\ref{Section:AnswerEllipsis}.

\section{{\tt ANSWER\_\-ELLIPSIS\_\-ON}}
\label{Section:ANSWER--ELLIPSIS--ON}

{\em [Switch on answer ellipsis.]}

In bidirectional translation applications, it is possible to use ``answer ellipsis'': one user asks a question, and non-sentential responses are treated as ellipsis. For example, in an English-to-French translator, the question ``Where is the pain?'' might be translated as ``O� avez-vous mal?'' Then, if answer ellipsis is switched on and there are suitable ellipsis declarations loaded, ``le soir'' might be interpreted as ``J'ai mal le soir''. 



See Section~\ref{Section:AnswerEllipsis}.

\section{{\tt BATCH\_\-DIALOGUE Arg1}}
\label{Section:BATCH--DIALOGUEArg1}

{\em [Process dialogue corpus with specified ID.]}

Parameterised version of BATCH\_\-DIALOGUE. Process the default dialogue
mode development corpus, defined by the dialogue\_\-corpus($\langle$Arg$\rangle$) config
file entry. The output file, defined by the
dialogue\_\-corpus\_\-results($\langle$Arg$\rangle$) config file entry, contains question
marks for dialogue processing steps that have not yet been judged. If
these are replaced by valid judgements, currently 'good', or 'bad',
the new judgements can be incorporated into the dialogue judgements
file (defined by the dialogue\_\-corpus\_\-judgements config file entry)
using the command UPDATE\_\-DIALOGUE\_\-JUDGEMENTS $\langle$Arg$\rangle$.



See Section~\ref{Section:RegressionTestingDialogue}.

\section{{\tt BATCH\_\-DIALOGUE}}
\label{Section:BATCH--DIALOGUE}

{\em [Process dialogue corpus.]}

Process the default dialogue mode development corpus, defined by the
dialogue\_\-corpus config file entry. The output file, defined by the
dialogue\_\-corpus\_\-results config file entry, contains question marks for
dialogue processing steps that have not yet been judged. If these are
replaced by valid judgements, currently 'good', or 'bad', the new
judgements can be incorporated into the dialogue judgements file
(defined by the dialogue\_\-corpus\_\-judgements config file entry) using
the command UPDATE\_\-DIALOGUE\_\-JUDGEMENTS.



See Section~\ref{Section:RegressionTestingDialogue}.

\section{{\tt BATCH\_\-DIALOGUE\_\-SPEECH Arg1}}
\label{Section:BATCH--DIALOGUE--SPEECHArg1}

{\em [Process dialogue speech corpus with specified ID.]}

Parameterised speech mode version of BATCH\_\-DIALOGUE. Process the
default dialogue mode speech corpus, defined by the
dialogue\_\-corpus($\langle$Arg$\rangle$) config file entry. The output file, defined by
the dialogue\_\-speech\_\-corpus\_\-results($\langle$Arg$\rangle$) config file entry, contains
question marks for dialogue processing steps that have not yet been
judged. If these are replaced by valid judgements, currently 'good',
or 'bad', the new judgements can be incorporated into the dialogue
judgements file (defined by the dialogue\_\-corpus\_\-judgements config file
entry) using the command UPDATE\_\-DIALOGUE\_\-JUDGEMENTS\_\-SPEECH $\langle$Arg$\rangle$.



See Section~\ref{Section:RegressionTestingDialogue}.

\section{{\tt BATCH\_\-DIALOGUE\_\-SPEECH}}
\label{Section:BATCH--DIALOGUE--SPEECH}

{\em [Process dialogue speech corpus.]}

Speech mode version of BATCH\_\-DIALOGUE. Process the default dialogue
mode speech corpus, defined by the dialogue\_\-speech\_\-corpus config file
entry. The output file, defined by the dialogue\_\-speechcorpus\_\-results
config file entry, contains question marks for dialogue processing
steps that have not yet been judged. If these are replaced by valid
judgements, currently 'good', or 'bad', the new judgements can be
incorporated into the dialogue judgements file (defined by the
dialogue\_\-corpus\_\-judgements config file entry) using the command
UPDATE\_\-DIALOGUE\_\-JUDGEMENTS\_\-SPEECH.



See Section~\ref{Section:RegressionTestingDialogue}.

\section{{\tt BATCH\_\-DIALOGUE\_\-SPEECH\_\-AGAIN Arg1}}
\label{Section:BATCH--DIALOGUE--SPEECH--AGAINArg1}

{\em [Process dialogue speech corpus with specified ID, using recognition results from previous run.]}

Like {\tt BATCH\_DIALOGUE\_SPEECH\_AGAIN}, but uses the version of the speech corpus tagged {\tt Arg1}. Input is taken from the transcriptions file specified by the config parameter
\begin{verbatim}
dialogue_speech_corpus(Arg1)
\end{verbatim}
and output is written to the file specified by the config parameter
\begin{verbatim}
dialogue_speech_corpus_results(Arg1)
\end{verbatim}

The config file also needs to define all the entries associated with spoken dialogue applications.



See Section~\ref{Section:RegressionTestingDialogue}.

\section{{\tt BATCH\_\-DIALOGUE\_\-SPEECH\_\-AGAIN}}
\label{Section:BATCH--DIALOGUE--SPEECH--AGAIN}

{\em [Process dialogue speech corpus, using recognition results from previous run.]}

Version of BATCH\_\-DIALOGUE\_\-SPEECH that skips the speech recognition
stage, and instead uses stored results from the previous run.



See Section~\ref{Section:RegressionTestingDialogue}.

\section{{\tt BATCH\_\-HELP Arg1}}
\label{Section:BATCH--HELPArg1}

{\em [Process help corpus with specified ID.]}

Like {\tt BATCH\_HELP}, but input is taken from the file defined by
\begin{verbatim}
help_corpus(Arg1)
\end{verbatim}
and output is written to the file defined by
\begin{verbatim}
help_corpus_results(Arg1)
\end{verbatim}



See Section~\ref{Section:HelpSystemCommandLine}.

\section{{\tt BATCH\_\-HELP}}
\label{Section:BATCH--HELP}

{\em [Process help corpus.]}

Relevant to applications using targeted help. The corpus defined by the config file parameter
\begin{verbatim}
help_corpus
\end{verbatim}
which should be in \verb!sent(...)! form, is passed through help processing, and the results
are written out to the file defined by
config file parameter
\begin{verbatim}
help_corpus_results
\end{verbatim}
Help resources must be defined and loaded.



See Section~\ref{Section:HelpSystemCommandLine}.

\section{{\tt BIDIRECTIONAL\_\-OFF}}
\label{Section:BIDIRECTIONAL--OFF}

{\em [Switch off bidirectional mode (default).]}

Relevant to bidirectional translation applications: switches off the mode switched on by {\tt BIDIRECTIONAL\_ON}.



See Section~\ref{Section:Bidirectional}.

\section{{\tt BIDIRECTIONAL\_\-ON}}
\label{Section:BIDIRECTIONAL--ON}

{\em [Switch on bidirectional\_\-mode.]}

Relevant to bidirectional translation applications: switches on a mode where input can be passed to either side of the bidirectional system. Commands for the ``question'' side are prefaced with ``Q:'', e.g.
\begin{verbatim}
Q: LOAD_TRANSLATE

Q: where is the pain
\end{verbatim}
while commands for the `answer'' side are prefaced with ``A:'', e.g.
\begin{verbatim}
A: EBL_LOAD

Q: en la cabeza
\end{verbatim}



See Section~\ref{Section:Bidirectional}.

\section{{\tt CAT Arg1}}
\label{Section:CATArg1}

{\em [Display information for specified category.]}

When doing grammar development, this command lets you display the list of features associated with a given syntactic category. It requires a grammar to be loaded. Here is an example with the English grammar:
\begin{verbatim}
>> CAT np
(Display information for specified category)

Features for category "np": [agr,case,conj,def,gapsin,gapsout,
nform,pronoun,sem,sem_n_type,takes_attrib_pp,takes_frequency_pp,
takes_loc_pp,takes_partitive,takes_post_mods,takes_to_pp,
takes_with_pp,wh]
\end{verbatim}



See Section~\ref{Section:GrammarDebuggingCommands}.

\section{{\tt CHECK\_\-ALTERF\_\-PATTERNS}}
\label{Section:CHECK--ALTERF--PATTERNS}

{\em [Check the consistency of the current Alterf patterns file.]}

Check the consistency of the current Alterf patterns file, defined by
the {\tt alterf\_\-patterns\_\-file config} file entry. Records in the
file should have the format
\begin{verbatim}
alterf_pattern(<Pattern>, <Atom>, <Sent>).
\end{verbatim}
or
\begin{verbatim}
alterf_pattern(<Pattern>, <Atom>, <Sent>) :- <Conds>.
\end{verbatim}
where \verb!<Pattern>! is the Alterf pattern, \verb!<Atom>! is the semantic
atom it corresponds to, \verb!<Sent>! is an example sentence illustrating
the patterns, and \verb!<Conds>! are optional Prolog conditions.

The command parses each \verb!<Sent>! using the currently loaded
grammar, and checks that the \verb!<Pattern>! matches it. It warns
about patterns that fail to match, and print summary statistics.



See Section~\ref{Section:LFPatterns}.

\section{{\tt CHECK\_\-BACKTRANSLATION Arg1}}
\label{Section:CHECK--BACKTRANSLATIONArg1}

{\em [Process Lang -$\rangle$ Lang output in Lang -$\rangle$ Int environment to check that back-translations parse.]}

Command relevant to interlingua-based translation applications which
use backtranslation (cf. Section~\ref{Section:Backtranslation}). The
assumption is that translation is from Source to Interlingua, and
backtranslation is thus from Interlingua to Source.

The argument to the command should be an output file produced by doing 
{\tt TRANSLATE\_\-CORPUS} (cf. Section~\ref{Section:TRANSLATE--CORPUS})
in the Source $\rightarrow$ Source environment. The command is however run
in the Source $\rightarrow$ Interlingua environment. The intent is to
check that the result of backtranslation is an expression which, when
parsed in the Source language and translated back into Interlingua,
would produce the same Interlingua representation as the one produced by performing
the Source $\rightarrow$ Source translation. Examples which fail to
give a match are flagged.



See Section~\ref{Section:TranslationRegressionText}.

\section{{\tt CHECK\_\-PARAPHRASES}}
\label{Section:CHECK--PARAPHRASES}

{\em [Check that transcription paraphrases are in coverage.]}

Command used for regression testing in speech translation
applications, when employing a paraphrase file
(cf. Section~\ref{Section:TranslationRegressionParaphrases}). 
There will typically be some out-of-coverage utterances which are
still close enough that they will successfully go through speech
understanding. However, since the transcription is out-of-coverage,
the scoring routines will have no way to know that the result
is incorrect, since the transcription produces no reference 
output to compare with.

Under these circumstances, it is possible to declare a {\em paraphrase
file}, which associates in-coverage sentences with out-of-coverage
transcriptions. A paraphrase file record has the following format:
\begin{verbatim}
paraphrase(<TranscriptAtom>, <ParaphraseAtom>)
\end{verbatim}
where \verb!<TranscriptAtom>! is the transcription and \verb!<ParaphraseAtom>!
is the paraphrase.

The command {\tt CHECK\_\-PARAPHRASES} assumes that a grammar is loaded.
It parses the \verb!<ParaphraseAtom>! fields in the paraphrase file, and
checks that they are all in coverage, issuing warnings for the ones that are not.




See Section~\ref{Section:TranslationRegressionParaphrases}.

\section{{\tt CLOSE\_\-DOWN\_\-RECOGNITION}}
\label{Section:CLOSE--DOWN--RECOGNITION}

{\em [Close down recognition resources: license manager, recserver, TTS and regserver.]}

If you are doing recognition from the Regulus command-line (cf. Section~\ref{Section:CommandLineSpeechInput}), you may want to use this command to close down the relevant processes after you are finished. This command lets you do it, though it currently only works under Windows/Cygwin.





See Section~\ref{Section:SpeechInput}.

\section{{\tt COMPACTION}}
\label{Section:COMPACTION}

{\em [Switch on compaction processing for Regulus to Nuance conversion (default).]}

Thie command is mostly included for historical reasons. You should not want to switch off compaction under normal circumstances.



See Section~\ref{Section:UGToGSL}.

\section{{\tt COMPILE\_\-ELLIPSIS\_\-PATTERNS}}
\label{Section:COMPILE--ELLIPSIS--PATTERNS}

{\em [Compile patterns used for ellipsis processing.]}


Compile the patterns used for ellipsis processing, which are defined by the ellipsis\_\-classes config file entry. The compiled patterns will be loaded next time you invoke LOAD\_\-TRANSLATE. 



See Section~\ref{Section:TranslationEllipsis}.

\section{{\tt COMPILE\_\-HELP}}
\label{Section:COMPILE--HELP}

{\em [Compile material for targeted help.]}

This command compiles run-time targeted help resources for a specific
language pair in a speech translation application. It expects the
following config file parameters to be defined:

\begin{itemize}

\item {\tt targeted\_\-help\_\-source\_\-files} A list of the form
\begin{verbatim}
[use_combined_interlingua_corpus(<SourceLang>, <TargetLang>),
 <CombinedInterlinguaCorpus>]
\end{verbatim}

where {\tt $langle$CombinedInterlinguaCorpus$\rangle$} is a combined interlingua
corpus (cf.~\ref{Section:InterlinguaCentredDevelopment}) and
{\tt $langle$SourceLang$\rangle$}, {\tt $langle$TargetLang$\rangle$} are identifiers for the
source and target languages. The combined interlingua corpus must be
created first, and contain the relevant languages.

\item {\tt targeted\_\-help\_\-classes\_\-file} A file of targeted
help class declarations for the source language.

\end{itemize}



See Section~\ref{Section:HelpSystemConstruction}.

\section{{\tt DCG}}
\label{Section:DCG}

{\em [Use DCG parser.]}


The grammar can be parsed using either the left-corner parser (the default) or the DCG parser. The left-corner parser is faster, but the DCG parser can be useful for debugging. In particular, it can be used to parse non-top constituents ; the left-corner parser lacks this capability.



See Section~\ref{Section:DCGParser}.

\section{{\tt DIALOGUE}}
\label{Section:DIALOGUE}

{\em [Do dialogue-style processing on input sentences.]}


In this mode, the sentence is parsed using the current parser. If any parses are found, the first one is processed through the code defined by the dialogue\_\-files config file entry.



See Section~\ref{Section:CommandLineOverview}.

\section{{\tt DIALOGUE\_\-SPEAKER}}
\label{Section:DIALOGUE--SPEAKER}

{\em [Show notional speaker (if any) currently being used for dialogue processing.]}

In dialogue processing applications where words like ``I'' and ``me''
are used, it can be important to know the identity of the speaker; for
example, in the Calendar application, one could say ``When is my next
meeting?'' or ``Am I attending a meeting on Friday?''  The {\tt
DIALOGUE\_\-SPEAKER} command makes it possible to print the identity
of the notional speaker from the command-line.  The notional speaker
can also be retrieved using the predicate {\tt
get\_\-notional\_\-speaker/1} in {\tt PrologLib/utilities.pl}.



See Section~\ref{Section:SettingDialogueContext}.

\section{{\tt DIALOGUE\_\-TIME}}
\label{Section:DIALOGUE--TIME}

{\em [Show time (notional or real) currently being used for dialogue processing.]}

In dialogue processing applications where expressions like ``today''
or ``next week'' are used, it is necessary to know the current
time. When doing regression testing, it is then useful to be able to
set a notional time, so that responses to time-dependent utterances
stay stable.

The {\tt DIALOGUE\_\-TIME} command makes it possible to print the
notional time from the command-line. The notional time can also be
retrieved using the predicate {\tt get\_\-notional\_\-time/1} in {\tt
PrologLib/utilities.pl}.



See Section~\ref{Section:SettingDialogueContext}.

\section{{\tt DOC Arg1}}
\label{Section:DOCArg1}

{\em [Print documentation for command or config file entry.]}

The {\tt DOC} command prints out detailed documentation for a command
or config file entry, when this is available. For example
\begin{verbatim}
DOC DIALOGUE_TIME
\end{verbatim}



See Section~\ref{Section:CommandLineDoc}.

\section{{\tt DUMP\_\-NBEST\_\-TRAINING\_\-DATA\_\-OFF}}
\label{Section:DUMP--NBEST--TRAINING--DATA--OFF}

{\em [Don't write out training data when doing batch processing of N-best results (default).]}


When doing batch speech translation, using {\tt
TRANSLATE\_\-SPEECH\_\-CORPUS} and allied commands, it is optionally
possible to write out N-best training data if the config file
parameter {\tt nbest\_\-training\_\-data\_\-file} is defined. Output
is written to the file in question. This command turns off the
behaviour.




See Section~\ref{Section:DialogueNBest}.

\section{{\tt DUMP\_\-NBEST\_\-TRAINING\_\-DATA\_\-ON}}
\label{Section:DUMP--NBEST--TRAINING--DATA--ON}

{\em [Write out training data when doing batch processing of N-best results.]}

When doing batch speech translation, using {\tt
TRANSLATE\_\-SPEECH\_\-CORPUS} and allied commands, it is optionally
possible to write out N-best training data if the config file
parameter {\tt nbest\_\-training\_\-data\_\-file} is defined. Output
is written to the file in question. This command turns on the
behaviour.



See Section~\ref{Section:DialogueNBest}.

\section{{\tt EBL}}
\label{Section:EBL}

{\em [Do main EBL processing: equivalent to LOAD, EBL\_\-TREEBANK, EBL\_\-TRAIN, EBL\_\-POSTPROCESS, EBL\_\-NUANCE.]}


This command does all the processing needed to build a specialised
Nuance grammar from scratch. You will need to have defined at least 
the following config file parameters:
\begin{itemize}

\item {\tt ebl\_\-corpus} The training corpus, which should consist of Prolog records of the form {\tt sent(...)}.

\item {\tt ebl\_\-operationality} The file defining the operationality criteria.

\item {\tt ebl\_\-nuance\_\-grammar} The output file which will contains the final Nuance grammar.

\end{itemize}

In many applications, you will also want the following:

\begin{itemize}

\item {\tt ebl\_\-include\_\-lex} A file of ``include-lex'' definitions, which specify lexical entries to be included directly.

\item {\tt ebl\_\-regulus\_\-component\_\-grammar} The specialised grammar that will be loaded by the {\tt EBL\_LOAD} command, if it is not the default one.

\item {\tt ebl\_\-ignore\_\-feats} Features to ignore when performing Regulus-to-Nuance compilation on the specialised grammar.

\end{itemize}




\section{{\tt EBL\_\-ANALYSIS}}
\label{Section:EBL--ANALYSIS}

{\em [Do main EBL processing, except for creation of Nuance grammar: equivalent to LOAD, EBL\_\-TREEBANK, EBL\_\-TRAIN, EBL\_\-POSTPROCESS.]}


This command does all the processing needed to build a specialised
Regulus grammar from scratch. You will need to have defined at least 
the following config file parameters:
\begin{itemize}

\item {\tt ebl\_\-corpus} The training corpus, which should consist of Prolog records of the form {\tt sent(...)}.

\item {\tt ebl\_\-operationality} The file defining the operationality criteria.

\end{itemize}

In many applications, you will also want the following:

\begin{itemize}

\item {\tt ebl\_\-include\_\-lex} A file of ``include-lex'' definitions, which specify lexical entries to be included directly.

\item {\tt ebl\_\-regulus\_\-component\_\-grammar} The specialised grammar that will be loaded by the {\tt EBL\_LOAD} command, if it is not the default one.

\end{itemize}




See Section~\ref{Section:InvokingSpecialisation}.

\section{{\tt EBL\_\-GEMINI}}
\label{Section:EBL--GEMINI}

{\em [Compile current specialised Regulus grammar into Gemini form.]}

Compile current specialised Regulus grammar into Gemini form. Same as
the {\tt GEMINI} command, but for the specialised grammar. The base name of
the Gemini files produced is defined by the {\tt ebl\_\-gemini\_\-grammar} config
file entry.



See Section~\ref{Section:Gemini}.

\section{{\tt EBL\_\-GENERATION}}
\label{Section:EBL--GENERATION}

{\em [Do main generation EBL processing: equivalent to LOAD, EBL\_\-TREEBANK, EBL\_\-TRAIN, EBL\_\-POSTPROCESS, EBL\_\-LOAD\_\-GENERATION.]}

This command does all the processing needed to build a specialised
Nuance generation grammar from scratch. You will need to have defined at least 
the following config file parameters:
\begin{itemize}

\item {\tt ebl\_\-corpus} The training corpus, which should consist of Prolog records of the form {\tt sent(...)}.

\item {\tt ebl\_\-operationality} The file defining the operationality criteria.

\item {\tt generation\_\-grammar} The output file which will contain the generation grammar.

\end{itemize}

In many applications, you will also want the following:

\begin{itemize}

\item {\tt ebl\_\-include\_\-lex} A file of ``include-lex'' definitions, which specify lexical entries to be included directly.

\item {\tt ebl\_\-regulus\_\-component\_\-grammar} The specialised grammar that will be loaded by the {\tt EBL\_LOAD} command, if it is not the default one.

\item {\tt generation\_\-incremental\_\-deepening\_\-parameters} Settings to control incremental deepening during generation. A typical value is {\tt [0, 50, 50]}.

\end{itemize}




See Section~\ref{Section:InvokingSpecialisation}.

\section{{\tt EBL\_\-GRAMMAR\_\-PROBS}}
\label{Section:EBL--GRAMMAR--PROBS}

{\em [Create Nuance grammar probs training set from current EBL training set or grammar\_\-probs\_\-data file.]}

Convert the current EBL training set, defined by the {\tt ebl\_\-corpus}
config file entry, into a form that can be used as training data by
the Nuance {\tt compute-grammar-probs utility}. The output training data is
placed in the file defined by the {\tt ebl\_\-grammar\_\-probs} config file entry.



See Section~\ref{Section:PCFG}.

\section{{\tt EBL\_\-LOAD}}
\label{Section:EBL--LOAD}

{\em [Load current specialised Regulus grammar in DCG and left-corner form.]}

Load current specialised Regulus grammar in DCG and left-corner
form. Same as the {\tt LOAD} command, but for the specialised grammar. 

The specialised grammar is taken from the setting of the {\tt
ebl\_\-regulus\_\-component\_\-grammar} if it is defined.



See Section~\ref{Section:SpecialisedParsing}.

\section{{\tt EBL\_\-LOAD\_\-GENERATION Arg1}}
\label{Section:EBL--LOAD--GENERATIONArg1}

{\em [Compile and load designated version of current specialised Regulus grammar for generation.]}

Parameterised version of {\tt EBL\_\-LOAD\_\-GENERATION}: compile and
load the specialised generation grammar for the subdomain tag
{\tt $\langle$SubdomainTag$\rangle$}. This will be the file
{\tt $\langle$prefix$\rangle$\_\-specialised\_\-no\_\-binarise\_\-$\langle$SubdomainTag$\rangle$.regulus},
where {\tt $\langle$prefix$\rangle$} is the value of the config file entry
{\tt working\_\-file\_\-prefix}. The resulting compiled generation grammar
is placed in the file defined by the
{\tt generation\_\-grammar($\langle$SubdomainTag$\rangle$)} config file
entry. 

Note that {\tt EBL\_\-LOAD\_\-GENERATION $\langle$SubdomainTag$\rangle$}
places the compiled generation grammar in the same place as
{\tt LOAD\_\-GENERATION $\langle$SubdomainTag$\rangle$}.



See Section~\ref{Section:SpecialisedGeneration}.

\section{{\tt EBL\_\-LOAD\_\-GENERATION}}
\label{Section:EBL--LOAD--GENERATION}

{\em [Compile and load current specialised Regulus grammar for generation.]}

Compile and load the current specialised generation grammar. This will
be the file {\tt $\langle$prefix$\rangle$\_\-specialised\_\-no\_\-binarise\_\-default.regulus},
where {\tt $\langle$prefix$\rangle$} is the value of the config file
entry {\tt working\_\-file\_\-prefix}. The resulting compiled
generation grammar is placed in the file defined by the {\tt
generation\_\-rules} config file entry.

Note that {\tt EBL\_\-LOAD\_\-GENERATION} places the compiled
generation grammar in the same place as LOAD\_\-GENERATION.



See Section~\ref{Section:SpecialisedGeneration}.

\section{{\tt EBL\_\-MODE}}
\label{Section:EBL--MODE}

{\em [Do EBL processing on input sentences.]}

Put the top-loop in a mode where it shows the results of doing
EBL-based processing on input sentences. For this to make sense,
you need to have loaded a general grammar and have a config file
entry for {\tt ebl\_\-operationality}.

When the top loop is in EBL mode, input sentences are parsed
and then subjected to EBL generalisation, using the rules in
{\tt ebl\_\-operationality}. The file is reloaded each time.
The learned rules are printed in schematic form, as they
are in the files output by the {EBL\_\-POSTPROCESS} and
related commands. Each rule is paired with the phrase used
to induce it.



See Section~\ref{Section:CommandLineOverview}.

\section{{\tt EBL\_\-NUANCE}}
\label{Section:EBL--NUANCE}

{\em [Compile current specialised Regulus grammar into Nuance GSL form.]}

Compile current specialised Regulus grammar into Nuance GSL form. Same
as the {\tt NUANCE} command, but for the specialised grammar. The
input is the file created by the {\tt EBL\_\-POSTPROCESS command}; the
output Nuance GSL grammar is placed in the file defined by the {\tt
ebl\_\-nuance\_\-grammar} config file entry.



See Section~\ref{Section:SpecialisedRecognition}.

\section{{\tt EBL\_\-POSTPROCESS}}
\label{Section:EBL--POSTPROCESS}

{\em [Postprocess results of EBL training into specialised Regulus grammar.]}

Create one or more specialised Regulus grammars out of the results produced by
the {EBL\_\-TRAIN} command. The grammars are created in two forms.
The file {\tt $\langle$prefix$\rangle$\_\-specialised\_\-no\_\-binarise\_\-$\langle$tag$\rangle$.regulus}
is the original one; the file {\tt $\langle$prefix$\rangle$\_\-specialised\_\-$\langle$tag$\rangle$.regulus}
has been subjected to a binarisation transformation, so that no rule has more than
two daughters. The binarised version is the one passed to the {\tt EBL\_\-NUANCE}
command, and is also the default file loaded by {\tt EBL\_\-LOAD}.



See Section~\ref{Section:InvokingSpecialisation}.

\section{{\tt EBL\_\-TRAIN}}
\label{Section:EBL--TRAIN}

{\em [Do EBL training on current treebank.]}

Takes the file built using the {\tt EBL\_\-TREEBANK} command and
performs EBL generalisation using operationality criteria defined
by {\tt ebl\_\-operationality}. The output needs to be processed further 
by the {\tt EBL\_\-POSTPROCESS} command.




See Section~\ref{Section:InvokingSpecialisation}.

\section{{\tt EBL\_\-TREEBANK}}
\label{Section:EBL--TREEBANK}

{\em [Parse all sentences in current EBL training set into treebank form.]}

Parse all sentences in current EBL training set, defined by the {\tt
ebl\_\-corpus} config file entry, to create a treebank file. Sentences that
fail to parse are printed out with warning messages, and a summary
statistic is produced at the end of the run. This is very useful for
checking where you are with coverage.



See Section~\ref{Section:InvokingSpecialisation}.

\section{{\tt ECHO\_\-OFF}}
\label{Section:ECHO--OFF}

{\em [Don't echo input sentences (default).]}

Switch off functionality to echo input text.
 



See Section~\ref{Section:RegulusBatch}.

\section{{\tt ECHO\_\-ON}}
\label{Section:ECHO--ON}

{\em [Echo input sentences (normally useful only in batch mode).]}

Switch on echoing of input text. In this mode, each input text string 
(as opposed to Regulus command) is printed out before being processed.
This is normally only useful when running in batch mode.



See Section~\ref{Section:RegulusBatch}.

\section{{\tt FEAT Arg1}}
\label{Section:FEATArg1}

{\em [Display information for specified feature.]}

If the argument is a feature in the currently loaded grammar,
print information showing the range of permitted values for 
that feature. For example:
\begin{verbatim}
>> FEAT agr
(Display information for specified feature)

Feature values for feature "agr": [[1,2,3],[masc,fem],[sg,pl]]
\end{verbatim}



See Section~\ref{Section:GrammarDebuggingCommands}.

\section{{\tt GEMINI}}
\label{Section:GEMINI}

{\em [Compile current Regulus grammar into Gemini form.]}

The grammar defined by the parameter {\tt regulus\_\-grammar} is
translated into Gemini form. The base name for the output grammar
file needs to be specified using the parameter {\tt gemini\_\-grammar}.



See Section~\ref{Section:Gemini}.

\section{{\tt GENERATE\_\-TRACE\_\-OFF}}
\label{Section:GENERATE--TRACE--OFF}

{\em [Switch off generation tracing (default).]}

In translation mode, switch off printing of the generation trace.



See Section~\ref{Section:GenerationTrace}.

\section{{\tt GENERATE\_\-TRACE\_\-ON}}
\label{Section:GENERATE--TRACE--ON}

{\em [Switch on generation tracing.]}

In translation mode, switch on printing of generation tracing. Each
example processed prints the tree for the generated target, together
with preference information. If multiple target sentences are
produced, the tree and preference information is printed for each one.



See Section~\ref{Section:GenerationTrace}.

\section{{\tt GENERATION}}
\label{Section:GENERATION}

{\em [Generate from parsed input sentences.]}

Run the system in ``generation mode''. Each input sentence is
analysed. If any parses are found, the first one is generated back
using the currently loaded generation grammar, showing all possible
generated strings. This is normally used for debugging the generation
grammar.



See Section~\ref{Section:CommandLineOverview}.

\section{{\tt HELP Arg1}}
\label{Section:HELPArg1}

{\em [Print help for commands whose name or description match the string.]}

The argument is matched against all commands and their short descriptions, and 
matching examples are displayed. Matching is not case-sensitive.

Example:
\begin{verbatim}
>> HELP trace
(Print help for commands whose name or description match the string)

6 commands matching "trace":

GENERATE_TRACE_OFF (Switch off generation tracing (default))
GENERATE_TRACE_ON (Switch on generation tracing)
INTERLINGUA_TRACE_OFF (Switch off interlingua tracing (default))
INTERLINGUA_TRACE_ON (Switch on interlingua tracing)
TRANSLATE_TRACE_OFF (Switch off translation tracing (default))
TRANSLATE_TRACE_ON (Switch on translation tracing)
\end{verbatim}



See Section~\ref{Section:CommandLineHelp}.

\section{{\tt HELP}}
\label{Section:HELP}

{\em [Print help for all commands.]}

Print all available commands, together with short descriptions of what
they do. It is usually preferable to restrict the search by giving an
argument to {\tt HELP}, since there are many commands.




See Section~\ref{Section:CommandLineHelp}.

\section{{\tt HELP\_\-CONFIG Arg1}}
\label{Section:HELP--CONFIGArg1}

{\em [Print help for config file entries whose name or description match the string.]}

Search for config file entries whose name matches the argument and
display them. Matching is not case-sensitive.

Example:
\begin{verbatim}
>> HELP_CONFIG tree 
(Print help for config file entries whose name or description match the string)

3 config file entries matching "tree":

alterf_treebank_file
ebl_treebank
ellipsis_classes_treebank_file
\end{verbatim}



See Section~\ref{Section:CommandLineHelp}.

\section{{\tt HELP\_\-RESPONSE\_\-OFF}}
\label{Section:HELP--RESPONSE--OFF}

{\em [Switch off help response in main loop (default off).]}

Switch off help processing for top-level inputs. This only makes
sense if help resources are loaded.



See Section~\ref{Section:HelpSystemCommandLine}.

\section{{\tt HELP\_\-RESPONSE\_\-ON}}
\label{Section:HELP--RESPONSE--ON}

{\em [Switch on help response in main loop (default off).]}

Switch on help processing for top-level inputs. This only makes sense
if help resources have been loaded using the {\tt LOAD\_\-HELP} command.
In that case, help processing is applied first, and the top 5 help
responses are printed together with trace information. After that,
normal processing is carried out.



See Section~\ref{Section:HelpSystemCommandLine}.

\section{{\tt INCREMENTAL\_\-TREEBANKING\_\-OFF}}
\label{Section:INCREMENTAL--TREEBANKING--OFF}

{\em [Don't try to reuse old treebank material (default on).]}

Invoking this command forces the {\tt EBL\_\-TREEBANK} command to reparse the
whole of the training corpus. By default, it attempts to reuse old analyses when
it considers that they should be reliable.



See Section~\ref{Section:IncrementalTreebanking}.

\section{{\tt INCREMENTAL\_\-TREEBANKING\_\-ON}}
\label{Section:INCREMENTAL--TREEBANKING--ON}

{\em [Try to reuse old treebank material when possible (default on).]}

When incremental treebanking is on (default), the {\tt
EBL\_\-TREEBANK} command attempts to reuse stored analyses of corpus
examples rather than parsing them again. It considers an analysis of a
sentence S safe a) if the only grammar rules that have changed since the
stored parse was created are lexical ones involving words not occuring
in S, b) the config file and the files it includes have not changed,
c) the analysis preferences have not changed.
 



See Section~\ref{Section:IncrementalTreebanking}.

\section{{\tt INIT\_\-DIALOGUE Arg1}}
\label{Section:INIT--DIALOGUEArg1}

{\em [Initialise the dialogue state, passing it the given argument.]}

In dialogue mode, initialises the dialogue state, passing it the
specified argument. For this to make sense, you need to have 
previously invoked the {\tt LOAD\_\-DIALOGUE} command, and 
the predicate {\tt initial\_\-dialogue\_\-state/2} needs to be
defined. This predicate will be called to obtain the new dialogue
state and perform any relevant initialisation. The argument
to {\tt INIT\_\-DIALOGUE} is passed as the first argument, and
the state is returned as the second argument.



See Section~\ref{Section:DialogueMode}.

\section{{\tt INIT\_\-DIALOGUE}}
\label{Section:INIT--DIALOGUE}

{\em [Initialise the dialogue state.]}

In dialogue mode, initialises the dialogue state. For this to make
sense, you need to have previously invoked the {\tt LOAD\_\-DIALOGUE}
command, and the predicate {\tt initial\_\-dialogue\_\-state/1} needs
to be defined. This predicate will be called to obtain the new
dialogue state.



See Section~\ref{Section:DialogueMode}.

\section{{\tt INTERLINGUA}}
\label{Section:INTERLINGUA}

{\em [Perform translation through interlingua.]}


Do translation through interlingua, i.e.\ by first applying
source-to-interlingua rules (from the file that {\tt
to\_\-interlingua\_\-rules} points to) and then interlingua-to-target
rules ((from the file that {\tt from\_\-interlingua\_\-rules} points
to). This applies both to interactive processing in translate mode,
and to batch processing using commands like {\tt TRANSLATE\_\-CORPUS},
{\tt TRANSLATE\_\-SPEECH\_\-CORPUS} and {\tt
TRANSLATE\_\-SPEECH\_\-CORPUS\_\-AGAIN}.



See Section~\ref{Section:Interlingua}.

\section{{\tt INTERLINGUA\_\-DEBUGGING\_\-OFF}}
\label{Section:INTERLINGUA--DEBUGGING--OFF}

{\em [Switch off interlingua debugging (default).]}

Converse of {\tt INTERLINGUA\_\-DEBUGGING\_\-ON}.



See Section~\ref{Section:InterlinguaDebugging}.

\section{{\tt INTERLINGUA\_\-DEBUGGING\_\-ON}}
\label{Section:INTERLINGUA--DEBUGGING--ON}

{\em [Switch on interlingua debugging.]}

Switch on interlingua debugging; relevant to translation applications
that use an interlingua checking grammar. In this mode, translations
that give rise to interlingua that is ill-formed according to the 
interlingua checking grammar are processed by subjecting the ill-formed
interlingua to all possible insertions, deletions and substitutions of 
a single interlingua element, until either a well-formed variant is
discovered or a timeout is exceeded. The first well-formed variant
found is displayed. 




See Section~\ref{Section:InterlinguaDebugging}.

\section{{\tt INTERLINGUA\_\-TRACE\_\-OFF}}
\label{Section:INTERLINGUA--TRACE--OFF}

{\em [Switch off interlingua tracing (default).]}

Converse of {\tt INTERLINGUA\_\-TRACE\_\-ON}.



\section{{\tt INTERLINGUA\_\-TRACE\_\-ON}}
\label{Section:INTERLINGUA--TRACE--ON}

{\em [Switch on interlingua tracing.]}

Relevant to translation applications using an interlingua checking
grammar.  If interlingua checking succeeds, print the interlingua
checking grammar's analysis tree.




\section{{\tt KILL\_\-NUANCE\_\-PARSERS}}
\label{Section:KILL--NUANCE--PARSERS}

{\em [Kill any outstanding nl-tool processes (may be necessary after doing NUANCE\_\-PARSER).]}

Every time you invoke the {\tt NUANCE\_\-PARSER} command, it starts up
a new {\tt nl-tool} process. This command allows you to shut down
all outstanding {\tt nl-tool} processes.




See Section~\ref{Section:NuanceParser}.

\section{{\tt LC}}
\label{Section:LC}

{\em [Use left-corner parser.]}


Sets the current parser back to the default left-corner parser. This is normally
used after invoking the {\tt DCG} or {\tt NUANCE\_\-PARSER} commands.



See Section~\ref{Section:LCParser}.

\section{{\tt LF\_\-POST\_\-PROCESSING\_\-OFF}}
\label{Section:LF--POST--PROCESSING--OFF}

{\em [Switch off semantic post-processing of LFs.]}


The grammar processing mechanism currently supports three main types
of semantics: linear, Almost Flat Functional (AFF) and RIACS. For
AFF and RIACS semantics, the original logical form produced by the
grammar needs to be post-processed.

This command switches off post-processing of LFs, making it possible
to examine the original logical form, and is in particular useful when
you suspect a post-processing bug.




See Section~\ref{Section:GrammarDebuggingCommandLine}.

\section{{\tt LF\_\-POST\_\-PROCESSING\_\-ON}}
\label{Section:LF--POST--PROCESSING--ON}

{\em [Switch on semantic post-processing of LFs (default).]}


Converse of {\tt LF\_\-POST\_\-PROCESSING\_\-OFF}.



See Section~\ref{Section:GrammarDebuggingCommandLine}.

\section{{\tt LINE\_\-INFO\_\-OFF}}
\label{Section:LINE--INFO--OFF}

{\em [Don't print line and file info for rules and lex entries in parse trees.]}

A typical parse tree printed without line info will look like this:
\begin{verbatim}
.MAIN
   utterance
      command
      /  verb lex(switch)
      |  onoff null lex(on)
      |  np
      |  /  lex(the)
      |  |  noun lex(light)
      \  \  null
\end{verbatim}



See Section~\ref{Section:ParseTrees}.

\section{{\tt LINE\_\-INFO\_\-ON}}
\label{Section:LINE--INFO--ON}

{\em [Print line and file info for rules and lex entries in parse trees (default).]}

A typical parse tree printed with line info will look like this:
\begin{verbatim}
.MAIN [TOY1_RULES:1-4]
   utterance [TOY1_RULES:5-9]
      command [TOY1_RULES:10-14]
      /  verb lex(switch) [TOY1_LEXICON:8-10]
      |  onoff null lex(on) [TOY1_LEXICON:24-25]
      |  np [TOY1_RULES:25-29]
      |  /  lex(the)
      |  |  noun lex(light) [TOY1_LEXICON:16-17]
      \  \  null

------------------------------- FILES -------------------------------

TOY1_LEXICON: d:/regulus/examples/toy1/regulus/toy1_lexicon.regulus
TOY1_RULES:   d:/regulus/examples/toy1/regulus/toy1_rules.regulus
\end{verbatim}



See Section~\ref{Section:ParseTrees}.

\section{{\tt LIST\_\-MISSING\_\-HELP\_\-DECLARATIONS}}
\label{Section:LIST--MISSING--HELP--DECLARATIONS}

{\em [Write out a list of lexical items that are not listed in targeted help declarations.]}

Relevant to applications that use targeted help; a grammar must be
loaded, and the parameters {\tt targeted\_\-help\_\-classes\_\-file} and {\tt
missing\_\-help\_\-class\_\-decls} need to be defined. The help classes file is
searched, and all lexical items in the grammar that are not defined
are listed in the missing decls file.



See Section~\ref{Section:HelpSystemClasses}.

\section{{\tt LOAD}}
\label{Section:LOAD}

{\em [Load current Regulus grammar in DCG and left-corner form.]}


Compile and load the Regulus grammar defined by the {\tt
regulus\_\-grammar} config file entry in DCG and left-corner form. If
the grammar files and the config file have not been modified since the
last invocation of the {\tt LOAD} command, left-corner compilation is
not performed, and the stored version of the compiled grammar is used.

If parse preferences and/or nbest preference files are defined, these
are also loaded. These files are specified by the parameters {\tt
parse\_\-preferences} and {\tt nbest\_\-preferences} respectively, and
can be also loaded using the {\tt LOAD\_\-PREFERENCES} command.





See Section~\ref{Section:LoadGrammar}.

\section{{\tt LOAD\_\-DEBUG}}
\label{Section:LOAD--DEBUG}

{\em [Load current Regulus grammar in DCG and left-corner form, including extra debugging rules in left-corner grammar.]}


Compile and load the Regulus grammar defined by the {\tt
regulus\_\-grammar} config file entry in DCG and left-corner form,
including extra rules useful for grammar debugging. This makes
parsing slightly slower.

When the grammar is loaded in this form, a top-level input of the form
\begin{verbatim}
<CategoryName> <Sentence> 
\end{verbatim} 
is treated as a request to parse {\tt $\langle$Sentence$\rangle$} as an instance of
{\tt $\langle$CategoryName$\rangle$}, printing out semantic and feature values. 
\ref{Figure:LOAD-DEBUG-example} shows an example using the Toy1 grammar.
\begin{figure}
\begin{verbatim}
>> LOAD_DEBUG

(...)

>> np the light
(Parsing with left-corner parser)

Analysis time: 0.00 seconds

Return value: [[device,light]]

Global value: []

Syn features: [sem_np_type=switchable\/dimmable,singplur=sing]

Parse tree:

np [TOY1_RULES:25-29]
/  lex(the)
\  noun lex(light) [TOY1_LEXICON:16-17]

------------------------------- FILES -------------------------------

TOY1_LEXICON: d:/regulus/examples/toy1/regulus/toy1_lexicon.regulus
TOY1_RULES:   d:/regulus/examples/toy1/regulus/toy1_rules.regulus
\end{verbatim} 
\caption{Example showing use of {\tt LOAD\_DEBUG}}
\label{Figure:LOAD-DEBUG-example}
\end{figure}



See Section~\ref{Section:LoadGrammar}.

\section{{\tt LOAD\_\-DIALOGUE}}
\label{Section:LOAD--DIALOGUE}

{\em [Load dialogue-related files.]}

Relevant to dialogue applications: compile the files defined by the
{\tt dialogue\_\-files} config file entry. These should at a minimum
define the predicates {\tt lf\_\-to\_\-dialogue\_\-move}, {\tt
initial\_\-dialogue\_\-state}, {\tt update\_\-dialogue\_\-state} and {\tt
abstract\_\-action\_\-to\_\-action} with appropriate arities.




See Section~\ref{Section:DialogueMode}.

\section{{\tt LOAD\_\-GENERATION Arg1}}
\label{Section:LOAD--GENERATIONArg1}

{\em [Compile and load current generator grammar, and store as designated subdomain grammar.]}

Compile and load the current generation grammar, defined by the
{\tt generation\_\-grammar} config file entry. The resulting compiled
generation grammar is placed in the file defined by the
{\tt generation\_\-grammar($\langle$Arg$\rangle$)} config file entry. This can
be useful if you are normally using grammar specialisation to build
the generation grammar.



See Section~\ref{Section:Generation}.

\section{{\tt LOAD\_\-GENERATION}}
\label{Section:LOAD--GENERATION}

{\em [Compile and load current generator grammar.]}

Compile and load the current generation grammar, defined by the {\tt
regulus\_\-grammar} or {\tt generation\_\-regulus\_\-grammar} config
file entry. The resulting compiled generation grammar is placed in the
file defined by the {\tt generation\_\-grammar} config file entry.



See Section~\ref{Section:Generation}.

\section{{\tt LOAD\_\-HELP}}
\label{Section:LOAD--HELP}

{\em [Load compiled material for targeted help.]}

Relevant to applications using targeted help. Loads the help resources
previously build by invoking the {\tt COMPILE\_\-HELP} command.




See Section~\ref{Section:HelpSystemCommandLine}.

\section{{\tt LOAD\_\-PREFERENCES}}
\label{Section:LOAD--PREFERENCES}

{\em [Load parse and N-best preference files.]}


Load parse preferences and/or nbest preference files if they are
defined. These files are specified by the parameters {\tt
parse\_\-preferences} and {\tt nbest\_\-preferences} respectively.



See Section~\ref{Section:ParsePreferences}.

\section{{\tt LOAD\_\-RECOGNITION Arg1}}
\label{Section:LOAD--RECOGNITIONArg1}

{\em [Load recognition resources: license manager, recserver, TTS and regserver, using specified port for Regserver.]}




See Section~\ref{Section:SpeechInput}.

\section{{\tt LOAD\_\-RECOGNITION}}
\label{Section:LOAD--RECOGNITION}

{\em [Load recognition resources: license manager, recserver, TTS and regserver.]}

Start Nuance speech resources to enable speech processing from the
Regulus command-line. The following are required: 
\begin{itemize}

\item The file {\tt \$REGULUS/scripts/run\_\-license.bat} needs to exist
and contain a valid invocation of the license manager. Typical
contents (not a real licence code) might be
\begin{verbatim}
nlm C:/Nuance/Vocalizer4.0/license.txt abc12-1234-a-ab12
\end{verbatim}

\item One of the parameters {\tt translation\_\-rec\_\-params} and {\tt
dialogue\_\-rec\_\-params} needs to exist, and have an appropriate
value which specifies the recognition package to use and the Nuance
parameters to pass the recogniser invocation. A typical value might be
\begin{verbatim}
[package=callslt_runtime(recogniser), grammar='.MAIN',
'rec.Pruning=1600', 'rec.DoNBest=TRUE', 'rec.NumNBest=6',
'rec.ConfidenceRejectionThreshold=0',
'ep.EndSeconds=1.5']).
\end{verbatim}

\item If the application is to use TTS, an appropriate invocation
to start Vocalizer must be supplied as the value of the parameter
{\tt tts\_\-command}. A typical value to start the English version
of Vocalizer 4.0 would be
\begin{verbatim}
'vocalizer -num_channels 1 -voice enhancedlaurie 
-voices_from_disk'
\end{verbatim}

\end{itemize}



See Section~\ref{Section:SpeechInput}.

\section{{\tt LOAD\_\-RECOGNITION\_\-GENERATION}}
\label{Section:LOAD--RECOGNITION--GENERATION}

{\em [Compile and load current generator grammar(s) for converting recognition results to other scripts.]}

Relevant to applications which use multiple parallel grammars, and
need to convert recognition results either to the {\em original
script} or to the {\em gloss script}. The parallel original script and
gloss script grammars first need to be compiled into generation
grammar form.  They must then be declared using one or both of the
parameters {\tt original\_\-script\_\-recognition\_\-generation\_\-rules} and
{\tt gloss\_\-recognition\_\-generation\_\-rules}. The command
{\tt LOAD\_\-RECOGNITION\_\-GENERATION} loads any parallel
generation grammars that may be declared.




See Section~\ref{Section:Generation}.

\section{{\tt LOAD\_\-SURFACE\_\-PATTERNS}}
\label{Section:LOAD--SURFACE--PATTERNS}

{\em [Load current surface patterns and associated files.]}

Relevant to applications which do surface parsing using Alterf; load
the Alterf surface pattern files. You can then parse in surface mode
using the {\tt SURFACE} command. The following config file entries
must be defined:
\begin{itemize}
\item {\tt surface\_\-patterns}

\item {\tt tagging\_\-grammar} 

\item {\tt target\_\-model} 

\item {\tt discriminants} 

\item {\tt surface\_\-postprocessing} 
\end{itemize}



See Section~\ref{Section:AlterfOverview}.

\section{{\tt LOAD\_\-TRANSLATE}}
\label{Section:LOAD--TRANSLATE}

{\em [Load translation-related files.]}


Load all translation-related files defined in the currently valid
config file. These consist of a subset of the following; the set of
files required depends on whether translation is interlingua-based or
direct, and whether translation is from source to target, from source
to interlingua, or from interlingua to target.

\begin{itemize} 

\item An interlingua checking grammar compiled into generation form,
defined by the {\tt interlingua\_\-structure} config file
entry. Required if translation is interlingua-based.

item An interlingua declarations file defined by the
{\tt interlingua\_\-declarations} config file entry. Required
if translation is interlingua-based.

\item One or more to\_\-interlingua rules files defined by the {\tt
to\_\-interlingua\_\-rules} config file entry. Required if
translation is interlingua-based.

\item One or more from\_\-interlingua rules files defined by the {\tt
from\_\-interlingua\_\-rules} config file entry.  Required if
translation is interlingua-based.

\item An ellipsis classes file (optional) defined by the
{\tt ellipsis\_\-classes} config file entry. If this is defined, you need to
compile it first using the {\tt COMPILE\_\-ELLIPSIS\_\-PATTERNS} command.

\item A generation grammar file (required, unless translation is from
source to interlingua) defined by the {\tt generation\_\-rules} config
file entry. This should be the compiled form of a Regulus grammar for
the target language. The compiled generation grammar must first be
created using the {\tt LOAD\_\-GENERATION} command.

\item A generation preferences file (optional) defined by the
{\tt generation\_\-preferences} config file entry.

\item A collocations file (optional) defined by the
{\tt collocation\_\-rules} config file entry.

\item An orthography rules file (optional) defined by the
{\tt orthography\_\-rules} config file entry.  

\item One or more transfer rules files defined by the
{\tt transfer\_\-rules} config file entry. This is only
required for direct (i.e.\ non-interlingua-based) translation
applications.
 
\end{itemize}

If the config file entries {\tt wavfile\_\-directory} and
{\tt wavfile\_\-recording\_\-script} are defined, implying that output
speech will be produced using recorded wavfiles, this command also
produces a new version of the file defined by
{\tt wavfile\_\-recording\_\-script}.



See Section~\ref{Section:TranslationMode}.

\section{{\tt MAKE\_\-TARGET\_\-GRAMMAR\_\-PROBS\_\-CORPUS Arg1}}
\label{Section:MAKE--TARGET--GRAMMAR--PROBS--CORPUSArg1}

{\em [Create Nuance grammar probs training set for given grammar from translation output.]}

Relevant to translation applications. Take the results held in the
file indicated by {\tt translation\_\-corpus\_\-results} and turn them into
a training file for Nuance PCFG tuning, using the argument as the
relevant Nuance grammar. Put the result in the file indicated by
{\tt target\_\-grammar\_\-probs}. Thus, for example, the call
\begin{verbatim}
MAKE_TARGET_GRAMMAR_PROBS_CORPUS .MAIN
\end{verbatim}
will create a file where, for each translation {\tt $\langle$Sent$\rangle$}
in {\tt translation\_\-corpus\_\-results}, the file {\tt target\_\-grammar\_\-probs}
will contain a record of the form
\begin{verbatim}
.MAIN <Sent>
\end{verbatim}



See Section~\ref{Section:PCFG}.

\section{{\tt MAKE\_\-TARGET\_\-SENT\_\-CORPUS}}
\label{Section:MAKE--TARGET--SENT--CORPUS}

{\em [Create sent-formatted corpus from translation output.]}

Relevant to translation applications. Take the results held in the
file indicated by {\tt translation\_\-corpus\_\-results} and turn them
into a sent-formatted file. Put the result in the file indicated by
{\tt target\_\-sent\_\-corpus}. Thus, for example, the call
\begin{verbatim}
MAKE_TARGET_SENT_CORPUS .MAIN
\end{verbatim}
will create a file where, for each translation {\tt $\langle$Sent$\rangle$}
in {\tt translation\_\-corpus\_\-results}, the file {\tt target\_\-sent\_\-corpus}
will contain a record of the form
\begin{verbatim}
sent(<Sent>).
\end{verbatim}



See Section~\ref{Section:PCFG}.

\section{{\tt NORMAL\_\-PROCESSING}}
\label{Section:NORMAL--PROCESSING}

{\em [Do normal processing on input sentences.]}

Switch sentence processing in the Regulus top-level back to the
default behaviour: the system attempts to parse the sentence using the
current parser. If successful, it prints relevant information, in
particular the logical form(s), the associated analysis tree(s),
and any associated preference information.




See Section~\ref{Section:CommandLineOverview}.

\section{{\tt NO\_\-COMPACTION}}
\label{Section:NO--COMPACTION}

{\em [Switch off compaction processing for Regulus to Nuance conversion.]}

Switch off the grammar compaction step at the end of Regulus-to-Nuance
compilation, as for example invoked by the {\tt NUANCE} command. 

You should not normally wish to do this, since compaction is almost
always beneficial and is very stable.



See Section~\ref{Section:UGToGSL}.

\section{{\tt NO\_\-ELLIPSIS\_\-PROCESSING}}
\label{Section:NO--ELLIPSIS--PROCESSING}

{\em [Unload any ellipsis processing rules that may be loaded.]}

Relevant to translation applications: remove any currently loaded
ellipsis rules. These rules will have been compiled by the command
{\tt COMPILE\_\-ELLIPSIS\_\-PATTERNS} and loaded by the command {\tt
LOAD\_\-TRANSLATE}.




See Section~\ref{Section:TranslationEllipsis}.

\section{{\tt NO\_\-INTERLINGUA}}
\label{Section:NO--INTERLINGUA}

{\em [Perform translation directly, i.e. not through interlingua.]}

Applies to translation applications: converse of the command {\tt
INTERLINGUA}, it sets the translation processing mode to perform
direct translation from source to target, i.e.\ not through the
interlingua. It follows that the current config file must define a
value for the parameter {\tt transfer\_\-rules}, which should point to
a file of direct translation rules.

This applies both to interactive processing when the TRANSLATE command
is in effect, and to batch processing using commands like
{\tt TRANSLATE\_\-CORPUS}, {\tt TRANSLATE\_\-SPEECH\_\-CORPUS} and
{\tt TRANSLATE\_\-SPEECH\_\-CORPUS\_\-AGAIN}.



See Section~\ref{Section:Interlingua}.

\section{{\tt NUANCE}}
\label{Section:NUANCE}

{\em [Compile current Regulus grammar into Nuance GSL form.]}


Compile current Regulus grammar into Nuance GSL form. You won't be
able to use this command in conjunction with a large general grammar,
since it currently runs out of memory during compilation --- this why
we need EBL. The {\tt NUANCE} command is useful for smaller Regulus
grammars, e.g. the Toy1 grammar.

The current Regulus grammar is defined by the {\tt regulus\_\-grammar}
config file entry, and the location of the generated Nuance grammar by
the {\tt nuance\_\-grammar} config file entry.



See Section~\ref{Section:UGToGSL}.

\section{{\tt NUANCE\_\-COMPILE}}
\label{Section:NUANCE--COMPILE}

{\em [Compile Nuance grammar into recogniser package.]}

Compile the generated Nuance grammar, defined by the
{\tt ebl\_\-nuance\_\-grammar} or {\tt nuance\_\-grammar} config file entry, into
a recognition package with the same name. This will be done using the
Nuance language pack defined by the {\tt nuance\_\-language\_\-pack} config
file entry and the extra parameters defined by the
{\tt nuance\_\-compile\_\-params} config file entry. Typical values for
these parameters are as follows: 
\begin{verbatim}
regulus_config(nuance_language_pack, 'English.America').
regulus_config(nuance_compile_params, 
               ['-auto_pron', '-dont_flatten']).  
\end{verbatim}



See Section~\ref{Section:GSLToNuance}.

\section{{\tt NUANCE\_\-COMPILE\_\-WITH\_\-PCFG}}
\label{Section:NUANCE--COMPILE--WITH--PCFG}

{\em [Compile Nuance grammar into recogniser package, first doing PCFG training.]}

First perform PCFG training on the generated Nuance grammar, defined
by the {\tt ebl\_\-nuance\_\-grammar} or {\tt nuance\_\-grammar} config file
entry. The training data is taken from the file defined by the
{\tt ebl\_\-grammar\_\-probs} config file entry.

Next, compile the PCFG-trained version of the Nuance grammar, produced
by the first step, into a recognition package with the same name. This
will be done using the Nuance language pack defined by the
{\tt nuance\_\-language\_\-pack} config file entry and the extra parameters
defined by the {\tt nuance\_\-compile\_\-params} config file entry. Typical
values for these parameters are as follows: 
\begin{verbatim}
regulus_config(nuance_language_pack, 'English.America').
regulus_config(nuance_compile_params, 
               ['-auto_pron', '-dont_flatten']).  
\end{verbatim}



See Section~\ref{Section:GSLToNuance}.

\section{{\tt NUANCE\_\-PARSER}}
\label{Section:NUANCE--PARSER}

{\em [Start new Nuance nl-tool process and use it as parser.]}


Start an {\tt nl-tool} process, and use it to do parsing. Any old {\tt nl-tool} processes
are first killed. The current config file needs to include either a 
{\tt dialogue\_\-rec\_\-params} declaration (for dialogue apps) or a {\tt translation\_\-rec\_\-params}
declaration (for speech translation apps); the declaration must 
contain definitions for {\tt 'package'} and {\tt 'grammar'}. The following is a 
typical example of a suitable declaration:

\begin{verbatim}
regulus_config(dialogue_rec_params,
               [package=calendar_runtime(recogniser), 
                grammar='.MAIN',
                'rec.Pruning=1600', 'rec.DoNBest=TRUE', 
                'rec.NumNBest=6']).
\end{verbatim}
Notes: 
\begin{itemize}
\item After {\tt NUANCE\_\-PARSER} is successfully invoked, {\tt nl-tool} is used for
ALL parsing, including batch processing with commands like {\tt TRANSLATE\_\-CORPUS}
and Prolog calls to {\tt parse\_\-with\_\-current\_\-parser/6}.
\item The Nuance parser only returns logical forms, not parse trees.
\end{itemize}



See Section~\ref{Section:NuanceParser}.

\section{{\tt PARSE\_\-HISTORY Args}}
\label{Section:PARSE--HISTORYArgs}

{\em [Show parse history for examples matching specified string.]}

Search the parsing history file created by the {\tt EBL\_\-MAKE\_\-TREEBANK}
command to find matching example. The argument is treated as a list of
words, which may optionally contain wildcards, and matching is performed
at the word (opposed to character) level. Each matching example is printed 
together with a date-stamp showing when it last produced a parse.
Here is a typical invocation:
\begin{verbatim}
>> PARSE_HISTORY i would * cheese
(Show parse history for examples matching specified string)

--- Read parsing history file (394 records) 
d:/call-slt/eng/generatedfiles/callslt_parsing_history.pl

Found 2 records matching pattern

2010-06-01_15-41-03 1 i would like the cheese plate
2010-06-01_15-41-06 1 i would like the macaroni cheese

\end{verbatim}



See Section~\ref{Section:ParseHistory}.

\section{{\tt PRINT\_\-TREE\_\-CATEGORIES\_\-OFF}}
\label{Section:PRINT--TREE--CATEGORIES--OFF}

{\em [Don't print categories in parse trees (default).]}

Converse of {\tt PRINT\_\-TREE\_\-CATEGORIES\_\-ON}.



See Section~\ref{Section:ParseTrees}.

\section{{\tt PRINT\_\-TREE\_\-CATEGORIES\_\-ON}}
\label{Section:PRINT--TREE--CATEGORIES--ON}

{\em [Print categories in parse trees.]}

When printing parse trees at top level, also show all the categories
in the tree. \ref{Figure:PRINT-TREE-CATEGORIES-example} shows an
example using the Toy1 grammar.
\begin{figure}
\begin{verbatim}
>> PRINT_TREE_CATEGORIES_ON
(Print categories in parse trees)

--- Performed command PRINT_TREE_CATEGORIES_ON, time = 0.02 seconds

>> switch on the light     
(Parsing with left-corner parser)

Analysis time: 0.00 seconds

Return value: [[action,switch],[device,light],
               [onoff,on],[utterance_type,command]]

Global value: []

Syn features: []

Parse tree:

.MAIN [TOY1_RULES:1-4]
   utterance [TOY1_RULES:5-9]
      command [TOY1_RULES:10-14]
      /  verb lex(switch) [TOY1_LEXICON:8-10]
      |  onoff null lex(on) [TOY1_LEXICON:24-25]
      |  np [TOY1_RULES:25-29]
      |  /  lex(the)
      |  |  noun lex(light) [TOY1_LEXICON:16-17]
      \  \  null

------------------------------- FILES -------------------------------

TOY1_LEXICON: d:/regulus/examples/toy1/regulus/toy1_lexicon.regulus
TOY1_RULES:   d:/regulus/examples/toy1/regulus/toy1_rules.regulus

Categories:

[('.MAIN':[]), 
 (command:[]), 
 (noun:[sem_np_type=switchable,singplur=sing]), 
 (np:[sem_np_type=switchable,singplur=sing]), 
 (onoff:[]), 
 (utterance:[]), 
 (verb:[obj_sem_np_type=switchable,singplur=sing,
        vform=imperative,vtype=switch])]
\end{verbatim} 
\caption{Example showing use of {\tt PRINT\_TREE\_CATEGORIES\_ON}}
\label{Figure:PRINT-TREE-CATEGORIES-example}
\end{figure}



See Section~\ref{Section:ParseTrees}.

\section{{\tt PRINT\_\-TREE\_\-SUMMARY\_\-OFF}}
\label{Section:PRINT--TREE--SUMMARY--OFF}

{\em [Don't print summary versions of parse trees (default).]}

Converse of {\tt PRINT\_\-TREE\_\-SUMMARY\_\-ON}.



See Section~\ref{Section:ParseTrees}.

\section{{\tt PRINT\_\-TREE\_\-SUMMARY\_\-ON}}
\label{Section:PRINT--TREE--SUMMARY--ON}

{\em [Print summary versions of parse trees.]}

When processing input sentences from the Regulus top-level, print a
summary of each parse tree. This is primarily useful if you need to
add {\tt tree\_\-includes\_\-structure} or {\tt tree\_\-doesnt\_\-include\_\-structure}
constraints in an EBL training corpus. \ref{Figure:PRINT-TREE-SUMMARY-example} shows an
example using the Toy1 grammar.
\begin{figure}
\begin{verbatim}
>> PRINT_TREE_SUMMARY_ON
(Print summary versions of parse trees)

--- Performed command PRINT_TREE_SUMMARY_ON, time = 0.00 seconds

>> switch on the light
(Parsing with left-corner parser)

Analysis time: 0.00 seconds

Return value: [[action,switch],[device,light],
               [onoff,on],[utterance_type,command]]

Global value: []

Syn features: []

Parse tree:

.MAIN [TOY1_RULES:1-4]
   utterance [TOY1_RULES:5-9]
      command [TOY1_RULES:10-14]
      /  verb lex(switch) [TOY1_LEXICON:8-10]
      |  onoff null lex(on) [TOY1_LEXICON:24-25]
      |  np [TOY1_RULES:25-29]
      |  /  lex(the)
      |  |  noun lex(light) [TOY1_LEXICON:16-17]
      \  \  null

------------------------------- FILES -------------------------------

TOY1_LEXICON: d:/regulus/examples/toy1/regulus/toy1_lexicon.regulus
TOY1_RULES:   d:/regulus/examples/toy1/regulus/toy1_rules.regulus

Summary:

('.MAIN' < 
 [(utterance < 
   [command<[verb<lex(switch),
                  onoff<lex(on),
                  np<[lex(the),
                      noun<lex(light),
                      empty_constituent]]])])
\end{verbatim} 
\caption{Example showing use of {\tt PRINT\_TREE\_SUMMARY\_ON}}
\label{Figure:PRINT-TREE-SUMMARY-example}
\end{figure}



See Section~\ref{Section:ParseTrees}.

\section{{\tt RANDOM\_\-GENERATE Arg1 Arg2}}
\label{Section:RANDOM--GENERATEArg1Arg2}

{\em [Randomly generate and print the specified number of sentences, with specified maximum depth.]}

Like {\tt RANDOM\_\-GENERATE Arg1 Arg2}, but use the second argument
to limit the maximum depth of the generated tree.




See Section~\ref{Section:RandomGenerationRegulus}.

\section{{\tt RANDOM\_\-GENERATE Arg1}}
\label{Section:RANDOM--GENERATEArg1}

{\em [Randomly generate and print the specified number of sentences.]}

Randomly generate valid sentences from the currently loaded
grammar. Here is an example using the Toy1 grammar:
\begin{verbatim}
>> RANDOM_GENERATE 5
(Randomly generate and print the specified number of sentences)
.....
are the fans on
is the fan off
dim the light
switch on the lights
are the lights in the living room in the living room switched off
\end{verbatim} 



See Section~\ref{Section:RandomGenerationRegulus}.

\section{{\tt RECOGNISE}}
\label{Section:RECOGNISE}

{\em [Take next loop input from live speech.]}

Assumed that recognition resources have been loaded using the {\tt
LOAD\_\-RECOGNITION} command; uses the current recogniser to 
perform recognition, then treats the 1-best recognition result 
as though it had been top-level text input. The {\tt RECOGNISE}
command can be used in any top-level mode.




See Section~\ref{Section:SpeechInput}.

\section{{\tt RELOAD\_\-CFG}}
\label{Section:RELOAD--CFG}

{\em [Reload current config file.]}

Reload the current Regulus config file, plus any files it may include.



See Section~\ref{Section:CommandLineIntro}.

\section{{\tt SET\_\-BATCH\_\-DIALOGUE\_\-FORMAT Arg1}}
\label{Section:SET--BATCH--DIALOGUE--FORMATArg1}

{\em [Set format for printing batch dialogue results. Default is "normal".]}

By default, the output of invoking {\tt BATCH\_\-DIALOGUE} and similar commands is to print all processing information. This command can be used to select other output formats, or to revert to the default format. The currently supported alternatives are the following:
\begin{itemize}

\item {\tt normal} Default format.

\item {\tt no\_paraphrases} Suppress printing of fields related to paraphrasing.

\item {\tt no\_datastructures} Suppress printing of all fields related to intermediate datastructures.

\end{itemize}



See Section~\ref{Section:SettingBatchDialogueFormat}.

\section{{\tt SET\_\-NBEST\_\-N Arg1}}
\label{Section:SET--NBEST--NArg1}

{\em [Set the maximum number of hypotheses used for N-best processing.]}

For offline speech processing commands, the parameters
{\tt translation\_\-rec\_\-params} and {\tt
dialogue\_\-rec\_\-params} control, among other things,
the number of N-best alternatives generated by Nuance.
For example, the value
\begin{verbatim}
[package=callslt_runtime(recogniser), grammar='.MAIN',
'rec.Pruning=1600', 'rec.DoNBest=TRUE', 'rec.NumNBest=6']).
\end{verbatim}
produces a maximum of 6 hypotheses.

This command makes it possible to reduce the number of N-best
hyptheses considered by language processing. Evidently, the number
cannot be increased.




See Section~\ref{Section:DialogueNBest}.

\section{{\tt SET\_\-NOTIONAL\_\-SPEAKER Arg1}}
\label{Section:SET--NOTIONAL--SPEAKERArg1}

{\em [Set notional name of speaker for dialogue processing..]}

In dialogue processing applications where words like ``I'' and ``me''
are used, it can be important to know the identity of the speaker; for
example, in the Calendar application, one could say ``When is my next
meeting?'' or ``Am I attending a meeting on Friday?'' When doing
regression testing, it is then useful to be able to set a notional
speaker, so that responses to time-dependent utterances stay stable.

The {\tt SET\_\-NOTIONAL\_\-SPEAKER} command makes it possible to set the identity 
of the notional speaker from the command-line. The argument should be an atom, e.g.
\begin{verbatim}
SET_NOTIONAL_SPEAKER manny
\end{verbatim}
The notional speaker can be retrieved using the predicate 
{\tt get\_\-notional\_\-time/1} in {\tt PrologLib/utilities.pl}.



See Section~\ref{Section:SettingDialogueContext}.

\section{{\tt SET\_\-NOTIONAL\_\-TIME Arg1}}
\label{Section:SET--NOTIONAL--TIMEArg1}

{\em [Set notional time for dialogue processing. Format = YYYY-MM-DD\_\-HH-MM-SS, e.g. 2006-12-31\_\-23-59-59.]}

In dialogue processing applications where expressions like ``today''
or ``next week'' are used, it is necessary to know the current
time. When doing regression testing, it is then useful to be able to
set a notional time, so that responses to time-dependent utterances
stay stable.

The {\tt SET\_\-NOTIONAL\_\-TIME} command makes it possible to set the identity 
of the notional time from the command-line. The format is 
YYYY-MM-DD\_HH-MM-SS, e.g.
\begin{verbatim}
SET_NOTIONAL_TIME 2010-08-04_15-17-55
\end{verbatim}
The notional time can be retrieved using the predicate 
{\tt get\_\-notional\_\-time/1} in {\tt PrologLib/utilities.pl}.



See Section~\ref{Section:SettingDialogueContext}.

\section{{\tt SET\_\-REGSERVER\_\-TIMEOUT Arg1}}
\label{Section:SET--REGSERVER--TIMEOUTArg1}

{\em [Set the time the system waits before starting up the Regserver.]}

When recognition resources are started by the {\tt
LOAD\_\-RECOGNITION} command, specify the number of
seconds to wait before starting the Regserver process.
The default value is 60.




See Section~\ref{Section:SpeechInput}.

\section{{\tt SPLIT\_\-SPEECH\_\-CORPUS Arg1 Arg2 Arg3 Arg4}}
\label{Section:SPLIT--SPEECH--CORPUSArg1Arg2Arg3Arg4}

{\em [Split speech corpus into in-coverage and out-of-coverage pieces with respect to the specified grammar. Arguments: $\langle$GrammarAtom$\rangle$, $\langle$CorpusId$\rangle$, $\langle$InCoverageCorpusId$\rangle$ $\langle$OutOfCoverageCorpusId$\rangle$.]}

Splits the speech translation corpus output file, defined by the {\tt
translation\_\-speech\_\-corpus($\langle$Arg2$\rangle$)} config file
entry, into 
\begin{itemize} 
\item an in-coverage part defined by a {\tt
translation\_\-speech\_\-corpus($\langle$Arg3$\rangle$)}
config file entry, and

\item an out-of-coverage part defined by a {\tt
translation\_\-speech\_\-corpus($\langle$Arg4$\rangle$)}
config file entry.  
\end{itemize}
Coverage is with respect to the top-level grammar {\tt
$\langle$GrammarName$\rangle$} (Arg1), which must be loaded.

Typical call: 
\begin{verbatim}
SPLIT_SPEECH_CORPUS .MAIN corpus2 in_coverage2 out_of_coverage2
\end{verbatim}



See Section~\ref{Section:SplittingCorpora}.

\section{{\tt SPLIT\_\-SPEECH\_\-CORPUS Arg1 Arg2 Arg3}}
\label{Section:SPLIT--SPEECH--CORPUSArg1Arg2Arg3}

{\em [Split default speech corpus into in-coverage and out-of-coverage pieces with respect to the specified grammar. Arguments: $\langle$GrammarAtom$\rangle$, $\langle$InCoverageCorpusId$\rangle$ $\langle$OutOfCoverageCorpusId$\rangle$.]}

Splits the speech translation corpus output file, defined by the {\tt
translation\_\-speech\_\-corpus} config file entry, into 
\begin{itemize} 
\item an in-coverage part defined by a {\tt
translation\_\-speech\_\-corpus($\langle$Arg2$\rangle$)}
config file entry, and

\item an out-of-coverage part defined by a {\tt
translation\_\-speech\_\-corpus($\langle$Arg3$\rangle$)}
config file entry.  
\end{itemize}
Coverage is with respect to the top-level grammar {\tt
$\langle$Arg1$\rangle$}, which must be loaded.

Typical call: 
\begin{verbatim}
SPLIT_SPEECH_CORPUS .MAIN in_coverage out_of_coverage
\end{verbatim}



See Section~\ref{Section:SplittingCorpora}.

\section{{\tt SPLIT\_\-SPEECH\_\-CORPUS training\_\-corpus Arg1 Arg2 Arg3}}
\label{Section:SPLIT--SPEECH--CORPUStraining--corpusArg1Arg2Arg3}

{\em [Split speech corpus into in-training and out-of-training pieces with respect to EBL training corpus. Arguments: $\langle$FromCorpusId$\rangle$, $\langle$InTrainingCorpusId$\rangle$ $\langle$OutOfTrainingCorpusId$\rangle$.]}




See Section~\ref{Section:SplittingCorpora}.

\section{{\tt STEPPER}}
\label{Section:STEPPER}

{\em [Start grammar stepper.]}




See Section~\ref{Section:GrammarDebuggingStepper}.

\section{{\tt STORE\_\-TRANSLATION\_\-TARGET\_\-VOCAB Arg1}}
\label{Section:STORE--TRANSLATION--TARGET--VOCABArg1}

{\em [Process Source -$\rangle$ Target output and store target vocabulary items in the predicate regulus\_\-preds:target\_\-vocabulary\_\-item.]}




See Section~\ref{Section:TranslationRegressionText}.

\section{{\tt SURFACE}}
\label{Section:SURFACE}

{\em [Use surface pattern-matching parser.]}

This assumes that surface pattern files have been loaded, using the
LOAD\_\-SURFACE\_\-PATTERNS command.



See Section~\ref{Section:AlterfOverview}.

\section{{\tt TRANSLATE}}
\label{Section:TRANSLATE}

{\em [Do translation-style processing on input sentences.]}


In this mode, the sentence is parsed using the current parser. If any parses are found, the first one is processed through translation and generation. Translation is performed using interlingual rules if the INTERLINGUA command has been applied, otherwise using direct transfer.



See Section~\ref{Section:CommandLineOverview}.

\section{{\tt TRANSLATE\_\-CORPUS Arg1}}
\label{Section:TRANSLATE--CORPUSArg1}

{\em [Process text translation corpus with specified ID.]}


Parameterised version of TRANSLATE\_\-CORPUS. Process the text mode
translation corpus with ID $\langle$Arg$\rangle$, defined by the
parameterised config file entry
translation\_\-corpus($\langle$Arg$\rangle$). The output file, defined
by the parameterised config file entry
translation\_\-corpus\_\-results($\langle$Arg$\rangle$), contains
question marks for translations that have not yet been judged. If
these are replaced by valid judgements, currently 'good', 'ok' or
'bad', the new judgements can be incorporated into the translation
judgements file (defined by the translation\_\-corpus\_\-judgements
config file entry) using the parameterised command
UPDATE\_\-TRANSLATION\_\-JUDGEMENTS $\langle$Arg$\rangle$.



See Section~\ref{Section:TranslationRegressionText}.

\section{{\tt TRANSLATE\_\-CORPUS}}
\label{Section:TRANSLATE--CORPUS}

{\em [Process text translation corpus.]}


Process the default text mode translation corpus, defined by the
translation\_\-corpus config file entry. The output file, defined by
the translation\_\-corpus\_\-results config file entry, contains
question marks for translations that have not yet been judged. If
these are replaced by valid judgements, currently 'good', 'ok' or
'bad', the new judgements can be incorporated into the translation
judgements file (defined by the translation\_\-corpus\_\-judgements
config file entry) using the command
UPDATE\_\-TRANSLATION\_\-JUDGEMENTS.



See Section~\ref{Section:TranslationRegressionText}.

\section{{\tt TRANSLATE\_\-PARSE\_\-TIMES Arg1}}
\label{Section:TRANSLATE--PARSE--TIMESArg1}

{\em [Print parse times for latest run on text translation corpus with specified ID.]}




See Section~\ref{Section:TranslationRegressionTextTimes}.

\section{{\tt TRANSLATE\_\-PARSE\_\-TIMES}}
\label{Section:TRANSLATE--PARSE--TIMES}

{\em [Print parse times for latest run on text translation corpus.]}




See Section~\ref{Section:TranslationRegressionTextTimes}.

\section{{\tt TRANSLATE\_\-SPEECH\_\-CORPUS Arg1}}
\label{Section:TRANSLATE--SPEECH--CORPUSArg1}

{\em [Process speech translation corpus with specified ID.]}

Parameterised version of TRANSLATE\_\-SPEECH\_\-CORPUS. Process speech mode
translation corpus, defined by the translation\_\-speech\_\-corpus($\langle$Arg$\rangle$)
config file entry. The output file, defined by the
translation\_\-speech\_\-corpus\_\-results($\langle$Arg) config file entry, contains
question marks for translations that have not yet been judged. If
these are replaced by valid judgements, currently 'good', 'ok' or
'bad', the new judgements can be incorporated into the stored
translation judgements file using the command
UPDATE\_\-TRANSLATION\_\-JUDGEMENTS\_\-SPEECH $\langle$Arg$\rangle$. A second output file,
defined by the translation\_\-corpus\_\-tmp\_\-recognition\_\-judgements($\langle$Arg$\rangle$)
config file entry, contains "blank" recognition judgements: here, the
question marks should be replaced with either 'y' (acceptable
recognition), or 'n' (unacceptable recognition). Recognition
judgements can be updated using the UPDATE\_\-RECOGNITION\_\-JUDGEMENTS
$\langle$Arg$\rangle$ command.



See Section~\ref{Section:TranslationRegressionSpeech}.

\section{{\tt TRANSLATE\_\-SPEECH\_\-CORPUS}}
\label{Section:TRANSLATE--SPEECH--CORPUS}

{\em [Process speech translation corpus.]}

Process speech mode translation corpus, defined by the
translation\_\-speech\_\-corpus config file entry. The output file, defined
by the translation\_\-speech\_\-corpus\_\-results config file entry, contains
question marks for translations that have not yet been judged. If
these are replaced by valid judgements, currently 'good', 'ok' or
'bad', the new judgements can be incorporated into the stored
translation judgements file using the command
UPDATE\_\-TRANSLATION\_\-JUDGEMENTS\_\-SPEECH. A second output file, defined by
the translation\_\-corpus\_\-tmp\_\-recognition\_\-judgements config file entry,
contains "blank" recognition judgements: here, the question marks
should be replaced with either 'y' (acceptable recognition), or 'n'
(unacceptable recognition). Recognition judgements can be updated
using the UPDATE\_\-RECOGNITION\_\-JUDGEMENTS command.



See Section~\ref{Section:TranslationRegressionSpeech}.

\section{{\tt TRANSLATE\_\-SPEECH\_\-CORPUS\_\-AGAIN Arg1}}
\label{Section:TRANSLATE--SPEECH--CORPUS--AGAINArg1}

{\em [Process speech translation corpus with specified ID, using recognition results from previous run.]}

Parameterised version of TRANSLATE\_\-SPEECH\_\-CORPUS\_\-AGAIN. Process speech mode translation corpus, starting from the results saved from the most recent invocation of the TRANSLATE\_\-SPEECH\_\-CORPUS $\langle$Arg$\rangle$ command. This is useful if you are testing speech translation performance, but have only changed the translation or generation files. The output files are the same as for the TRANSLATE\_\-SPEECH\_\-CORPUS $\langle$Arg$\rangle$ command.



See Section~\ref{Section:TranslationRegressionSpeechRerun}.

\section{{\tt TRANSLATE\_\-SPEECH\_\-CORPUS\_\-AGAIN}}
\label{Section:TRANSLATE--SPEECH--CORPUS--AGAIN}

{\em [Process speech translation corpus, using recognition results from previous run.]}

Process speech mode translation corpus, starting from the results
saved from the most recent invocation of the TRANSLATE\_\-SPEECH\_\-CORPUS
command. This is useful if you are testing speech translation
performance, but have only changed the translation or generation
files. The output files are the same as for the
TRANSLATE\_\-SPEECH\_\-CORPUS command.



See Section~\ref{Section:TranslationRegressionSpeechRerun}.

\section{{\tt TRANSLATE\_\-TRACE\_\-OFF}}
\label{Section:TRANSLATE--TRACE--OFF}

{\em [Switch off translation tracing (default).]}




See Section~\ref{Section:TranslationTrace}.

\section{{\tt TRANSLATE\_\-TRACE\_\-ON}}
\label{Section:TRANSLATE--TRACE--ON}

{\em [Switch on translation tracing.]}




See Section~\ref{Section:TranslationTrace}.

\section{{\tt UNSET\_\-NOTIONAL\_\-SPEAKER}}
\label{Section:UNSET--NOTIONAL--SPEAKER}

{\em [Remove setting of notional name for dialogue processing.]}




See Section~\ref{Section:SettingDialogueContext}.

\section{{\tt UNSET\_\-NOTIONAL\_\-TIME}}
\label{Section:UNSET--NOTIONAL--TIME}

{\em [Use real as opposed to notional time for dialogue processing.]}




See Section~\ref{Section:SettingDialogueContext}.

\section{{\tt UPDATE\_\-DIALOGUE\_\-JUDGEMENTS Arg1}}
\label{Section:UPDATE--DIALOGUE--JUDGEMENTSArg1}

{\em [Update dialogue judgements file with specified ID from annotated dialogue corpus output.]}

Parameterised version of UPDATE\_\-DIALOGUE\_\-JUDGEMENTS. Update the
dialogue judgements file, defined by the dialogue\_\-corpus\_\-judgements
config file entry, from the output of the dialogue corpus output file
with ID $\langle$Arg$\rangle$, defined by the parameterised config file entry
dialogue\_\-corpus\_\-results($\langle$Arg$\rangle$). This command should be used after
editing the output file produced by the parameterised command
BATCH\_\-DIALOGUE $\langle$Arg$\rangle$. Editing should replace question marks by valid
judgements, currently 'good' or 'bad'.



See Section~\ref{Section:RegressionTestingDialogue}.

\section{{\tt UPDATE\_\-DIALOGUE\_\-JUDGEMENTS}}
\label{Section:UPDATE--DIALOGUE--JUDGEMENTS}

{\em [Update dialogue judgements file from annotated dialogue corpus output.]}

Update the dialogue judgements file, defined by the
dialogue\_\-corpus\_\-judgements config file entry, from the output of the
default text dialogue corpus output file, defined by the
dialogue\_\-corpus\_\-results config file entry. This command should be used
after editing the output file produced by the BATCH\_\-DIALOGUE
command. Editing should replace question marks by valid judgements,
currently 'good', or 'bad'.



See Section~\ref{Section:RegressionTestingDialogue}.

\section{{\tt UPDATE\_\-DIALOGUE\_\-JUDGEMENTS\_\-SPEECH Arg1}}
\label{Section:UPDATE--DIALOGUE--JUDGEMENTS--SPEECHArg1}

{\em [Update dialogue judgements file with specified ID from annotated speech dialogue corpus output.]}

Parameterised version of UPDATE\_\-DIALOGUE\_\-JUDGEMENTS\_\-SPEECH. Update the
dialogue judgements file, defined by the dialogue\_\-corpus\_\-judgements
config file entry, from the output of the dialogue corpus output file
with ID $\langle$Arg$\rangle$, defined by the parameterised config file entry
dialogue\_\-corpus\_\-results($\langle$Arg$\rangle$). This command should be used after
editing the output file produced by the parameterised command
BATCH\_\-DIALOGUES\_\-SPEECH $\langle$Arg$\rangle$. Editing should replace question marks by
valid judgements, currently 'good' or 'bad'.



See Section~\ref{Section:RegressionTestingDialogue}.

\section{{\tt UPDATE\_\-DIALOGUE\_\-JUDGEMENTS\_\-SPEECH}}
\label{Section:UPDATE--DIALOGUE--JUDGEMENTS--SPEECH}

{\em [Update dialogue judgements file from annotated speech dialogue corpus output.]}

Update the dialogue judgements file, defined by the
dialogue\_\-corpus\_\-judgements config file entry, from the output of the
default speech dialogue corpus output file, defined by the
dialogue\_\-speech\_\-corpus\_\-results config file entry. This command should
be used after editing the output file produced by the
BATCH\_\-DIALOGUES\_\-SPEECH command. Editing should replace question marks
by valid judgements, currently 'good', or 'bad'.



See Section~\ref{Section:RegressionTestingDialogue}.

\section{{\tt UPDATE\_\-RECOGNITION\_\-JUDGEMENTS Arg1}}
\label{Section:UPDATE--RECOGNITION--JUDGEMENTSArg1}

{\em [Update recognition judgements file from temporary translation corpus recognition judgements with specified ID.]}

Parameterised version of UPDATE\_\-RECOGNITION\_\-JUDGEMENTS. Update
recognition judgements file, defined by the
translation\_\-corpus\_\-recognition\_\-judgements config file entry, from the
temporary translation corpus recognition judgements file, defined by
the translation\_\-corpus\_\-tmp\_\-recognition\_\-judgements($\langle$Arg$\rangle$) config file
entry and produced by the TRANSLATE\_\-SPEECH\_\-CORPUS $\langle$Arg$\rangle$ or
TRANSLATE\_\-SPEECH\_\-CORPUS\_\-AGAIN $\langle$Arg$\rangle$ commands. This command should be
used after editing the temporary translation corpus recognition
judgements file. Editing should replace question marks by valid
judgements, currently 'y' or 'n'.



See Section~\ref{Section:TranslationJudgingProlog}.

\section{{\tt UPDATE\_\-RECOGNITION\_\-JUDGEMENTS}}
\label{Section:UPDATE--RECOGNITION--JUDGEMENTS}

{\em [Update recognition judgements file from temporary translation corpus recognition judgements.]}

Update recognition judgements file, defined by the
translation\_\-corpus\_\-recognition\_\-judgements config file entry, from the
temporary translation corpus recognition judgements file, defined by
the translation\_\-corpus\_\-tmp\_\-recognition\_\-judgements config file entry
and produced by the TRANSLATE\_\-SPEECH\_\-CORPUS or
TRANSLATE\_\-SPEECH\_\-CORPUS\_\-AGAIN commands. This command should be used
after editing the temporary translation corpus recognition judgements
file. Editing should replace question marks by valid judgements,
currently 'y' or 'n'.



See Section~\ref{Section:TranslationJudgingProlog}.

\section{{\tt UPDATE\_\-TRANSLATION\_\-JUDGEMENTS Arg1}}
\label{Section:UPDATE--TRANSLATION--JUDGEMENTSArg1}

{\em [Update translation judgements file from annotated translation corpus output with specified ID.]}

Parameterised version of UPDATE\_\-TRANSLATION\_\-JUDGEMENTS. Update
the translation judgements file, defined by the
translation\_\-corpus\_\-judgements config file entry, from the output
of the text translation corpus output file with ID
$\langle$Arg$\rangle$, defined by the parameterised config file entry
translation\_\-corpus\_\-results($\langle$Arg$\rangle$). This command
should be used after editing the output file produced by the
parameterised command TRANSLATE\_\-CORPUS
$\langle$Arg$\rangle$. Editing should replace question marks by valid
judgements, currently 'good', 'ok' or 'bad'.



See Section~\ref{Section:TranslationJudgingProlog}.

\section{{\tt UPDATE\_\-TRANSLATION\_\-JUDGEMENTS}}
\label{Section:UPDATE--TRANSLATION--JUDGEMENTS}

{\em [Update translation judgements file from annotated translation corpus output.]}


Update the translation judgements file,  defined by the translation\_\-corpus\_\-judgements config file entry, from the output of the default text translation corpus output file, defined by the translation\_\-corpus\_\-results config file entry. This command should be used after editing the output file produced by the TRANSLATE\_\-CORPUS command. Editing should replace question marks by valid judgements, currently 'good', 'ok' or 'bad'.



See Section~\ref{Section:TranslationJudgingProlog}.

\section{{\tt UPDATE\_\-TRANSLATION\_\-JUDGEMENTS\_\-CSV Arg1}}
\label{Section:UPDATE--TRANSLATION--JUDGEMENTS--CSVArg1}

{\em [Update translation judgements file from CSV version of annotated translation corpus output with specified ID.]}

Parameterised version of
UPDATE\_\-TRANSLATION\_\-JUDGEMENTS\_\-CSV. Update the translation
judgements file, defined by the translation\_\-corpus\_\-judgements
config file entry, from the output of the text translation corpus
output file with ID $\langle$Arg$\rangle$, defined by the
parameterised config file entry
translation\_\-corpus\_\-results($\langle$Arg$\rangle$). This command
should be used after editing the CSV version of the output file
produced by the parameterised command TRANSLATE\_\-CORPUS
$\langle$Arg$\rangle$. Editing should replace question marks in the
first column by valid judgements, currently 'good', 'ok' or 'bad'.



See Section~\ref{Section:TranslationJudgingCSV}.

\section{{\tt UPDATE\_\-TRANSLATION\_\-JUDGEMENTS\_\-CSV}}
\label{Section:UPDATE--TRANSLATION--JUDGEMENTS--CSV}

{\em [Update translation judgements file from CSV version of annotated translation corpus output.]}


Update the translation judgements file, defined by the
translation\_\-corpus\_\-judgements config file entry, from the output
of the default text translation corpus output file, defined by the
translation\_\-corpus\_\-results config file entry. This command
should be used after editing the CSV version of the output file
produced by the TRANSLATE\_\-CORPUS command. Editing should replace
question marks in the first column by valid judgements, currently
'good', 'ok' or 'bad'.



See Section~\ref{Section:TranslationJudgingCSV}.

\section{{\tt UPDATE\_\-TRANSLATION\_\-JUDGEMENTS\_\-SPEECH Arg1}}
\label{Section:UPDATE--TRANSLATION--JUDGEMENTS--SPEECHArg1}

{\em [Update translation judgements file from annotated speech translation corpus output with specified ID.]}

Parameterised version of UPDATE\_\-TRANSLATION\_\-JUDGEMENTS\_\-SPEECH. Update
the translation judgements file, defined by the
translation\_\-corpus\_\-judgements config file entry, from the output of
the speech translation corpus output file, defined by the
translation\_\-speech\_\-corpus\_\-results($\langle$Arg$\rangle$) config file entry. This
command should be used after editing the output file produced by the
TRANSLATE\_\-SPEECH\_\-CORPUS $\langle$Arg$\rangle$ or TRANSLATE\_\-SPEECH\_\-CORPUS\_\-AGAIN $\langle$Arg$\rangle$
command. Editing should replace question marks by valid judgements,
currently 'good', 'ok' or 'bad'.



See Section~\ref{Section:TranslationJudgingProlog}.

\section{{\tt UPDATE\_\-TRANSLATION\_\-JUDGEMENTS\_\-SPEECH}}
\label{Section:UPDATE--TRANSLATION--JUDGEMENTS--SPEECH}

{\em [Update translation judgements file from annotated speech translation corpus output.]}

Update the translation judgements file, defined by the
translation\_\-corpus\_\-judgements config file entry, from the output of
the speech translation corpus output file, defined by the
translation\_\-speech\_\-corpus\_\-results config file entry. This command
should be used after editing the output file produced by the
TRANSLATE\_\-SPEECH\_\-CORPUS or TRANSLATE\_\-SPEECH\_\-CORPUS\_\-AGAIN
command. Editing should replace question marks by valid judgements,
currently 'good', 'ok' or 'bad'.



See Section~\ref{Section:TranslationJudgingProlog}.

\section{{\tt UPDATE\_\-TRANSLATION\_\-JUDGEMENTS\_\-SPEECH\_\-CSV Arg1}}
\label{Section:UPDATE--TRANSLATION--JUDGEMENTS--SPEECH--CSVArg1}

{\em [Update translation judgements file from CSV version of annotated speech translation corpus output with specified ID.]}

Parameterised version of
UPDATE\_\-TRANSLATION\_\-JUDGEMENTS\_\-SPEECH\_\-CSV. Update the
translation judgements file, defined by the
translation\_\-corpus\_\-judgements config file entry, from the output
of the speech translation corpus output file, defined by the
translation\_\-speech\_\-corpus\_\-results($\langle$Arg$\rangle$)
config file entry. This command should be used after editing the CSV
version of the output file produced by the
TRANSLATE\_\-SPEECH\_\-CORPUS $\langle$Arg$\rangle$ or
TRANSLATE\_\-SPEECH\_\-CORPUS\_\-AGAIN $\langle$Arg$\rangle$
command. Editing should replace question marks in the first column by
valid judgements, currently 'good', 'ok' or 'bad'.



See Section~\ref{Section:TranslationJudgingCSV}.

\section{{\tt UPDATE\_\-TRANSLATION\_\-JUDGEMENTS\_\-SPEECH\_\-CSV}}
\label{Section:UPDATE--TRANSLATION--JUDGEMENTS--SPEECH--CSV}

{\em [Update translation judgements file from CSV version of annotated speech translation corpus output.]}

Update the translation judgements file, defined by the
translation\_\-corpus\_\-judgements config file entry, from the output
of the speech translation corpus output file, defined by the
translation\_\-speech\_\-corpus\_\-results config file entry. This
command should be used after editing the CSV version of the output
file produced by the TRANSLATE\_\-SPEECH\_\-CORPUS or
TRANSLATE\_\-SPEECH\_\-CORPUS\_\-AGAIN command. Editing should replace
question marks in the first column by valid judgements, currently
'good', 'ok' or 'bad'.



See Section~\ref{Section:TranslationJudgingCSV}.

\section{{\tt WAVFILES Arg1}}
\label{Section:WAVFILESArg1}

{\em [Show most recent N wavfiles recorded from speech input at top-level.]}


Show the last $\langle$Arg$\rangle$ wavfiles recorded using
recognition from the top-level (the RECOGNISE command). Each file is
displayed together with a timestamp and associated text. The
associated text is a transcription if one is available, or the
recognition result otherwise.

The files shown by this command can be used as top-level 
recorded speech input by typing
\begin{verbatim}
WAVFILE <Wavfile>
\end{verbatim}
e.g.
\begin{verbatim}
WAVFILE c:/Regulus/recorded_wavfiles/2008-04-24_22-36-03/utt05.wav
\end{verbatim}



See Section~\ref{Section:RecordedWavfiles}.

\section{{\tt WAVFILES}}
\label{Section:WAVFILES}

{\em [Show wavfiles recorded from speech input at top-level.]}


Show wavfiles recorded using recognition from the top-level
(the RECOGNISE command). Each file is displayed together with
a timestamp and associated text. The associated text is a 
transcription if one is available, or the recognition result
otherwise. 

The files shown by this command can be used as top-level 
recorded speech input by typing
\begin{verbatim}
WAVFILE <Wavfile>
\end{verbatim}
e.g.
\begin{verbatim}
WAVFILE c:/Regulus/recorded_wavfiles/2008-04-24_22-36-03/utt05.wav
\end{verbatim}



See Section~\ref{Section:RecordedWavfiles}.
 % 15.3.89


\newpage

\bibliography{lit}
\bibliographystyle{mhshort}
\newpage

\appendix
\section{PREDEFINED OPERATORS OF CEC}
\label{PredefinedOperators}
\begin{center}
\begin{tabular}{|lrrcc|} \hline
CEC operators & priority & fix & reserved & changable \\ \hline
\cec{=>} & 950 & xfx & + & - \\
\cec{=} & 700 & xfx & + & - \\
\cec{:} & 600 & xfy & + & - \\
\cec{and} & 850 & fy & + & - \\
\cec{@} & 50 & fx & + & - \\ \hline
\cec{->} & 1050 & xfy & - & - \\ 
\cec{cons}, \cec{op} & 950 & fx & - & - \\
\cec{in} & 935 & xfy & - & - \\
\cec{let} & 910 & fy & - & - \\
\cec{var} & 1100 & fx & - & - \\
\cec{<} & 700 & xfx & - & - \\
\cec{using} & 600 & xfx & - & - \\
\cec{module}, \cec{order} & 500 & fx & - & - \\
\cec{+} & 500 & yfx & - & - \\
\cec{for} & 400 & xfx & - & - \\
\cec{*} & 400 & yfx & - & - \\ \hline
\cec{:-}, \cec{-->} & 1200 & xfx & - & + \\
\cec{:-}, \cec{?-}  & 1200 & fx  & - & + \\
\cec{public}, \cec{multifile}, \cec{mode}, \cec{meta_predicate}, \cec{dynamic} & 1150 & fx & - & + \\
\cec{;}             & 1100 & xfy & - & + \\
\cec{,}             & 1000 & xfy & - & + \\
\cec{spy}, \cec{nospy}, \cec{\+} & 900 & fy & - & + \\
\cec{not} & 900 & fx & - & + \\
\cec{is}, \cec{=..}, \cec{==}, \cec{\==}, 
\cec{@<}, \cec{@>}, \cec{@=<}, \cec{@>=}, \cec{=:=}, \cec{=\=},
\cec{>}, \cec{=<}, \cec{>=} & 700 & xfx & - & + \\
\cec{-}, \cec{/\}, \cec{\/} & 500 & yfx & - & + \\
\cec{|} & 500 & xfy & - & + \\
\cec{-} & 500 & fx & - & + \\
\cec{/}, \cec{//}, \cec{<<}, \cec{>>} & 400 & yfx & - & + \\
\cec{mod} & 300 & xfx & - & + \\
\cec{^} & 200 & xfy & - & + \\ 
\cec{?} & 200 & fx & - & + \\
\hline
\end{tabular}
\end{center}

If an operator is {\em reserved}, it cannot me used in any signature. If 
an operator is not {\em changable}, it can be used in a signature, but 
its priority and fix may not be changed. 

\section{INSTALLATION OF CEC}
The CEC distribution tape contains one tar file, which
includes the CEC-System. The size of the CEC sources is about 763 kbytes, the whole system needs
about  7657 kbytes.
To install CEC execute the following steps:\bigskip

\noindent
Create a directory ``\cec{cec}'' where you want to file in CEC, 
and use the tar-command to read the tar file.

\noindent
Then change to the directory ``\cec{cec/obj/cec}'' and correct
the value of environment variables\bigskip

\cec{PROSPECTRA}

\cec{QPROLOG}\bigskip

\noindent
in the file ``\cec{Makefile}'' before calling\bigskip

{\bf make cec}\bigskip

\noindent
After successful creation of a executable \cec{cec} call\bigskip

{\bf make clean}

{\bf make install}\bigskip

\noindent
to complete the installation of CEC.\bigskip

\noindent
The executable file is located in the directory ``\cec{cec/bin}'' and
you can work with CEC as specified in the user manual, e.g. \bigskip

\verb+| ?-+ {\bf cd 'demo/cec/math'.}

\verb+| ?-+ {\bf in(abelianGroup,poly1).}

\verb+| ?-+ {\bf c.}

\end{document}
