\documentstyle[german]{article}

\textwidth 17cm
\addtolength{\baselineskip}{1.0ex}
\addtolength{\parsep}{0.5ex}
\oddsidemargin -0.4cm
\evensidemargin -0.4cm
\topmargin 0pt
\headheight 0pt
\headsep 0pt
\footheight 0pt
\footskip 40pt
\textheight 660pt

%--------------------------------- HACKS ------------------------------------

\newcommand{\inMath}[1]{\ifmmode{#1}\else${#1}$\fi}
\newcommand{\bold}[1]{{\bf #1}}

% \joinrel and \relbar for 11pt, used for "linear" \longrightarrow
\def\joinrel{\mathrel{\mkern-3.3mu}}
\def\relbar{\mathrel{\raise0.25pt\hbox{$\smash-$}}}

% Macros for preventing pagebreaks:
% No pagebreak appears between \connect and \disconnect
% ! The macros must be used in pairs
\catcode`\@=11
\def\connect{\global\advance\@beginparpenalty 10000\relax}
\def\disconnect{\global\advance\@beginparpenalty -10000\relax}
\catcode`\@=12

%------------------------- MACROS FOR TEXT STYLE -------------------------

% Environment for completion examples
% The script of an completion examples can be put between \begin{screen} 
% and \end{screen}. Blanks, carriage return, subscript, superscript, math 
% shift and parameter character will be typeset as in verbatim. 
% A frame will be put around the script

\catcode`\@=11
\def\blank{\ }
\def\obeyblanks{\catcode`\ \active}
{\obeyblanks\global\let =\blank}

\def\screen{\def\@halignto{}\@screen}

{\catcode`\^^M=\active % these lines must end with %
  \gdef\tablines{\catcode`\^^M\active \let^^M\@linecr}%
  \global\let^^M\par} % this is in case ^^M appears in a \write

\def\@screen{\begin{center}\leavevmode \hbox \bgroup\def\arraystretch{0.6}\def\tabcolsep{1mm}\footnotesize $\@makeother\_\@makeother\^\@makeother\$\@makeother\#\obeyblanks\frenchspacing\let\@acol\@tabacol 
   \let\@classz\@tabclassz
   \let\@classiv\@tabclassiv \tt \@noligs
   \let\\\@linecr\tablines\@array[c]{@{}|p{13cm}|@{}}\hline\@gobblecr\vskip-3pt\@gobblecr}

\def\endscreen{\hline\crcr\egroup\egroup{\catcode`\$=3}$\egroup\normalsize\vspace*{2ex}\pagebreak[0]\end{center}}

\def\@linecr{\cr}
\catcode`\@=12

% User input 
% Usage    : \user{#1}
% Arguments: 	1. User input
\def\user#1{{\bf #1}}

% Description of an user option
% Usage    : \userOption{#1}
% Arguments: 	1. Option
\def\userOption#1{``{\bf #1}''}

% Parts of a specification
% works like the verb-macro
% Usage: \cec{#1}
\def\doCECspecials{\do\\\do\$\do\&%
\do\#\do\^\do\^^K\do\_\do\^^A\do\%\do\~}
\def\cec{\begingroup \catcode``=13  \noligs 
\tt \let\do\makeother \doCECspecials\frenchspacing\obeyspaces\cecTerm}
\def\cecTerm#1{#1\endgroup}

% Filename
% Usage: \file{#1}
\newcommand{\file}[1]{\kw{#1}}

% Suffix of a file
% Usage: \suffix{#1}
\newcommand{\suffix}[1]{\kw{#1}}

% Reference-arrow
\newcommand{\refArrow}[0]{\mbox{$\rightarrow$}}


%-------------------------------- GRAMMARS --------------------------------

\catcode`\@=11

% Environment for grammar descriptions
% The grammar must be described between \begin{syntax} and \end{syntax}

\def\syntax{\def\@halignto{}\@syntax}
\def\@syntax{\bigskip\leavevmode \hbox \bgroup $\let\@acol\@tabacol 
   \let\@classz\@tabclassz
   \let\@classiv\@tabclassiv \let\\\@tabularcr\@array[c]{lcl}}
\def\endsyntax{\crcr\egroup\egroup $\egroup\vspace*{2ex}\pagebreak[0]}

% Non-Terminals
% Usage    : \nt{#1}
% Arguments: 1. Non-terminal symbol
\def\nt{\hbox{$<$}\begingroup\rm\catcode`\_=13\def\_{\_}\nonTerminal}
\def\nonTerminal#1{#1\endgroup\hbox{$>$}}

% Keywords
% Usage    : \kw{#1}
% Arguments: 1. Terminal symbol

\def\makeother#1{\catcode`#112\relax}

\def\doKWspecials{\do\ \do\\\do\$\do\&%
\do\#\do\^\do\^^K\do\_\do\^^A\do\%\do\~}

\def\kw{\begingroup \catcode``=13  \noligs 
\tt \let\do\makeother \doKWspecials\keyword}
\def\keyword#1{#1\endgroup}

\begingroup
\catcode``=13
\gdef\noligs{\let`=\lquote}
\endgroup

\def\lquote{{\kern\z@}`}
\catcode`\@=12

% Definition of a non-terminal
\def\IS{& ::= &}

% Alternative
\def\OR{\\ & & \ts}

% End of the definition of a non-terminal
\def\END{\\}

% Additional symbols on the right hand side of a definition
\def\AND{\\ & &}

\def\GETON{\\ & &}

% Alternative
\newcommand{\ts}{\mbox{$\mid$ \hspace{0.15ex}}}


%------------------------------- COMMAND DESCRIPTION ---------------------
% Definition of a command
% Arguments: 	1. Identifier of the command
%		2. Arguments of the command
\newcommand{\com}[2]{\kw{#1(}#2\kw{\hspace{0.1em}\kw{)}}}

% Identifier of the command
\newcommand{\comName}[1]{\kw{#1}}

% Argument of a command
% Arguments:	1. Argument of the command
\newcommand{\comArg}[1]{{\it #1}}

% Trennung between arguments 
\newcommand{\ad}{\kw{,}\,}

% Reference to an command definition
\newcommand{\comRef}[1]{\kw{#1}}

% Environment for command descriptions
\def\command[#1]{\topsep 0pt\connect\noindent #1\list{}{\listparindent 1.5em
 \itemindent\listparindent
 \rightmargin\leftmargin \parsep 0pt plus 1pt}\item[]\disconnect\hskip -\parindent}
\def\endcommand{\endlist\medskip}


%--------------------- SUBTERMS AND SUBTERMREPLACEMENT ----------------------

% Subterm
% Arguments: 	1. Term
%	     	2. Occurrence
\newcommand{\subterm}[2]{\inMath{{#1}/{#2}}}

% Subtermreplacement
% Arguments: 	1. The term in which the replacement takes place
%		2. Occurrence of the subterm that is to replace
%		3. Term that will take the place of the old subterm
\newcommand{\replace}[3]{\inMath{{#1}[{#2} \leftarrow {#3}]}}


%------------------------------ SUBSTITUTION ------------------------------

% Variable-Replacement
% Arguments:	1. Variable
% 		2. Term
\newcommand{\varRep}[2]{\inMath{{#1}/{#2}}}

% Composition of variable-replacements
\newcommand{\repSep}{,}

% Transformation of a composition of variable-replacements into a
% substitution
\newcommand{\subst}[1]{\inMath{\{{#1}\}}}

% Set of all substitutions
\newcommand{\subs}[1]{\inMath{{\rm SUB}_{#1}}}

% Subsumption
\newcommand{\subsumeq}{\inMath{\leq}}

% Domain-Operator
\newcommand{\domain}[1]{\inMath{{\cal D}({#1})}}

% Introduced-Operator
\newcommand{\introd}[1]{\inMath{{\cal I}({#1})}}

% Representation of a substitution as equation-system
\newcommand{\substeqrep}[1]{\inMath{[{#1}]}}

% Composition of two substitutions
% Arguments:	1. Substitution that will be applied first
%		2. Substitution 
\newcommand{\composesubst}[2]{\inMath{{#2}\circ{#1}}}

% Application of a substitution
% Arguments:    1. Substitution
%		2. Argument of application
\newcommand{\applysubst}[2]{\inMath{{#2}{#1}}}


%------------------------- EQUATIONS AND RULES -------------------------

% Equation
% Arguments:	1. Left side of the equation
%		2. Right side of the equation
\newcommand{\eq}[2]{\inMath{{#1} = {#2}}}

% Conditional equation
% Arguments:	1. Condition
%		2. Conclusion
% Examples: \condEq{\eq{a}{b}\condAnd\eq{c}{d}}{\eq{e}{f}}
%	    \condEq{\cond{\eq{a_1}{b_1}}{\eq{a_n}{b_n}}}{\eq{a}{b}}
\newcommand{\condEq}[2]{\inMath{{#1} \Rightarrow {#2}}}

% Condition of a conditional equation
% Arguments:	1. First equation of the condition
%		2. Last  equation of the condition
\newcommand{\cond}[2]{\inMath{{#1} \wedge \ldots \wedge {#2}}}

% Rule
% Arguments:	1. Left  side of the rule
%		2. Right side of the rule
\newcommand{\rewRule}[2]{\inMath{{#1} \rightarrow {#2}}}

% Condtional rule
% Arguments:    1. Condition
%		2. Conclusion
% Examples: \condRule{\eq{a}{b}\condAnd\eq{c}{d}}{\rewRule{e}{f}}
%	    \condEq{\cond{\eq{a_1}{b_1}}{\eq{a_n}{b_n}}}{\rewRule{a}{b}}
\newcommand{\condRule}[2]{\inMath{{#1} \Rightarrow {#2}}}

\newcommand{\condAnd}{\wedge}

% Left side of a rule with an given index
\newcommand{\ruleli}[1]{\inMath{l_{#1}}}

% Right side of a rule with an given index
\newcommand{\ruleri}[1]{\inMath{r_{#1}}}

% Rule of an given index
\newcommand{\rulei}[1]{\inMath{\ruleli{#1} \rightarrow \ruleri{#1}}}


%-------------------------------- REWRITING --------------------------------

\newcommand{\newarrow}[0]{-\hspace{-1.7mm}-\hspace{-2mm}-\hspace{-3mm}\succ}
\newcommand{\inewarrow}[0]{\prec\hspace{-3.5mm}-\hspace{-2mm}-\hspace{-2mm}-}

% Rewriting-Relations
% On-step-rewriting with one parameter (e.g. the rewriting system)
\newcommand{\rewRel}[2]{\inMath{\mathrel{{#1}_{#2}}}}
\newcommand{\rew}[1]{\rewRel{\longrightarrow}{#1}}
\newcommand{\irew}[1]{\rewRel{\longleftarrow}{#1}}
\newcommand{\rewqr}[1]{\rewRel{\newarrow}{#1}}
\newcommand{\irewqr}[1]{\rewRel{\inewarrow}{#1}}

% Transitive closure of one-step-rewriting with one parameter
\newcommand{\rewRelT}[2]{\inMath{\mathrel{\stackrel{+}{#1}_{#2}}}}
\newcommand{\rewT}[1]{\rewRelT{\longrightarrow}{#1}}
\newcommand{\irewT}[1]{\rewRelT{\longleftarrow}{#1}}
\newcommand{\rewqrT}[1]{\rewRelT{\newarrow}{#1}}
\newcommand{\irewqrT}[1]{\rewRelT{\inewarrow}{#1}}

% Reflexive, transitive closure of one-step-rewriting with one parameter
\newcommand{\rewRelRT}[2]{\inMath{\mathrel{\stackrel{*}{#1}_{#2}}}}
\newcommand{\rewRT}[1]{\rewRelRT{\longrightarrow}{#1}}
\newcommand{\irewRT}[1]{\rewRelRT{\longleftarrow}{#1}}
\newcommand{\rewqrRT}[1]{\rewRelRT{\newarrow}{#1}}
\newcommand{\irewqrRT}[1]{\rewRelRT{\inewarrow}{#1}}

\newcommand{\rewqrnf}[1]{\multIndRel{\newarrow}{}{nf}{}{#1}}
\newcommand{\irewqrnf}[1]{\multIndRel{\inewarrow}{}{nf}{}{#1}}
\newcommand{\nfrelqr}[1]{\rewqrnf{#1}\irewqrnf{#1}}

% Symmetrical, reflexive, transitive closure of one-step-rewriting with
% one parameter
\newcommand{\rewSRT}[1]{\inMath{\mathrel{\stackrel{*}{\longleftrightarrow}_{#1}}}}

% Symmetrical closure of one-step-rewriting
\newcommand{\equ}[1]{\mbox{$\mathrel{\leftrightarrow_{#1}}$}}

% Reflexiv, symmetrical and transitive closure of one-step-rewriting
\newcommand{\equs}[1]{\mbox{$\mathrel{\stackrel{\ast}{\leftrightarrow}_{#1}}$}}

% Normalform
\newcommand{\nf}[1]{\inMath{#1\hspace*{-0.2ex}\downarrow}}
\newcommand{\nfr}[1]{\inMath{#1\hspace*{-0.2ex}\downarrow}}
\catcode`\@=11
% "\condsub#1#2" yields "", if #2 is empty, and "#1{#2}" otherwise
\def\condsub#1#2{\c@ndsubx \c@ndsuba#2\c@ndsubb\c@ndsubc\c@ndsuba\c@ndsubb
    #1{#2}\c@ndsubc\c@ndsubd}
\def\c@ndsubx #1\c@ndsuba\c@ndsubb#2\c@ndsubc#3\c@ndsubd{#2}%
\catcode`\@=12
\newcommand{\multIndRel}[5]{\mathrel{\vphantom{\mathop{#1}\limits
    \condsub^{#2}\condsub_{#4}}\smash{\mathop{#1}\limits
    \condsub^{#2}\condsub_{#4}}\condsub^{#3}\condsub_{#5}}}
\newcommand{\rewr}[4]{\multIndRel{\longrightarrow}{#1}{#2}{#3}{#4}}
\newcommand{\irewr}[4]{\multIndRel{\longleftarrow}{#1}{#2}{#3}{#4}}
\newcommand{\srewr}[4]{\multIndRel{\longleftrightarrow}{#1}{#2}{#3}{#4}}
\newcommand{\eqrel}[2]{\multIndRel{\longleftrightarrow}{#1}{}{}{#2}}
\newcommand{\eqv}[4]{\multIndRel{\equiv}{#1}{#2}{#3}{#4}}
\newcommand{\nfrel}[1]{{\big\downarrow_{#1}}}

%------------------------------- COMPLETION ------------------------------

% Environment for the description of a inference rule for completion
% Examples: \begin{CRule} ... \end{CRule}
%	    \begin{CRule}[(cp), Computation of an critical pair] ... \end{CRule}
\catcode`\@=11
\def\CRule{\par\@ifnextchar[{\@namedRule}{}}
\def\@namedRule[#1,#2]{\noindent \makebox[11em][l]{\it #1} {\it #2}}
\def\endCRule{}

% Deduction Rule with additional condition
\def\ifdeducRule#1#2if#3{\[ \frac{#1}{#2},\ \ {\rm if\ }#3 \]}

% Simple deduction rule
\def\deducRule#1#2{\[ \frac{#1}{#2} \]}
\catcode`\@=12

% Proof ordering
\newcommand{\PO}{\inMath{>_{\cal P}}}

% Termination ordering
\def\TO>{>}

% Tripel
\newcommand{\tripel}[3]{\inMath{(#1,\ #2,\ #3)}}

% 
\newcommand{\stateEl}[2]{\inMath{#1 : #2}}


%------------------------- MATHEMATICAL CONSTRUCTS -------------------------

% Nat
\newcommand{\Nat}{\mbox{$I\hspace{-0.3em}N$}}

% Int
\newcommand{\Int}{\mbox{$Z\hspace{-0.5em}Z$}}

% Reals
\newcommand{\Real}{\mbox{$I\hspace{-0.3em}R$}}

% ... without ...
\newcommand{\without}[2]{\inMath{{#1} \backslash {#2}}}

% Power set
\newcommand{\powerset}[1]{\inMath{{\cal P}({#1})}}

% ... and ... are disjoint
\newcommand{\disjunct}[2]{\inMath{{#1}\cap{#2}=\emptyset}}

% Restriction to ...
\newcommand{\restrict}[1]{\inMath{|_{#1}}}

% not member of ...
\newcommand{\nin}{\inMath{{\not \in}}}

% Congruence relation
% Arguments:	1. Relation
\newcommand{\congruent}[1]{\mbox{$\,\equiv_{#1}\,$}}


% Logical operators
% Evaluation environment: math mode
\newcommand{\logand}{\;\wedge\;}
\newcommand{\logor}{\;\vee\;}
\newcommand{\implies}{\;\Longrightarrow\;}

% Symbols for definitions
% Evaluation environment: math mode
\newcommand{\dfiff}{\;:\Longleftrightarrow\;}
\newcommand{\dfeq}{\;:=\;}

% Set
\newcommand{\set}[1]{\inMath{\{{#1}\}}}

% ZF-Set
% Arguments:	1. Set
%		2. Predicate
\newcommand{\predset}[2]{\inMath{\{{#1}\;|\;{#2}\}}}

% QED
\newcommand{\qed}{\mbox{\kern2em}\hfill\llap{$\Box$}}

\newenvironment{spec}{\vspace{1ex}\par\addtolength{\parindent}{1.0cm}\obeylines\obeyspaces\tt}{\vspace{1ex}}

\newcommand{\cand}{\wedge}

\begin{document}

\section{Induktionsbeweise mit CEC}

\subsection{Spezifikationserweiterung}

Um Induktionsbeweise f"uhren zu k"onnen, wird die Spezifikation erweitert.
\begin{itemize}
\item F"ur jede Sorte $s$ wird ein Typpr"adikat\footnote{
      Pr"adikate werden hier als Operatoren kodiert, d.h.\ ein Pr"adikat
      \preddecl{\pi}{s_1 \times \ldots \times s_n} wird als Operator
      \opdecl{\pi}{s_1 \times \ldots \times s_n}{\sort{pred}} kodiert, und
      ein Atom \( \pi(t_1,\ldots,t_n) \) wird zu der Gleichung
      \( \pi(t_1,\ldots,tn) = \pred{true} \),
      wobei \sort{pred} eine neue Sorte und 
      \opdecl{\pred{true}}{}{\sort{pred}} ein neuer Operator ist.
      }
      \( \preddecl{\tp{s}}{s} \),
\item f"ur jede Untersortenbeziehung\footnote{
      Im CEC wird diese Gleichung f"ur die Injektion
      erzeugt, die bei der "Ubersetzung der Untersortenbeziehung 
      in eine many-sorted Spezifikation entsteht.
      }
      $s'<s$ wird eine bedingte Gleichung
      \[ \clause{\tp{s'}(x)}{\tp{s}(x)} \]
      und
\item f"ur jeden Operator \opdecl{\omega}{s_1 \times \ldots \times s_n}{s}
      wird eine Typgleichung
      \[ \clause{\tp{s_1}(x_1) \cand \ldots \cand \tp{s_n}(x_n)}
                {\tp{s}(\omega(x_1,\ldots,x_n))}
      \]
\end{itemize}
hinzugef"ugt.

Diese Erweiterung hat die folgenden Eigenschaften:
\begin{itemize}
\item Die Erweiterung ist persistent, d.h.\ die G"ultigkeit von Gleichungen
      in der alten Spezifikation wird nicht ver"andert, und
\item \( \tp{s}(t) \) ist genau dann beweisbar, wenn $t$ zu einem Grundterm 
      "aquivalent ist.
\end{itemize}


\subsection{Basissignatur}

In der Spezifikation mu"s eine Teilsignatur angegeben werden, genannt
Basissignatur, so da"s es zu jedem Grundterm in der urspr"unglichen
Signatur einen "aquivalenten Grundterm in der Basissignatur gibt. F"ur
das erzeugte TES mu"s sogar gelten, da"s sich jeder Grundterm auf einen
Basisgrundterm reduzieren l"a"st, also m"ussen alle Grundnormalformen
Basisterme sein.

Die Basissignatur wird mit Hilfe von Deklarationen der Form
\[\rm
\begin{array}{l@{\;:==\;}l}
<declaration> & <sort> {\tt generatedBy} <operator>,\ldots,<operator>
\end{array}
\]
angegeben.

Es mu"s zu jeder Sorte eine Obersorte geben, die eine Deklaration
besitzt.
\[\rm s \quad{\tt generatedBy}\; \ldots, Op, \ldots \]
bedeutet, da"s alle Operatordeklarationen f"ur $Op$ mit einer Zielsorte,
die Untersorte von $s$ ist, in der Basissignatur enthalten sind.
Im order-sorted Fall kann eine Signatur mehrere Deklarationen f"ur einen
Operator enthalten. Es ist hier m"oglich, f"ur einen Operator nur einen Teil
dieser Deklarationen in der Basissignatur zu haben.

Daraus ergibt sich f"ur den reinen many-sorted Fall, da"s es zu jeder
Sorte genau eine solche Deklaration geben mu"s, und die Operatoren in
der Basissignatur sind dann genau die in den Deklarationen aufgef"uhrten.

\medskip
{\bf Beispiel:}
\begin{verse}\tt
stack generatedBy empty, push.       \\
                                     \hfill\\
nestack < stack.                     \\
                                     \hfill\\
op empty : stack.                    \\
op push : (elem * stack -> nestack). \\
op top : (nestack -> stack).
\end{verse}
Die Basissignatur sieht dann so aus:
\begin{verse}\tt
nestack < stack.                     \\
                                     \hfill\\
op empty : stack.                    \\
op push : (elem * stack -> nestack). \\
\end{verse}

F"ur das erzeugte TES wird gefordert, da"s sich jeder Grundterm auf
einen Basisgrundterm reduzieren l"a"st.
Da"s die Forderung erf"ullt wird, wird bei der Vervollst"andigung durch
eine spezielle Strategie garantiert. Es werden nur Typgleichungen der
Form
\[ \clause{\tp{s_1}(x_1) \cand \ldots \cand \tp{s_n}(x_n)}
          {\tp{s}(\omega(x_1,\ldots,x_n))} \]
zu Regeln orientiert, deren Operatoren $\omega$ in der Basissignatur
liegen. Alle anderen Typgleichungen werden nichtoperational. 

{\em Beachte:}
\( \tp{s}(\omega(x_1,\ldots,x_n)) \)
ist ein order-sorted Term. In der many-sorted Form k"onnen noch
Injektionen oberhalb von $\omega$ stehen.


\subsection{Induktive Beweise}

Sei \clause{\Gamma}{\varphi} eine bedingte Gleichung, und
\varsOf{\clause{\Gamma}{\varphi}}=\eset{x_1,\ldots,x_n}
Definiere
\[ \tau(\clause{\Gamma}{\varphi}) 
   \dfeq \clause{\tp{s_1}(x_1) \cand \ldots \cand \tp{s_n}(x_n) \cand \Gamma}
                {\varphi}
\]

Eine bedingte Gleichung $C$ ist genau dann im initialen Modell g"ultig,
wenn $\tau(C)$ im freien Modell g"ultig ist.

{\em Beachte:}
$C$ ist nicht notwendig in allen minimalen Modellen g"ultig.

Eine bedingte Gleichung $C$ ist genau dann im freien Modell \T{D(X)} einer
Spezifikation $D$ g"ultig, wenn sich die deduktive Theorie\footnote{
Die deduktive Theorie einer Menge von bedingten Gleichungen ist die Menge der
{\em unbedingten} Gleichungen, die sich daraus ableiten lassen.
}
von $E$ durch Hinzuf"ugen von $C$ nicht "andert, formal

\[ \T{D(X)} \models C  \iff  \ThOf{E}=\ThOf{\union{D}{\eset{C}}} \]

{\em Beachte:}
$C$ ist dann nicht notwendig in allen Modellen g"ultig.

Die Bedingung
\( \ThOf{E}=\ThOf{\union{D}{\eset{C}}} \)
l"a"st sich mit Hilfe von CEC wie folgt nachweisen:
\begin{itemize}
\item $E$ vervollst"andigen. Dann ist \ThOf{E} durch Rewriting entscheidbar.
\item $C$ als neue Gleichung hinzuf"ugen.
\item vervollst"andigen, dabei werden alle bedingten Gleichungen nichtoperational
      gemacht. Treten dabei keine neuen unbedingten Gleichungen auf, so gilt
      \( \ThOf{E}=\ThOf{\union{D}{\eset{C}}} \),
      da weiterhin die gleichen Regeln vorhanden sind.
\item Tritt eine neue unbedingte Gleichung $\varphi$ auf, die nicht durch
      Rewriting eliminiert werden kann, so gilt
      \( \ThOf{E} \neq \ThOf{\union{D}{\eset{C}}} \),
      denn \( \varphi \in \ThOf{\union{E}{\eset{C}}} \),
      aber \( \varphi \not\in \ThOf{E} \).
\end{itemize}
Das Verfahren ist refutationsvollst"andig. Angenommen die
Vervollst"andigung terminiert nicht. Dann ist die Menge der
persistenten Regeln $R$ ein vollst"andiges TES f"ur \union{E}{\eset{C}},
$R$ "andert sich aber nicht, daher gilt
\( \ThOf{E}=\ThOf{\union{D}{\eset{C}}} \).


\subsection{Bedienung}

Um induktive Beweise mit dem CEC-System zu f"uhren sind folgende Schritte
notwendig.
\begin{enumerate}
\item {\em Deklarationen f"ur Basissignatur in der Spezifikation erg"anzen}.

\item {\em Spezifikationserweiterung einschalten}.\hfill\\
      Dazu das Prolog-Goal
      \begin{verse}
         {\tt domainConstraints:==on.}
      \end{verse}
      auswerten.
\item {\em Spezifikation einlesen}.\hfill\\
      Dabei wird die Spezifikation automatisch um Typisierungsinformationen
      erweitert.

\item {\em Vervollst"andigen}.\hfill\\
      Die Gleichungen der urspr"unglichen Spezifikation k"onnen wie zuvor
      behandelt werden. Die Typgleichungen werden entsprechend der in der
      Spezifikation angegebenen Basissignatur behandelt. Der Benutzer mu"s
      bei nichtoperationalen Gleichungen nur noch die Bedingung zum Super-
      positionieren ausw"ahlen. Ist die Basissignatur nicht korrekt gew"ahlt,
      tritt eine unbedingte Gleichung auf, die nichtoperational gemacht
      werden mu"s. Das System ist dann nicht vollst"andig.\footnote{
      Zur Zeit wird das nicht deutlich angezeigt.}

\item {\em Beweisen}.\hfill\\
      Es k"onnen mehrere bedingte Gleichungen simultan bewiesen werden.
      Die Kommandos lauten:
      \begin{verse}
        {\tt prove([ $E_1$, $\ldots$, $E_n$]).} \\
        {\tt prove($E$).}
      \end{verse}
      \( E_1, \ldots, E_n,E \) sind bedingte Gleichungen. 
      F"ur jede Gleichung $E$ wird \( \tau(E) \) zur Spezifikation hinzugef"ugt,
      dann wird die Vervollst"andigung gestartet.
      Alle bedingten Gleichungen werden automatisch nichtoperational,
      der Benutzer mu"s nur noch die Bedingung zum Superpositionieren w"ahlen.
      Die Wahl einer Bedingung
        \( \tp{s}(x) \)
      kann man als Wahl von $x$ als Induktionsvariable betrachten.
      \begin{itemize}
      \item Tritt eine unbedingte Gleichung auf, ist dies ein Fehlschlag, die
            Gleichungen gelten im initialen Modell nicht.
      \item Terminiert die Vervollst"andigung, sind die Gleichungen
            \( \tau(E) \) im freien Modell g"ultig. Sie bleiben als 
            nichtoperationale Gleichungen erhalten und werden bei weiteren
            Beweisen als Lemmata benutzt.
      \end{itemize}
\end{enumerate}


\subsection{Spezifikationen aus mehreren Modulen}

Es ist erlaubt, Spezifikationen wie bisher zu erweitern
oder zu kombinieren.
Beim Erweitern darf die Basisspezifikation
nur um solche Operatoren erweitert werden, deren Zielsorte nicht in
der urspr"unglichen Spezifikation liegt.
Dadurch wird erzwungen, da"s die Erweiterung der Spezifikation
hinreichend vollst"andig ist.
Beim Kombinieren m"ussen die Deklarationen bez"uglich der
Basissignatur zueinander passen, d.h.\ solche Operatoren, deren
Zielsorte in beiden Signaturen enthalten ist, m"ussen entweder in
beiden Basissignaturen oder in keiner enthalten sein.

Induktive Beweise d"urfen erst in der endg"ultigen Spezifikation gef"uhrt
werden. Sonst besteht die M"oglichkeit, da"s die bein Beweisen hinzugef"ugten
nichtoperationalen Gleichungen beim vervollst"andigen der Erweiterung
neue Regeln erzeugen, die die Gleichungstheorie ver"andern.

\medskip
{\bf Beispiel:} Vervollst"andige
\begin{verse}\tt
module test1.            \\
                         \hfill\\
nat generatedBy 0,s.     \\
op 0 : nat.              \\
op s : (nat -> nat).     \\
\end{verse}

Beweise die Injektivit"at von {\tt s} im initialen Modell
\begin{verse}\tt
prove(s(x)=s(y)=>x=y).
\end{verse}

Nun vervollst"andige
\begin{verse}\tt
module test2 using test1. \\
                          \hfill\\
s(s(x))=s(x).             \\
\end{verse}

Dann wird die neue Gleichung
\begin{verse}\tt
s(x)=x
\end{verse}
erzeugt, obwohl sie nicht in {\tt test2} gilt.

\end{document}
