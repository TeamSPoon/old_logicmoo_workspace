\section{Introduction}

This manual describes the use of CEC,
a system for conditional equational completion. We assume familiarity
with well known notions in the term rewriting area e.g.
{\it
signature, term over a signature,
(conditional) term rewrite rule, applicability of a rewrite rule, 
\ldots}, cf. e.g. \cite{HO80}, \cite{Kap84}. The concepts are discussed only as far as
necessary to allow a meaningful interaction with the system. Theoretical
foundations of the concepts implemented in CEC are not discussed. 
Hints to the literature will be given whenever such concepts are introduced.

CEC can be used under Quintus-Prolog2.x.
An installation guide for CEC is to be found in the appendix of this manual.


There are some CEC-specific notions used throughout this manual. The input to 
CEC is called a 
{\it specification}. Also any later state of the initial
specification is called a specification. To distinguish the specification
which will be completed when invoking the completion procedure from any other
specification CEC presently knows about, this specification is called the 
{\it current specification}.

CEC includes a help-function. Type 
``\user{??}\nt{space}\user{.}'' to obtain a 
list of all available CEC-commands and type
``\user{?}\nt{FunctionName}\user{.}''
or ``\user{?(}\nt{FunctionName}\user{).}'' to 
get a short description for
CEC-command with the name {\it FunctionName}.
