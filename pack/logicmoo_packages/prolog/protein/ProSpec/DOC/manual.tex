\documentclass[fleqn]{scrartcl}

%{{{ Packages and such
\usepackage{goodies}
\usepackage{xspace}
\usepackage{named}
\usepackage{mathti}
\usepackage{theorems}

\newcommand{\prospec}{\textsf{ProSpec}\xspace}
\newcommand{\eqtrafo}{\textsf{EQTrafo}\xspace}
\newcommand{\sort}{\textsc{Sort}\xspace}
%}}}

\begin{document}
%{{{ Title
\bibliographystyle{named}

\title{The \prospec and \eqtrafo User's Manual}
\author{Peter Baumgartner and Bernd Thomas\\
Universit{\"a}t Koblenz-Landau\\
Rheinau 1\\
56075 Koblenz\\[2ex]
\texttt{\normalsize \{peter,bthomas\}@informatik.uni-koblenz.de}}
\date{Draft of \today}

\maketitle
%}}}

\section{Introduction}
\label{sec:Introduction}
%{{{ Introduction text
This document describes two tools developed at the AI Group at the
University of Koblenz. Both are intended  as frontends to be used in
conjunction with classical, first order clause logic theorem provers
(such as PROTEIN \cite{Baumgartner:Furbach:PROTEIN:CADE12:94}). 
\begin{description}
\item[\prospec:] \prospec takes as input a file consisting of a {\em sort
  declaration\/} and some {\em well-sorted formulas\/} (not neccessarily
  clauses), possibly containing the equality predicate. The output is 
  a set of unsorted clauses with equality. In brief, \prospec
  transforms sorted logic with equality into unsorted logic with
  equality in a semantics preserving way.


\item[\eqtrafo:] \eqtrafo takes as input a file consisting of clauses,
  possibly containing the equality predicate. The output is a set of
  clauses not containing the equality predicate. In brief \eqtrafo
  transforms clause logic with equality into clause logic without
  equality in a semantics preserving way.
\end{description}
Hence, typically a sorted specification with equality is proved using the
command sequence \prospec --- \eqtrafo --- Protein.
%}}}



\section{\protect{\prospec}}
\label{sec:Prospec}
%{{{ Introductory text
We will first describe the semantics of the transformation; the
subsequent \prospec user's guide will describe the concrete syntax and
how to run the program. 
%}}}

\subsection{Syntax and Semantics of the Transformation}
\label{sec:ProspecSemantics}
%{{{ Text
By  a {\em sorted
  specification\/} we mean a pair consisting of a {\em sort declaration\/}
  and a set of {\em well-sorted sorted formulas\/}.

Following standard terminology, a {\em sort declaration\/} for a signature
$\Sigma$ consists of 
\begin{itemize}
\item a set $\sort$ of symbols, called {\em sorts\/}, 
\item a mapping $s:Var \mapsto \sort$ which maps each variable of
  $\Sigma$ to a sort, 
\item a set of {\em term declarations\/} of the form $t \in S$, where $S \in
\sort$, 
\item a set of {\em predicate declarations\/} of the form
$P : S_1 \times \cdots \times S_n$, where $P$ is an
$n$-ary predicate symbol and $S_i \in \sort$,
\item and a set of {\em subsort declarations\/} of the form $S_1
  \sqsubseteq S_2$ where $S_1,S_2 \in \sort$.
\end{itemize}
For instance, using the convention that a variable $x$ with sort
$s$ (i.e. $s(x) = S$) is displayed as $x_S$, a valid term
declaration is $succ(succ(x_{Even})) \in Even$ (provided $Even \in
\sort$). An example for a subsort declaration would be $Nat
\sqsubseteq Integer$.

Although these notions all are standard, we gave them here in order to
formulate  restrictions apply in our case:
\begin{itemize}
\item The sort hierarchie must be a tree: whenever $S \sqsubseteq
   S_1$ and $S \sqsubseteq S_2$ then $S_1 = S_2$. 
\item For every $n-ary$ predicate symbol $P$ there is exactly one
  predicate declaration
  \begin{displaymath}
    P : S_1 \times \cdots \times S_n\quad \text{where $S_i
      \in \sort$, for $1 \leq i \leq n$.}
  \end{displaymath}
\item For every $n-ary$ function symbol $f$ there is exactly one
  term declaration
  \begin{displaymath}
    f(x^1_{S_1},\ldots,x^n_{S_n}) \in S \quad\text{where  $S,\,S_i
      \in \sort$, $x_i \neq x_j$, for $1
      \leq i,j \leq n$ and $i \neq j$.}
  \end{displaymath}
This is also noted as
\begin{displaymath}
  f: S_1 \times \cdots \times S_n \mapsto S
\end{displaymath}
This restriction is also called the {\em elementary\/} restriction of sorted
signatures, and the term declarations are also called {\em function
declarations\/}. 
\end{itemize}
By a {\em declaration\/} we mean either a predicate declaration or a
function declaration. The tree-restriction and the restrictions posed
in the declarations are quite severe. Nethertheless they turned out to
be useful in practice. 

The restrictions stated so far are precisely the same as in
\cite{Schmitt:Wernecke:TableauSortedLogics:89}. The reasons is that
our \prospec transformation coincides with the predicative encoding of
sorts of Schmitt and Wernecke, and hence has to make the same
assumptions about the sort declarations.
 
However, we want to be more general and allow
{\em polymorphism\/} in our declarations (again, this is standard,
see~\cite{Beierle}). 
The idea is to  allow a {\em variable\/}
in place where a {\em sort\/} is required. These {\em sort variables\/} are
taken from a set disjoint to the object variables. Further, we would
like to have declarations of the form $cons: U \times list(U) \mapsto
list(U)$. This motivates the following definition of a {\em sort\/}
\begin{definition}[Polymorphic Sorts]
Let $\cal B \neq \emptyset$  be a set of symbols, called {\em basic sorts\/}, let $\cal U$
be a set of variables, called {\em sort variables\/}, and let $\cal F$
be a set of function symbols 
of given arity, called {\em sort functions\/}. 
The set \sort is the smallest set consisting of sorts, where a sort
is defined inductively as follows:
\begin{itemize}
\item Every basic sort $B \in {\cal B}$ is a sort.
\item Every variable $U \in {\cal U}$ is a sort.
\item If $S_1,\ldots,S_n$ are sorts, then $F(S_1,\ldots,S_n)$ is a sort,
  where $F \in {\cal F}$.
\end{itemize}
\end{definition}

When taking this modified definition of \sort, the restrictions on
sort declarations posed above can remain unchanged, except 
that the following  restricion on subsort declarations has to be made:
\begin{description}
\item Every  {\em subsort declaration\/} is of the form $S_1
  \sqsubseteq S_2$ where $S_1,S_2 \in {\cal B}$.
\end{description}
With this restriction, the transitive closure of ``$\sqsubseteq$''
can be extended to a partial order on \sort (see~\cite{Beierle}).

Now, equality is declared as follows: ${} = {} : U \times U$, where
$U \in {\cal U}$. Notice that this is an ``unusual'' declaration of
equality, since it permits equations only on terms of the {\em same\/}
sort (the ``usual'' declaration would be ${} = {} : t \times t$
where $t$ is a basic sort and is the least upper bound of all basic
sorts). This restriction is motivated by the equality transformation
\eqtrafo, which is complete for this declaration only.

A further consequence of this declaration concerns semantics: for
interpretations it has
to be required that different sorts are mapped to domains which do not
have any elements in common. 

{\large\bf STOP 2.6.97}

% 
% Let $D[U_1,\ldots,U_n]$ denote
% a declarion containing exactly the different sort variables
% $U_1,\ldots,U_n$. By $D[U_1,\ldots,U_{i-1},S_i,U_{i+1},\ldots,U_n]$ we mean
% the declaration which is obtained from  $D[U_1,\ldots,U_n]$ by replacing
% all occurrences of $U_i$ by $S_i \in \sort$. 


It has to be clarified what a {\em well-sorted formula\/} is. In brief,
one has to take care when building formulas that all terms and atoms
obey the sort declarations: whenever according to the declaration part
a term of sort $S$ is expected, then only a term of sort $S$ or
subsort of $S$ may be used.  
Since this is completely standard we
will not repeat the definition here (See
e.g.~\cite{Oberschelp:OrderSortedPredicateLogic:TableauSortedLogics:89}).
The only thing to note is that our restrictions imply that each term
gets a {\em unique\/} sort\footnote{Our sorted signatures are
  {\em subterm-closed\/}: every subterm of a well-sorted term is
  well-sorted as well.}

With respect to  semantics, one has to use {\em sorted
interpretations\/}. Again, this is standard. The only thing to mention
here is that a sorted interpretation $\cal I$ maps each sort $S \in \sort$ to a
{\em non-empty\/} domain ${\cal I}(S)$. 

\prospec transforms a well-sorted specification $\Phi_S$ into an unsorted
clause set $\Phi$ in such a way that the semantics of the sorts is
preserved. Such transformations are well-known in the literature and
are referred to as {\em sort relativations\/}. Relativations are expected to
satisfy the following property:
\begin{description}
\item[Sort theorem:] $\Phi_S$ is satisfiable with some sort
  interpretation if and only if $\Phi$ is satisfiable in some unsorted
  interpretation.
\end{description}
Since the relativation carried out by \prospec satisfies the sort
theorem, we thus have a semantics for sorted specifications, by simply
taking the semantics of the relativised version.

Designing a sort relativation is a tradeoff between {\em expressibility\/}
and {\em efficiency\/}. In \prospec we decided to have rather severe
restrictions on sort declarations, which, on the other hand, allow for
a very simple transformation. Further, and more important, the
relativised clause set can 
be treated very efficiently, because all the reasoning stemming from
the sort information is carried out by {\em ordinary\/} unification. 
For instance, we do not allow the declaration that different sorts
have a {\em common\/} subsort, because this would either need a
non-standard unification algorithm or a much more complex
transformation. 
%}}}

\subsection{\prospec User's Guide}
\label{sec:ProspecUsersGuide}
%{{{ Text
%}}}





\bibliography{/home/peter/TEX/peter,/home/peter/TEX/logic}
\end{document}

%%% Local Variables: 
%%% folded-file: t
%%% mode: latex
%%% TeX-master: t
%%% TeX-parse-self: t
%%% TeX-auto-save: t
%%% End: 
